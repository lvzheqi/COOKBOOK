\author{Rossmann Tester}
\title{CookBook}
\begin{document}
\maketitle
\tableofcontents\pagebreak 
 
%----------------------------------------------------
\nopagebreak{ 
\section{SONSTIGE} 

\subsection{Curry Powder}

\index{Gewürzmischung@{Gewürzmischung\/}!Curry Powder@{Curry Powder\/}}
}{
\begin{multicols}{2}



1/4 Tasse gemahlener Zimt

2 1/2 EL gemahlener Koriander

2 1/2 EL gemahlenes Kurkuma

3 EL gemahlener Kreuzkümmel

1 EL gemahlener Bockshornkleesamen

2 EL trockene Senfkörner

2 EL gemahlener Cardamom

2 EL Knoblauchsalz

2 EL gemahlener Mohn

2 1/2 EL gemahlene Chili

2 1/2 EL gemahlener schwarzer Pfeffer

1 EL gemahlener Ingwer


\end{multicols}

Alles gut vermischen und sofort verwenden oder kühl und trocken lagern.


{\bfseries Menge:} 4 Portionen

{\bfseries Quelle:} www.pounddesigns.com 

} 

%----------------------
\nopagebreak{ 
\subsection{Firehouse Chili Powder}

\index{Gewürzmischung@{Gewürzmischung\/}!Firehouse Chili Powder@{Firehouse Chili Powder\/}}
}{
\begin{multicols}{2}



6 EL Paprikapulver

2 EL Kurkuma

1 EL getrocknete Chili

1 TL Kreuzkümmel

1 TL Oregano

1/2 TL Cayennepfeffer

1/2 TL Knoblauchpulver

1/2 TL Salz

1/2 TL gemahlene Nelken


\end{multicols}

Alles gut vermischen und sofort verwenden oder kühl und trocken lagern.


{\bfseries Menge:} 4 Portionen

{\bfseries Quelle:} www.pounddesigns.com 

} 

%----------------------
\nopagebreak{ 
\subsection{Garam Masala}

\index{Gewürzmischung@{Gewürzmischung\/}!Garam Masala@{Garam Masala\/}}
}{
\begin{multicols}{2}



1 EL gemahlener Koriander

2 TL schwarzer Pfeffer

2 TL gemahlener Cardamom

1 TL gemahlener Zimt

1 TL gemahlene Nelken


\end{multicols}

Alles gut vermischen und sofort verwenden oder kühl und trocken lagern.


{\bfseries Menge:} 4 Portionen

{\bfseries Quelle:} www.pounddesigns.com 

} 

%----------------------
\nopagebreak{ 
\subsection{Glühwein}

\index{Alkohol@{Alkohol\/}!Glühwein@{Glühwein\/}}
\index{Wein@{Wein\/}!Glühwein@{Glühwein\/}}
}{
\begin{multicols}{2}



1 Flasche Rotwein

2 EL Zucker

1 EL Zitronensaft

1 Nelken

1 Stückchen Stangenzimt


\end{multicols}

Zutaten vermischen und erhitzen aber nicht kochen.

Für Teepunsch: Einen 1/4 l frisch gebrühten schwarzen Tee zufügen.


{\bfseries Menge:} 4 Portionen

{\bfseries Quelle:} Muttis Kochbuch 

} 

%----------------------
\nopagebreak{ 
\subsection{Italienische Kräuter}

}{
\begin{multicols}{2}



Oregano

Basilikum

Thymian

Rosmarin

Majoran

Salbei


\end{multicols}

Alles gut vermischen und sofort verwenden oder kühl und trocken lagern.


{\bfseries Quelle:} Packung italienische Kräuter 

} 

%----------------------
\nopagebreak{ 
\subsection{Mithi Lassi}

\index{Getränke@{Getränke\/}!Mithi Lassi@{Mithi Lassi\/}}
\index{Indisch@{Indisch\/}!Mithi Lassi@{Mithi Lassi\/}}
\index{Joghurt@{Joghurt\/}!Mithi Lassi@{Mithi Lassi\/}}
\index{P4@{P4\/}!Mithi Lassi@{Mithi Lassi\/}}
\index{Süß@{Süß\/}!Mithi Lassi@{Mithi Lassi\/}}
}{
\begin{multicols}{2}



500 g Joghurt

1/4 l Eiswasser

4 EL Zucker

Zerstoßenes Eis


\end{multicols}

Joghurt, Eiswasser und Zucker mit einem Schneebesen oder einem Mixer so lange verrühren, bis die Oberfläche schaumig wird.
Zerstoßenes Eis dazugeben und das Getränke sofort servieren oder kalt stellen.

Lassi ist ein sehr beliebtes und wegen seines guten Geschmacks
und hohen Nährwertes sehr verbreitetes Getränk in Indien.

Variante: Man kann zusätzlich frisches Obst (Bananen, Mango,
Erdbeeren) in den Mixer geben.


{\bfseries Menge:} 4 Portionen

{\bfseries Quelle:} www.kochfreunde.de 

} 

\pagebreak 
 
%----------------------------------------------------
\nopagebreak{ 
\section{SALAT} 

\subsection{24--Stunden--Salat}

\index{einfach@{einfach\/}!24--Stunden--Salat@{24--Stunden--Salat\/}}
\index{Salat@{Salat\/}!24--Stunden--Salat@{24--Stunden--Salat\/}}
\index{vorbereiten@{vorbereiten\/}!24--Stunden--Salat@{24--Stunden--Salat\/}}
}{
\begin{multicols}{2}



1 Glas Sellerie

6 Eier, hartgekocht

6 Scheiben Kochschinken

1 Dose Mais

1 mittelgroße Dose Ananas

1 Dose Pilze

1 großes Glas Miracel Whip

2 mittelgroße Stangen Lauch

100 g geriebener Käse


\end{multicols}

Zutaten abtropfen lassen, Eier in Scheiben schneiden, Kochschinken würfeln. In einer flachen Schüssel (am besten Glas) mit flachem Boden von unten nach oben schichten: Sellerie, Eier, Schinken, Mais, Ananas, Pilze. Miracel Whip darüberstreichen. Porree in sehr dünne Streifen schneiden, darüber streuen. Nach Geschmack Käse darüberstreuen. NICHT verrühren, 24 Stunden ziehen lassen.


{\bfseries Menge:} 4 Portionen

{\bfseries Inhalt pro Portion:} 3491~kJ, 31~g Eiweiß, 49~g Fett, 63~g Kohlenhydrate, 836~kcal, 5.23~BE

{\bfseries Quelle:} Internet 

} 

%----------------------
\nopagebreak{ 
\subsection{Bulgur--Tomaten--Salat Tabuleh}

\index{arabisch@{arabisch\/}!Bulgur--Tomaten--Salat Tabuleh@{Bulgur--Tomaten--Salat Tabuleh\/}}
\index{Bulgur@{Bulgur\/}!Bulgur--Tomaten--Salat Tabuleh@{Bulgur--Tomaten--Salat Tabuleh\/}}
\index{Tomaten@{Tomaten\/}!Bulgur--Tomaten--Salat Tabuleh@{Bulgur--Tomaten--Salat Tabuleh\/}}
\index{vegetarisch@{vegetarisch\/}!Bulgur--Tomaten--Salat Tabuleh@{Bulgur--Tomaten--Salat Tabuleh\/}}
\index{vorbereiten@{vorbereiten\/}!Bulgur--Tomaten--Salat Tabuleh@{Bulgur--Tomaten--Salat Tabuleh\/}}
}{
\begin{multicols}{2}



250 g Bulgur

500 g Tomaten

1 Gemüsezwiebel

3 Knoblauchzehen

1 Bund glatte Petersilie (oder mehr)

6 EL Olivenöl

1 Zitrone (Saft)

1 TL Kreuzkümmel, gemahlen

schwarzer Pfeffer


\end{multicols}

Bulgur besteht aus vorgekochtem, wieder getrockenetem und geschrotetem Weizen. Während früher der Bulgur der Hauptbestandteil dieses überaus populären Salates aus Syrien und dem Libanon war, verschiebt sich heute die Zusammensetzung zugunsten von immer mehr Petersilie.

Den Bulgur mit reichlich warmem Wasser (oder Gemüsebrühe) bedecken und etwa 20 Minuten ziehen lassen. Falls nötig noch etwas Wasser dazugeben. In der Zwischenzeit die Tomaten waschen, von den Stielansätzen befreien und in nicht zu kleine Würfel schneiden. Die Zwiebel schälen und ebenfalls in Würfel schneiden. Den Knoblauch schälen und durch die Knoblauchpresse drücken. Die Petersilie waschen, fein hacken. Alles in eine große Schüssel geben. Das Olivenöl mit dem Zitronensaft und dem Kreuzkümmel verrühren und über die Zutaten in der Schüssel gießen. Den Bulgur in ein Sieb geben, gut abtropfen lassen. Alles gut vermengen. Den Salat mit Salz und Pfeffer abschmecken und etwa 1 Stunde ziehen lassen.


{\bfseries Menge:} 4 Portionen

{\bfseries Quelle:} The Cooking of the Middle East 

} 

%----------------------
\nopagebreak{ 
\subsection{Champignonsalat}

\index{Champignons@{Champignons\/}!Champignonsalat@{Champignonsalat\/}}
\index{vegetarisch@{vegetarisch\/}!Champignonsalat@{Champignonsalat\/}}
\index{vorbereiten@{vorbereiten\/}!Champignonsalat@{Champignonsalat\/}}
}{
\begin{multicols}{2}



2 EL Olivenöl

2 EL Weißwein

500 g kleine Champignons

2 EL Olivenöl zusätzlich

2 TL Weißweinessig

2 TL Honig

2 TL geschroteter Senf


\end{multicols}

Öl und Wein in einer schweren Pfanne erhitzen. Champignons hinzufügen und bei mittlerer Hitze etwa 6 Minuten köcheln lassen, bis sie gar sind. In eine Servierschüssel geben und abkühlen lassen. Zusätzliches Öl, Essig, Honig, Senf und etwaige Flüssigkeit von den Champignons gut vermischen. Über die Champignons gießen und vorsichtig mischen. Zimmerwarm servieren.


{\bfseries Menge:} 4 Portionen

{\bfseries Quelle:} Anne Wilson: Köstliche Salate 

} 

%----------------------
\nopagebreak{ 
\subsection{Couscous--Salat mit Orangensoße und Chicorée}

\index{Chicoree@{Chicoree\/}!Couscous--Salat mit Orangensoße und Chicorée@{Couscous--Salat mit Orangensoße und Chicorée\/}}
\index{Couscous@{Couscous\/}!Couscous--Salat mit Orangensoße und Chicorée@{Couscous--Salat mit Orangensoße und Chicorée\/}}
\index{Imbiss@{Imbiss\/}!Couscous--Salat mit Orangensoße und Chicorée@{Couscous--Salat mit Orangensoße und Chicorée\/}}
\index{Orangen@{Orangen\/}!Couscous--Salat mit Orangensoße und Chicorée@{Couscous--Salat mit Orangensoße und Chicorée\/}}
\index{P1@{P1\/}!Couscous--Salat mit Orangensoße und Chicorée@{Couscous--Salat mit Orangensoße und Chicorée\/}}
\index{Vegetarisch@{Vegetarisch\/}!Couscous--Salat mit Orangensoße und Chicorée@{Couscous--Salat mit Orangensoße und Chicorée\/}}
}{
\begin{multicols}{2}



1/2 Orange (oder Saft)

Salz

1 TL Olivenöl

1 Messerspitze Harissa

1/2 Bund Petersilie

1 Chicorée

40 g Couscous (gekocht ca. 100 g)

etwas Schnittlauch


\end{multicols}

Die Orangenhälfte auspressen, den Saft mit Salz, Öl und Harissa verrühren. Petersilie und Chicorée in Streifen schneiden und mit dem Couscous unterheben, etwas durchziehen lassen. Mit Schnittlauchröllchen bestreuen.


{\bfseries Bemerkung:} Fett: 6 g 

{\bfseries Menge:} 1 pro Portion

{\bfseries Quelle:} Brigitte: Raffinierte Diät 2001 

} 

%----------------------
\nopagebreak{ 
\subsection{Einfacher Blattsalat mit karamelisierten Walnüssen}

\index{Salat@{Salat\/}!Einfacher Blattsalat mit karamelisierten Walnüssen@{Einfacher Blattsalat mit karamelisierten Walnüssen\/}}
\index{vegetarisch@{vegetarisch\/}!Einfacher Blattsalat mit karamelisierten Walnüssen@{Einfacher Blattsalat mit karamelisierten Walnüssen\/}}
\index{Walnüsse@{Walnüsse\/}!Einfacher Blattsalat mit karamelisierten Walnüssen@{Einfacher Blattsalat mit karamelisierten Walnüssen\/}}
}{
\begin{multicols}{2}



1 Kopf Blattsalat

75 g Walnusskerne

15 g Butter

2 EL Zucker

Salz

2 TL Dijonsenf

3 EL Himbeeressig

weißer Pfeffer

7 EL Walnussöl


\end{multicols}

Den Salat putzen und in kaltem Wasser waschen und trocken schleudern.

Die Walnusskerne bei mittlerer Hitze ohne zusätzliches Fett leicht rösten. Mit dem Zucker bestreuen und schmelzen lassen. Butter zugeben und umrühren. Auf einen leicht geölten Teller geben und abkühlen lassen. Grob hacken.

Senf, Himbeeressig mit etwas Salz und Pfeffer verrühren. Das Walnussöl dazugeben und zu einer glatten Vinaigrette verrühren. Den Salat mit der Soße beträufeln und mit den Walnusskernen bestreuen.


{\bfseries Menge:} 4 Portionen

{\bfseries Quelle:} www.vox.de: Schmeckt nicht, gibt's nicht 

} 

%----------------------
\nopagebreak{ 
\subsection{Fruchtiger Rotkohlsalat}

\index{Alkohol@{Alkohol\/}!Fruchtiger Rotkohlsalat@{Fruchtiger Rotkohlsalat\/}}
\index{Früchte@{Früchte\/}!Fruchtiger Rotkohlsalat@{Fruchtiger Rotkohlsalat\/}}
\index{Rotkohl@{Rotkohl\/}!Fruchtiger Rotkohlsalat@{Fruchtiger Rotkohlsalat\/}}
\index{vegetarisch@{vegetarisch\/}!Fruchtiger Rotkohlsalat@{Fruchtiger Rotkohlsalat\/}}
}{
\begin{multicols}{2}



6 getrocknete Feigen

100 ml Portwein (Tawny)

400 g Rotkohl

2 EL Rotweinessig

3 EL Aceto Balsamico

1 TL Zucker

Salz

schwarzer Pfeffer aus der Mühle

4 EL Walnussöl

2 EL Sonnenblumenöl

1 säuerlicher Apfel (Boskop)

3 Ananasringe


\end{multicols}

Die Feigen achteln und zugedeckt über Nacht im Portwein ziehen lassen. Rotkohl putzen, waschen, vierteln und ohne Strunk in feine Streifen schneiden. Mit etwas Salz geschmeidig kneten. Essig mit Zucker, Salz und Pfeffer verrühren, dann das Öl unterschlagen. Rotkohl mit der Sauce mischen und gut 1 Stunde ziehen lassen. Den geschälten, entkernten Apfel in feine Würfel, die abgetropften Ananasringe in schmale Spalten schneiden. Mitsamt den Feigen unter den Kohl heben. Den Salat eventuell mit Salz und Pfeffer nachwürzen. Der Rotkohlsalat schmeckt gut zu Entenbraten oder Gänsebrust.


{\bfseries Menge:} 4 Portionen

{\bfseries Quelle:} Winterliches Gemüse. Frisch und pfiffig 

} 

%----------------------
\nopagebreak{ 
\subsection{Frühlingsplatte mit Dips}

\index{Salat@{Salat\/}!Frühlingsplatte mit Dips@{Frühlingsplatte mit Dips\/}}
\index{vegetarisch@{vegetarisch\/}!Frühlingsplatte mit Dips@{Frühlingsplatte mit Dips\/}}
}{
\begin{multicols}{2}

\textit{Dips}



100 g Blauschimmelkäse

50 ml Milch

weißer Pfeffer

2 EL Obstessig

1 EL Schnittlauchröllchen

2 Nektarinen

150 g Dickmilch

2 EL Orangensaft

1 TL Curry

1/2 TL Honig

Salz

4 Frühlingszwiebeln

1 Bund Radieschen

2 EL Öl

Zucker

\textit{Salate}



250 g Champignons

1 TL Zitronensaft

1 TL Senf

300 g Möhren

300 g Kohlrabi

1 Handvoll junge Spinatblätter oder Feldsalat


\end{multicols}

Den Käse mit einer Gabel zerdrücken, dabei die Milch, etwas Pfeffer und 1 EL Essig zugeben. Schnittlauch darunterrühren. Die Nektarinen waschen, halbieren, entsteinen und fein würfeln. Dickmilch, Orangensaft, Curry, Honig und Salz daruntermischen. Frühlingszwiebeln und Radieschen waschen und putzen. Zwiebeln in Ringe, Radieschen in Scheiben teilen. Beides mit Öl und übrigem Essig anmachen und mit Salz, Pfeffer und Zucker würzen. Champignons putzen, in Scheiben teilen und mit Salz, Pfeffer, Zitronensaft und Senf vermischen. Möhren und Kohlrabi schälen und raspeln. Spinat putzen, abbrausen, trockentupfen und eine Platte damit belegen. Die Gemüse darauf anrichten und mit den Saucen servieren.


{\bfseries Menge:} 4 Portionen

{\bfseries Quelle:} Leicht genießen 

} 

%----------------------
\nopagebreak{ 
\subsection{Gebratener Spargelsalat mit Pestovinaigrette}

\index{Radicchio@{Radicchio\/}!Gebratener Spargelsalat mit Pestovinaigrette@{Gebratener Spargelsalat mit Pestovinaigrette\/}}
\index{Spargel@{Spargel\/}!Gebratener Spargelsalat mit Pestovinaigrette@{Gebratener Spargelsalat mit Pestovinaigrette\/}}
\index{Vegetarisch@{Vegetarisch\/}!Gebratener Spargelsalat mit Pestovinaigrette@{Gebratener Spargelsalat mit Pestovinaigrette\/}}
}{
\begin{multicols}{2}



1 kg weißer Spargel

Salz

Zucker

2 EL Öl

1 Knoblauchzehe

1 Bund Basilikum

20 g Pinienkerne

150 ml sehr gutes Olivenöl

1 Zitrone (Saft)

schwarzer Pfeffer

1 Radicchio

\bild{image/cookbook_1.jpg}


\end{multicols}

Spargel schälen, die Enden abschneiden und die Stangen in sehr dünne Scheiben schneiden, die Köpfe halbieren. Die Spargelscheiben im heißen Öl kräftig anbraten. Mit Salz, Pfeffer und kurz vor Ende einer Prise Zucker würzen. Dann den klein geschnittenen Radiccio unterheben, durchschwenken und auf eine Platte geben.

Die Pinienkerne in einer Pfanne leicht rösten. Inzwischen Knoblauch pellen und andrücken. Basilikum zupfen. Alles in die Küchenmaschine geben und fein zerhäckseln. Nun Olivenöl und Zitronensaft dazugeben, nochmals häckseln.
Vinaigrette über den Spargelsalat geben und servieren.


{\bfseries Menge:} 4 Portionen

{\bfseries Quelle:} www.vox.de: Schmeckt nicht, gibt's nicht 2006-05-24 

} 

%----------------------
\nopagebreak{ 
\subsection{Marinierte Zucchini mit Joghurtdressing}

\index{Joghurt@{Joghurt\/}!Marinierte Zucchini mit Joghurtdressing@{Marinierte Zucchini mit Joghurtdressing\/}}
\index{vegetarisch@{vegetarisch\/}!Marinierte Zucchini mit Joghurtdressing@{Marinierte Zucchini mit Joghurtdressing\/}}
\index{Zucchini@{Zucchini\/}!Marinierte Zucchini mit Joghurtdressing@{Marinierte Zucchini mit Joghurtdressing\/}}
}{
\begin{multicols}{2}



4 Stiele frische Minze

1 Knoblauchzehe

1 Msp. Chiliflocken

Salz

Pfeffer

400 g Zucchini

4 Frühlingszwiebeln

200 ml Joghurt

\bild{image/cookbook_2.jpg}


\end{multicols}

Minze fein hacken. Knoblauch pellen und pressen. Mit dem Joghurt verrühren.
Die Chiliflocken dazugeben und mit Salz und Pfeffer würzen.
Die Zucchini waschen und mit dem Küchenhobel längs in dünne Scheiben hobeln. Salzen und 20 Minuten in einer Schüssel ziehen lassen.
Frühlingszwiebeln in feine Ringe schneiden. Die Zucchini abtropfen lassen und mit den Zwiebeln und der Joghurtsauce vermischen.


{\bfseries Menge:} 4 Portionen

{\bfseries Quelle:} www.vox.de: Schmeckt nicht, gibt's nicht 2006-05-31 

} 

%----------------------
\nopagebreak{ 
\subsection{Nudel--Erbsen--Salat}

\index{Erbsen@{Erbsen\/}!Nudel--Erbsen--Salat@{Nudel--Erbsen--Salat\/}}
\index{Imbiss@{Imbiss\/}!Nudel--Erbsen--Salat@{Nudel--Erbsen--Salat\/}}
\index{P1@{P1\/}!Nudel--Erbsen--Salat@{Nudel--Erbsen--Salat\/}}
\index{Pasta@{Pasta\/}!Nudel--Erbsen--Salat@{Nudel--Erbsen--Salat\/}}
\index{Vorbereiten@{Vorbereiten\/}!Nudel--Erbsen--Salat@{Nudel--Erbsen--Salat\/}}
}{
\begin{multicols}{2}



1/2 Clementine

etwas Schnittlauch und Koriandergrün oder Petersilie

Salz

Currypulver

3 EL Gemüsebrühe

4 TL Salatcreme

2 Scheiben gekochter Schinken

50 g TK-Erbsen

50 g Nudeln (gekocht ca. 125 g)


\end{multicols}

Die Clementine klein schneiden, Kräuter hacken, beides mi Salz, Curry, Gemüsebrühe und Salatcreme verrühren, Fruchtfleisch dabei etwas zerdrücken. Den Schinken in dünne Streifen schneiden und mit den gefrorenen Erbsen und den gekochten Nudeln untermischen. Gut durchziehen lassen. Vor dem Essen mit Salz und Curry abschmecken.


{\bfseries Menge:} 1 Portion

{\bfseries Quelle:} Brigitte: Ideal-Diät 2002 

} 

%----------------------
\nopagebreak{ 
\subsection{Nudel--Schinken--Salat}

\index{Gewürzgurken@{Gewürzgurken\/}!Nudel--Schinken--Salat@{Nudel--Schinken--Salat\/}}
\index{Imbiss@{Imbiss\/}!Nudel--Schinken--Salat@{Nudel--Schinken--Salat\/}}
\index{Kresse@{Kresse\/}!Nudel--Schinken--Salat@{Nudel--Schinken--Salat\/}}
\index{P1@{P1\/}!Nudel--Schinken--Salat@{Nudel--Schinken--Salat\/}}
\index{Tomaten@{Tomaten\/}!Nudel--Schinken--Salat@{Nudel--Schinken--Salat\/}}
\index{Vorbereiten@{Vorbereiten\/}!Nudel--Schinken--Salat@{Nudel--Schinken--Salat\/}}
}{
\begin{multicols}{2}



30 g Vollkornnudeln (oder ca. 75 g gekochte)

Salz

2 EL Gemüsebrühe

frisch gemahlener Pfeffer

4 Tomaten

2 kleine Gewürzgurken

1 Scheibe gekochter Schinken (20 g)

1/2 Beet Kresse oder

ein paar Raukeblätter


\end{multicols}

Nudeln in Salzwasser nach Packungsangabe kochen, abgießen und abkühlen lassen (oder gekochte Nudeln verwenden). Salatcreme mit Gemüse-Hefebrühe, Salz und Pfeffer in einer Schüssel verrühren. Tomaten, Gurken und Schinken klein schneiden, mit den gekochten Nudeln und der Kresse zur Soße geben und mischen. Etwas durchziehen lassen und noch einmal abschmecken.


{\bfseries Bemerkung:} Fett: 5 g 

{\bfseries Menge:} 1 Portion

{\bfseries Quelle:} Brigitte: Mütter-Diät 2003 

} 

%----------------------
\nopagebreak{ 
\subsection{Radicchio-Salat mit Orangen}

\index{Orangen@{Orangen\/}!Radicchio-Salat mit Orangen@{Radicchio-Salat mit Orangen\/}}
\index{Radicchio@{Radicchio\/}!Radicchio-Salat mit Orangen@{Radicchio-Salat mit Orangen\/}}
\index{vegetarisch@{vegetarisch\/}!Radicchio-Salat mit Orangen@{Radicchio-Salat mit Orangen\/}}
}{
\begin{multicols}{2}



2 weiße Zwiebeln

2 große Orangen

4 EL Aceto Balsamico

Salz

Cayennepfeffer

2 Köpfe Radicchio

2 EL Haselnussöl

2 EL Speiseöl

25 g gehobelte Haselnüsse


\end{multicols}

Die Zwiebeln schälen, eine in dünne Ringe schneiden, die andere fein würfeln. Die Orangen ebenfalls schälen, dabei die pelzige Innenhaut entfernen. Mit einem scharfen Messer die Fruchtfilets aus den Trennhäuten lösen. Den Saft aus den Trennhäuten drücken, mit Aceto Balsamico, Salz und Cayennepfeffer verrühren, dann beide Ölsorten unterschlagen. Die Radicchio-Köpfe halbieren, waschen, gründlich trockentupfen und quer in 2 cm breite Streifen schneiden. Mit der Salatsauce und der gewürfelten Ziebel vermischen. Die Zwiebelringe und die Orangenfilets dekorativ auf vier Tellern anrichten, den Salat in die Mitte geben und mit gehobelten Haselnüssen bestreuen. Dazu passen dünne Scheiben von geröstetem Vollkornbaguette.


{\bfseries Menge:} 4 Portionen

{\bfseries Quelle:} Winterliche Gemüse. Frisch und pfiffig 

} 

%----------------------
\nopagebreak{ 
\subsection{Rettichsalat mit Harzer Käse}

\index{Käse@{Käse\/}!Rettichsalat mit Harzer Käse@{Rettichsalat mit Harzer Käse\/}}
\index{Rettich@{Rettich\/}!Rettichsalat mit Harzer Käse@{Rettichsalat mit Harzer Käse\/}}
\index{vegetarisch@{vegetarisch\/}!Rettichsalat mit Harzer Käse@{Rettichsalat mit Harzer Käse\/}}
}{
\begin{multicols}{2}



300 g Rettich

125 g Harzer Käse

4 EL Senf

8 EL Apfelsaft

Salz

weißer Pfeffer

1 EL Öl

150 g Eichblattsalat

1 Kästchen Kresse


\end{multicols}

Rettich schälen, je nach Stärke längs halbieren oder vierteln und in dünne Scheiben hobeln. Harzer Käse in kleine Würfel schneiden und mit dem Rettich locker vermischen. Senf mit Apfelsaft, Salz, Pfeffer und Öl verquirlen, über den Rettich und den Harzer Käse gießen und 10 Minuten durchziehen lassen. Eichblattsalat putzen, waschen, trockenschütteln, in mundgerechte Stücke zupfen und auf flache Tellern anrichten. Rettich und Harzer Käse samt Marinade darauf verteilen. Die Kresse mit der Schere abschneiden, im Sieb kurz waschen, trockentupfen und über den Salat streuen. Dazu passen Laugenbrezeln gut.


{\bfseries Menge:} 4 Portionen

{\bfseries Quelle:} Gemüse: Jung, frisch, unwiderstehlich 

} 

%----------------------
\nopagebreak{ 
\subsection{Rote-Bete-Salat mit Nüssen}

\index{Feldsalat@{Feldsalat\/}!Rote-Bete-Salat mit Nüssen@{Rote-Bete-Salat mit Nüssen\/}}
\index{Rote@{Rote\/}!Rote-Bete-Salat mit Nüssen@{Rote-Bete-Salat mit Nüssen\/}}
\index{Bete@{Bete\/}!Rote-Bete-Salat mit Nüssen@{Rote-Bete-Salat mit Nüssen\/}}
\index{vegetarisch@{vegetarisch\/}!Rote-Bete-Salat mit Nüssen@{Rote-Bete-Salat mit Nüssen\/}}
}{
\begin{multicols}{2}



800 g kleine rote Beten

2 weiße Zwiebeln

3 EL Aceto Balsamico

2 EL Rotweinessig

Zucker

weißer Pfeffer aus der Mühle

Salz

2 EL Sonnenblumenöl

3 EL Walnussöl

50 g gehackte Walnüsse

100 g Feldsalat


\end{multicols}

Die roten Beten waschen und in reichlich Salzwasser zugedeckt etwa 30 Minuten garen. Abgießen, kalt abschrecken und schälen. (Oder gleich die gekochten, geschälten roten Beten kaufen, aber keine sauer eingelegten!) Rote Beten in etwa 1 cm große Würfel, die geschälten Zwiebeln in schmale Streifen schneiden. Beide Essigsorten mit Salz, Zucker und Pfeffer verrühren, das Öl unterschlagen. Rote Beten und Zwiebeln in dieser Sauce 20 Minuten ziehen lassen. Den Feldsalat putzen, waschen, trockenschleudern und unter den Salat heben. Den Salat mit den Nüssen bestreut rasch servieren.

Variation: Ebenso kann man Selleriesalat zubereiten. Dann aber für die Sauce einen milden Weißweinessig nehmen.


{\bfseries Menge:} 4 Portionen

{\bfseries Quelle:} Winterliche Gemüse. Frisch und pfiffig. 

} 

%----------------------
\nopagebreak{ 
\subsection{Salat aus roten Linsen}

\index{Frischkäse@{Frischkäse\/}!Salat aus roten Linsen@{Salat aus roten Linsen\/}}
\index{Linsen@{Linsen\/}!Salat aus roten Linsen@{Salat aus roten Linsen\/}}
\index{Tomaten@{Tomaten\/}!Salat aus roten Linsen@{Salat aus roten Linsen\/}}
\index{vegetarisch@{vegetarisch\/}!Salat aus roten Linsen@{Salat aus roten Linsen\/}}
\index{Zwiebeln@{Zwiebeln\/}!Salat aus roten Linsen@{Salat aus roten Linsen\/}}
}{
\begin{multicols}{2}



Salz

250 g rote Linsen

1 Bund Lauchzwiebeln

3 Tomaten

3 EL Essig

1 EL Senf

3 EL Öl

einige Spritzer Tabasco

2 Becher körniger Frischkäse a 200g (alternativ: Nomadenkäse)


\end{multicols}

Eineinhalb Liter Salzwasser zum Kochen bringen, Linsen 3 Min. sprudelnd darin kochen. Abtropfen lassen.

Lauchzwiebeln putzen, waschen und in feine Ringe schneiden. Tomaten kleinschneiden. Essig mit Senf, Öl, Salz und Tabasco verrühren. Über die Salatzutaten gießen und 15 Minuten durchziehen lassen. Zum Essen abgetropften Frischkäse unterrühren.

Dazu passt Bauernbrot.

Als Vorspeise für 6 Personen, als Hauptgericht für 4 Personen.


{\bfseries Bemerkung:} Eiweiß: 20 g, Fett: 25 g, Kohlenhydrate: 32 g, kcal: 480 

{\bfseries Menge:} 6 Portionen

{\bfseries Quelle:} Brigitte Rezepte: Die 300 beliebtesten Sammelrezepte 

} 

%----------------------
\nopagebreak{ 
\subsection{Warmer Kartoffelsalat}

\index{Kartoffeln@{Kartoffeln\/}!Warmer Kartoffelsalat@{Warmer Kartoffelsalat\/}}
}{
\begin{multicols}{2}



750 g Kartoffeln

65 g durchwachsener Speck

1 Zwiebel

1 1/2 EL Mehl

1/4 l Brühe

Essig

Salz

Pfeffer


\end{multicols}

Die Kartoffeln kochen, pellen und in dünne Scheiben schneiden.

Die Speck- und Zwiebelwürfel so lange erhitzen, bis sie glasig sind. Das Mehl zugeben und nach kurzem Dünsten die Brühe auffüllen, gut durchkochen. Kräftig mit Essig, Salz, Pfeffer und nach Belieben ganz wenig Zucker abschmecken.

Die noch warmen Kartoffeln zufügen, vorsichtig unterrühren und nur kurz durchziehen lassen.


{\bfseries Bemerkung:} kJ: 1212, kcal: 290 

{\bfseries Menge:} 4 Portionen

{\bfseries Quelle:} Wir kochen gut 

} 

\pagebreak 
 
%----------------------------------------------------
\nopagebreak{ 
\section{VEGETARISCH} 

\subsection{Allgäuer Käsekrapfen auf Tomatenrahm}

\index{Backen@{Backen\/}!Allgäuer Käsekrapfen auf Tomatenrahm@{Allgäuer Käsekrapfen auf Tomatenrahm\/}}
\index{Käse@{Käse\/}!Allgäuer Käsekrapfen auf Tomatenrahm@{Allgäuer Käsekrapfen auf Tomatenrahm\/}}
\index{P2@{P2\/}!Allgäuer Käsekrapfen auf Tomatenrahm@{Allgäuer Käsekrapfen auf Tomatenrahm\/}}
\index{Pikant@{Pikant\/}!Allgäuer Käsekrapfen auf Tomatenrahm@{Allgäuer Käsekrapfen auf Tomatenrahm\/}}
\index{Tomaten@{Tomaten\/}!Allgäuer Käsekrapfen auf Tomatenrahm@{Allgäuer Käsekrapfen auf Tomatenrahm\/}}
\index{vegetarisch@{vegetarisch\/}!Allgäuer Käsekrapfen auf Tomatenrahm@{Allgäuer Käsekrapfen auf Tomatenrahm\/}}
}{
\begin{multicols}{2}



1/4 l Milch

70 g Butter

150 g Mehl

4 Eier, eventuell mehr je nach, Größe

1 Prise Salz

Muskat

weißer Pfeffer

120 g Allgäuer Emmentaler, gerieben

Öl oder Butterschmalz zum Ausbacken

\textit{Tomatenrahm}



1 EL Butter

1/2 Zwiebel

3 Fleischtomaten

100 ml Tomatensaft

1 Knoblauchzehe

4 EL Sahne

Jodsalz

weißer Pfeffer

Basilikum; frisch


\end{multicols}

Milch mit Butter, Salz, Muskat und Pfeffer zum Kochen bringen, Mehl mit dem Schneebesen einrühren. Mit dem Kochlöffel die Masse bei voller Hitze gut abrösten, bis sie sich gut vom Topfrand löst. In eine Schüssel umfüllen, abkühlen lassen, nach und nach die Eier unterrühren. Geriebenen Emmentaler zusetzen, verrühren. Aus dieser Masse mit dem Kaffeelöffel kleine Krusteln abstechen, im heißen Butterschmalz/Öl schwimmend goldbraun ausbacken.

Tomatenrahm: Butter erhitzen, fein gehackte Zwiebeln darin glasig
dünsten. Kleingeschnittene Fleischtomaten und gehackte Knoblauchzehe zugeben, mit Tomatensaft auffüllen, alles kurze Zeit durchkochen.

Sauce passieren, mit Sahne verfeinern, mit Jodsalz und Pfeffer abschmecken, mit frischem Basilikum verfeinern.

Auf Tellern anrichten, Käsekrapfen daraufsetzen.


{\bfseries Menge:} 2 Portionen

{\bfseries Inhalt pro Portion:} 5445~kJ, 41~g Eiweiß, 97~g Fett, 64~g Kohlenhydrate, 1300~kcal, 5.41~BE

{\bfseries Quelle:} www.kochfreunde.de (Sat.1 text 089.1994) 

} 

%----------------------
\nopagebreak{ 
\subsection{Birnen--Gorgonzola--Quiche}

\index{Birne@{Birne\/}!Birnen--Gorgonzola--Quiche@{Birnen--Gorgonzola--Quiche\/}}
\index{Käse@{Käse\/}!Birnen--Gorgonzola--Quiche@{Birnen--Gorgonzola--Quiche\/}}
\index{vegetarisch@{vegetarisch\/}!Birnen--Gorgonzola--Quiche@{Birnen--Gorgonzola--Quiche\/}}
\index{vorbereiten@{vorbereiten\/}!Birnen--Gorgonzola--Quiche@{Birnen--Gorgonzola--Quiche\/}}
}{
\begin{multicols}{2}



250 g Mehl

125 g Butter

1 Ei für den Teig

1 Zitrone oder Zitronensaft

1 Prise Salz, Pfeffer

1 Messerspitze Backpulver

2 große, reife Williams-Birnen

300 g Gorgonzola

3 Eier für die Füllung

125 g Sahne oder Schmand

Frischhaltefolie oder Backpapier


\end{multicols}

Den Backofen auf 20$^\circ$C vorheizen. Aus Mehl, zimmerwarmer Butter, Salz, einem Ei und Backpulver einen Teig kneten und in Folie gewickelt circa 1/2 Stunde in den Kühlschrank legen. Dann den Teig auf eine Folie legen und flach andrücken, mit einem zweiten Bogen Folie bedeckt ausrollen, so bleibt der Teig nicht am Nudelholz hängen. Nun den ausgerollten Teig in eine ausgebutterte Quiche-Form legen und vorsichtig bis zum Rand hoch ziehen. Die Birnen schälen, halbieren, das Kerngehäuse herausschneiden, in möglichst dünne Scheiben schneiden und dann mit etwas Zitronensaft beträufeln, damit sie nicht so schnell braun werden. Nun die Birnenscheiben auf dem Teig auslegen, den Gorgonzola in dünne Scheiben schneiden oder zerbröselt auf den Birnen verteilen. Die restlichen Eier mit der Sahne oder dem Schmand verkleppern, pfeffern und über die Quiche gießen. Im vorgeheizten Backofen etwa 40 Minuten backen, bis die Quiche goldbraun ist.


{\bfseries Menge:} 4 Portionen

{\bfseries Quelle:} alfredissimo! 

} 

%----------------------
\nopagebreak{ 
\subsection{Bärlauch--Risotto}

\index{Bärlauch@{Bärlauch\/}!Bärlauch--Risotto@{Bärlauch--Risotto\/}}
\index{Reis@{Reis\/}!Bärlauch--Risotto@{Bärlauch--Risotto\/}}
\index{vegetarisch@{vegetarisch\/}!Bärlauch--Risotto@{Bärlauch--Risotto\/}}
}{
\begin{multicols}{2}



20 mittelgroße Bärlauchblätter

2 Schalotten

2 El Olivenöl

400 g Risotto-Reis (Rundkorn)

1/8 l trockener Weißwein

1 l Gemüsebrühe (Instant)

3 EL Sahne

Salz

Pfeffer aus der Mühle

50 g gehobelter Pamesan

etwas Zitronensaft


\end{multicols}

Einige Bärlauchblätter für die Dekoration beiseite legen, die übrigen fein hacken.

Schalotten abziehen und fein hacken. Olivenöl erhitzen. Schalotten und Risotto-Reis darin glasig dünsten. Mit Weißwein ablöschen und einkochen lassen. Ein Drittel der Gemüsebrühe dazugießen, die Hitze reduzieren und das Ganze ab und zu umrühren. Die restliche Gemüsebrühe nach und nach dazugießen.

Wenn die Brühe aufgebraucht ist, Sahne und den gehackten Bärlauch unterheben. Risotto mit Salz und Pfeffer sowie etwas Zitronensaft abschmecken.

Risotto auf vier Teller verteilen und den gehobelten Parmesan darüber streuen. Mit den übrigen Bärlauchblättern garnieren.


{\bfseries Menge:} 4 Portionen

{\bfseries Quelle:} Bio Eck, Hannover, Alte Döhrener Straße 

} 

%----------------------
\nopagebreak{ 
\subsection{Camembert in Bierteig}

\index{Bierteig@{Bierteig\/}!Camembert in Bierteig@{Camembert in Bierteig\/}}
\index{Käse@{Käse\/}!Camembert in Bierteig@{Camembert in Bierteig\/}}
\index{vegetarisch@{vegetarisch\/}!Camembert in Bierteig@{Camembert in Bierteig\/}}
}{
\begin{multicols}{2}



1 Camembert

1 Ei

3 gehäufte EL Mehl

5 EL Bier

Öl zum Backen


\end{multicols}

Aus Ei, Mehl und Bier einen Teig anrühren. Camembert in Bierteig wenden und einige Minuten in Öl backen. Auf Toast servieren. Dazu passen heiße Schattenmorellen oder Preißelbeeren.


{\bfseries Menge:} 1 Portionen

{\bfseries Quelle:} Mutti 

} 

%----------------------
\nopagebreak{ 
\subsection{Emmentaler--Kräuter--Roulade  mit Pilzfüllung}

\index{Käseroulade@{Käseroulade\/}!Emmentaler--Kräuter--Roulade  mit Pilzfüllung@{Emmentaler--Kräuter--Roulade  mit Pilzfüllung\/}}
\index{vegetarisch@{vegetarisch\/}!Emmentaler--Kräuter--Roulade  mit Pilzfüllung@{Emmentaler--Kräuter--Roulade  mit Pilzfüllung\/}}
}{
\begin{multicols}{2}

\textit{Füllung}



30 g Butter

600 g Champignons, in dünne Scheiben geschnitten

2 Knoblauchzehen, feingehackt

Salz

frisch gemahlener schwarzer Pfeffer

Rouladengrundmasse mit Emmentaler und Kräutern


\end{multicols}

Die Butter auf kleiner Flamme erhitzen. Die Pilze und den Knoblauch einrühren. Etwa 12-15 Minuten dünsten, bis die Pilze weich sind und ihr Saft verkocht ist. Mit Salz und Pfeffer abschmecken.

Die gebackene Teigplatte mit den Pilzen belegen und aufrollen. Mit einer Rotweinsauce servieren.


{\bfseries Menge:} 4 Portionen

{\bfseries Quelle:} Rose Elliot: Klassische vegetarische Küche 

} 

%----------------------
\nopagebreak{ 
\subsection{Feta im Mangoldkleid}

\index{Feta@{Feta\/}!Feta im Mangoldkleid@{Feta im Mangoldkleid\/}}
\index{Mangold@{Mangold\/}!Feta im Mangoldkleid@{Feta im Mangoldkleid\/}}
\index{vegetarisch@{vegetarisch\/}!Feta im Mangoldkleid@{Feta im Mangoldkleid\/}}
}{
\begin{multicols}{2}

\textit{Feta}



750 g Mangold

400 g Fetakäse

1 Bund Dill

Paprikapaste

\textit{Sauce}



1 Schalotte

1 Glas Rote Bete, eingelegt

1 EL Butter

1/8 l Weißwein

2 EL Crème fraîche

Estragon

Salz

Pfeffer


\end{multicols}

Mangold putzen, waschen und die Stiele abschneiden. Blätter in reichlich Salzwasser 1 Minute blanchieren. Abtropfen lassen und auf Küchenkrepp ausbreiten, mit Pfeffer bestreuen. Fetascheiben in vier Stücke schneiden, mit Paprikapaste bestreichen, Dill darauflegen. Mit je 3 bis 4 Mangoldblättern umwickeln. Die Päckchen in einen Dampfeinsatz legen, über kochendem Wasser im geschlossenen Topf dämpfen. Für die Sauce Schalotte hacken, rote Bete in Stifte schneiden. Schalotte in Butter glasig dünsten. Rote Bete dazu geben. Wein angießen, würzen. Crème fraîche unterrühren, kurz aufkochen, gehackten Estragon in die Sauce rühren.


{\bfseries Menge:} 4 Portionen

{\bfseries Quelle:} Prisma (Zeitschrift) 

} 

%----------------------
\nopagebreak{ 
\subsection{Gebackene Zucchini mit Blauschimmelkäse}

\index{Zucchini@{Zucchini\/}!Gebackene Zucchini mit Blauschimmelkäse@{Gebackene Zucchini mit Blauschimmelkäse\/}}
\index{vegetarisch@{vegetarisch\/}!Gebackene Zucchini mit Blauschimmelkäse@{Gebackene Zucchini mit Blauschimmelkäse\/}}
\index{einfach@{einfach\/}!Gebackene Zucchini mit Blauschimmelkäse@{Gebackene Zucchini mit Blauschimmelkäse\/}}
}{
\begin{multicols}{2}



1 Zucchini a 200g

1 TL Öl

Salz

Pfeffer

50 g Roquefort


\end{multicols}

Ofen auf 200 Grad vorhaeizen, Zucchini gründlich waschen, trockenreiebn, längs halbieren und mit etwas Olivenöl, Salz und Pfeffer in eine ofenfeste Form geben.

Zunächst 20 min mit der Schnittseite nach unten garen, bis die Zucchini weich ist und die Haut Blasen wirft.
Dann umdrehen, mit Roquefort-Scheiben belegen und ca. 3 min überbacken, bis der Käse etwas verläuft.
Heiß servieren.


{\bfseries Menge:} 1 Portion

{\bfseries Zeit:} Gesamtzeit 30 min, Kochzeit 25 min, Vorbereitungszeit: 5 min

{\bfseries Quelle:} Frauke 

} 

%----------------------
\nopagebreak{ 
\subsection{Gefüllte Riesenchampignons}

\index{Käse@{Käse\/}!Gefüllte Riesenchampignons@{Gefüllte Riesenchampignons\/}}
\index{Pilze@{Pilze\/}!Gefüllte Riesenchampignons@{Gefüllte Riesenchampignons\/}}
\index{vegetarisch@{vegetarisch\/}!Gefüllte Riesenchampignons@{Gefüllte Riesenchampignons\/}}
}{
\begin{multicols}{2}



8 sehr große Champignons (250g)

1 Zwiebel

1 TL Majoran, gehackt

4 EL Olivenöl

Salz

schwarzer Pfeffer

200 g Sahneschmelzkäse oder Feta

1 Knoblauchzehe


\end{multicols}

Champignons säubern. Die Stiele aus den Hüten herausdrehen und fein hacken. Zwiebel und Knoblauch schälen und sehr fein hacken. 2 EL Öl in einer Pfanne erhitzen und darin die gehackten Stiele mit Zwiebel und Knoblauch anbraten. Mit Salz und Pfeffer würzen. Nach und nach den Käse zufügen und schmelzen lassen. Die Masse mit Majoran würzen und in die Champignonhüte füllen. Das restliche Öl in der Pfanne erhitzen, die gefüllten Chapignons hinein setzen und 10 Minuten braten. Dazu schmeckt am besten ofenwarmes Baguette.


{\bfseries Menge:} 4 Portionen

{\bfseries Quelle:} Gemüse: Jung, frisch, unwiderstehlich 

} 

%----------------------
\nopagebreak{ 
\subsection{Geröstetes Provenzalisches Gemüse}

\index{Auberginen@{Auberginen\/}!Geröstetes Provenzalisches Gemüse@{Geröstetes Provenzalisches Gemüse\/}}
\index{Fenchel@{Fenchel\/}!Geröstetes Provenzalisches Gemüse@{Geröstetes Provenzalisches Gemüse\/}}
\index{Knoblauch@{Knoblauch\/}!Geröstetes Provenzalisches Gemüse@{Geröstetes Provenzalisches Gemüse\/}}
\index{Paprika@{Paprika\/}!Geröstetes Provenzalisches Gemüse@{Geröstetes Provenzalisches Gemüse\/}}
\index{vegetarisch@{vegetarisch\/}!Geröstetes Provenzalisches Gemüse@{Geröstetes Provenzalisches Gemüse\/}}
}{
\begin{multicols}{2}



3 Fenchelknollen

3 rote Paprikaschoten

3 Auberginen oder Zucchini

3 rote Zwiebeln

Olivenöl

1 Knoblauchknolle

2 EL Balsamicoessig

frisches Basilikum

Salz

frisch gemahlener Pfeffer


\end{multicols}

Den Ofen auf 23$^\circ$C vorheizen.

Die Fenchelknollen putzen, dabei äußere zu harte Blätter entfernen. Die Knollen achteln, jedoch nicht zerpflücken. Jedes Achtel soll noch am Wurzelende zusammenhängen. den fenchek etwa 8 min vorkochen oder dämpfen. Er darf dabei nicht zu weich werden und zerfallen. Abtropfen lassen und mit Küchenpapier trockentupfen.

Die Paprika in breite Streifen schneiden, dabei Stielansatz und Samen entfernen. Nicht schälen; nach dem Rösten lässt sich das weiche Fruchtfleisch leicht von den Schalen schaben.

Den Stielansatz der Auberginen entfernen. Die Auberginen in große Würfel schneiden.

Die Zwiebeln schälen und wie den Fenchel in Achtel schneiden.

Die Gemüse mit Öl bestreichen, in einen Bräter setzen und 20 Minuten im vorgeheizten Ofen rösten.

Danach das Gemüse wenden, damit es gleichmäßig gart und die in Zehen zerpflückte Knoblauchknolle (ungeschält) dazwischenstreuen. Die Temperatur auf 180$^\circ$C herunterschalten und das Gemüse weitere 15-20 min garen, bis es weich und leicht gebräunt ist.

Das geröstete Gemüse in eine Serveirschüssel geben. Mit dem Baslsamessig beträufeln, mit Salz und Pfeffer würzen und mit fein geschnittenem Basilikum bestreuen. Heiß oder lauwarm servieren.

Das Gericht ist als Hauptgericht berechnet, als Beilage reichen 3/4 der Menge.


{\bfseries Menge:} 4 Portionen

} 

%----------------------
\nopagebreak{ 
\subsection{Gougère (Käse-Brandteig-Ring)}

\index{Käse@{Käse\/}!Gougère (Käse-Brandteig-Ring)@{Gougère (Käse-Brandteig-Ring)\/}}
\index{vegetarisch@{vegetarisch\/}!Gougère (Käse-Brandteig-Ring)@{Gougère (Käse-Brandteig-Ring)\/}}
}{
\begin{multicols}{2}



50 g Butter

150 ml Wasser

75 g ungebleichtes weißes Mehl

3 Eier

75 g würziger Hartkäse, in 5-mm-Würfel geschnitten

300 ml Tomatensauce


\end{multicols}

Ofen auf 15$^\circ$C, Gas Stufe 2, vorheizen. Butter und Wasser in mittelgroßem Topf bei mäßiger Hitze aufkochen. Sobald es kocht, Topf vom Herd nehmen und das ganze Mehl hineinschütten, dabei kräftig mit einem Holzlöffel abschlagen, bis das Mehl aufgesogen ist und sich der Teigkloß vom Boden löst. Leicht abkühlen lassen, dann nacheinander die Eier zugeben, dabei kräftig umrühren, bis der Teig weich und glänzend ist. Nur soviele Eier mehmen, dass der Teig noch fest genug ist und in Form bleibt. Zwei Drittel des Käses untermischen.

Teig mit einem Löffel oder einer Spritze auf ein gut gefettetes Backblech (oder eine Tarteform) geben; dabei einen Ring von 25 cm Durchmesser formen. Die restlichen Käsewürfel daraufsetzen, Antihaft-Folie darumgeben, damit der Ring in Form bleibt.

In die mittlere Schiene des vorgeheizten Ofens stellen, dabei Hitze langsam steigern; Brandteig geht am besten bei steigender Temperatur, 40 Minuten backen, Hitze abschalten, bei geöffneter Tür noch 5-10 Minuten abkühlen lassen; währenddessen die Sauce aufwärmen. Die Gougère auf eine heiße Servierplatte geben und mit Sauce anrichten.


{\bfseries Bemerkung:} Eiweiß: 13 g, kJ: 1480 

{\bfseries Menge:} 4 pro Portion

{\bfseries Quelle:} Paul Southey: Die Feinheiten der vegetarischen Küche 

} 

%----------------------
\nopagebreak{ 
\subsection{Greyerzerroulade mit rotem Paprika}

\index{Käseroulade@{Käseroulade\/}!Greyerzerroulade mit rotem Paprika@{Greyerzerroulade mit rotem Paprika\/}}
\index{vegetarisch@{vegetarisch\/}!Greyerzerroulade mit rotem Paprika@{Greyerzerroulade mit rotem Paprika\/}}
}{
\begin{multicols}{2}

\textit{Füllung}



4 große rote Paprikaschoten

125 g Frischkäse

2 EL Wasser

Rouladengrundmasse mit Greyerzer ohne Kräuter


\end{multicols}

Die Paprikaschoten vierteln. Mit der Hautseite nach oben unter den heißen Grill legen, bis die Haut Blasen wirft und schwarz wird. Abkühlen lassen, die Haut abziehen, Stielansatz und Samenstand entfernen.

Den Frischkäse mit etwa 2 EL Wasser glattrühren und auf die gebackene Teigplatte streichen. Mit Paprikastreifen belegen und aufrollen. Mit einer Kräutersauce servieren.


{\bfseries Menge:} 4 Portionen

{\bfseries Quelle:} Rose Elliot. Klassische vegetarische Küche 

} 

%----------------------
\nopagebreak{ 
\subsection{Grünkernlaib mit Blumenkohl}

\index{Blumenkohl@{Blumenkohl\/}!Grünkernlaib mit Blumenkohl@{Grünkernlaib mit Blumenkohl\/}}
\index{Grünkern@{Grünkern\/}!Grünkernlaib mit Blumenkohl@{Grünkernlaib mit Blumenkohl\/}}
\index{vegetarisch@{vegetarisch\/}!Grünkernlaib mit Blumenkohl@{Grünkernlaib mit Blumenkohl\/}}
}{
\begin{multicols}{2}



100 g Zwiebeln

100 g Möhren

300 g Lauch

250 g Grünkerngrütze (oder -schrot)

50 g Butter

3/4 l Gemüsebrühe

100 g Mandeln, gerieben

3 Eier

120 g Gouda, gerieben

100 g Vollkorntoastbrösel

Salz

Thymian

Pfeffer

Muskat

100 g Emmentaler, grob geraffelt

1 Blumenkohl


\end{multicols}

Zwiebeln, Möhren, Lauch fein würfeln. Grünkerngrütze in der Butter dünsten, Gemüse zugeben. Mit Gemüsebrühe auffüllen und zugedeckt 20 Minuten ausquellen lassen. Dann mit Mandeln, Eiern, Bröseln und Gouda mischen. Mit Salz, Pfeffer, Thymian und Muskat herzhaft würzen. 10 Minuten ruhen lassen. Den Teig zu einem Laib formen und bei 175 Grad ca. 35 Minuten backen. Mit Emmentaler bestreuen und weitere 5 Minuten backen. Dann 5 Minuten ruhen lassen und servieren. Dazu passt z.B. Blumenkohl mit Sauce.


{\bfseries Menge:} 4 Portionen

{\bfseries Quelle:} www.kochfreunde.de 

} 

%----------------------
\nopagebreak{ 
\subsection{Guntersblumer Spundekäs}

\index{Quark@{Quark\/}!Guntersblumer Spundekäs@{Guntersblumer Spundekäs\/}}
\index{vegetarisch@{vegetarisch\/}!Guntersblumer Spundekäs@{Guntersblumer Spundekäs\/}}
}{
\begin{multicols}{2}

\textit{ca. 3 Pfund für 10 Personen}



500 g Quark (40\% Fett i.T.)

250 g Kräuterquark

1 Becher Frischkäse

1 Becher Sahne

1 Becher Schmand

1 Becher Crème fraîche

100 g Butter

1 EL Senf

1 EL Ketchup

Knoblauch nach Belieben

Salz

Pfeffer

Paprika


\end{multicols}

Alle Quark- und Sahnesorten gut vermischen. Butter zerlaufen lassen und unterheben. Senf und Ketchup dazugeben. Mit Salz, Pfeffer, Paprika und evtl. Knoblauch abschmecken.

Die fetteren Zutaten können auch durch etwas weniger fette ersetzt werden.


{\bfseries Menge:} 10 Portionen

{\bfseries Quelle:} Renate Hedderich 

} 

%----------------------
\nopagebreak{ 
\subsection{Kartoffelkuchen}

\index{Kartoffeln@{Kartoffeln\/}!Kartoffelkuchen@{Kartoffelkuchen\/}}
\index{vegetarisch@{vegetarisch\/}!Kartoffelkuchen@{Kartoffelkuchen\/}}
}{
\begin{multicols}{2}



4 kleine Zwiebeln

1 Bund Schnittlauch

1 kg vorwiegend festkochende Kartoffeln

6 Eier

Salz

schwarzer Pfeffer

Muskat

Öl für die Form

2 Bund Dill

2 Becher saure Sahne (400 g)

4 Tomaten


\end{multicols}

2 Zwiebeln schälen und sehr fein würfeln. Den Schnittlauch abspülen, trockentupfen und in Röllchen schneiden. Die Kartoffeln schälen, waschen und trockentupfen. Eine Hälfte davon grob, die andere fein raspeln. Die Eier verquirlen, Kartoffeln mit Zwiebeln und Schnittlauch daruntermischen und kräftig mit Salz, Pfeffer und Muskat würzen. Die Masse in eine gefettete Tarteform (Durchmesser 28 cm) füllen und im 22$^\circ$C heißen Ofen 20--25 Minuten backen.

Dill abspülen, trockentupfen und fein zerschneiden. Saure Sahne (evtl. zur Hälfte durch Joghurt ersetzen) in einer Schüssel mit dem Dill verrühren und pikant mit Salz und Pfeffer abschmecken. Die Tomaten waschen, von den Stengelansätzen befreien und trockentupfen. dann in dünne Scheiben schneiden. Die übrigen Zwiebeln schälen und in hauchdünne Ringe schneiden.

Kartoffelkuchen heiß in Stücke teilen und mit saurer Sahne, Tomaten und Zwiebeln anrichten.


{\bfseries Menge:} 4 Portionen

{\bfseries Quelle:} Kartoffeln -- einfach köstlich 

} 

%----------------------
\nopagebreak{ 
\subsection{Käse-Mais-Muffins}

\index{Appetizer@{Appetizer\/}!Käse-Mais-Muffins@{Käse-Mais-Muffins\/}}
\index{Käse@{Käse\/}!Käse-Mais-Muffins@{Käse-Mais-Muffins\/}}
\index{vegetarisch@{vegetarisch\/}!Käse-Mais-Muffins@{Käse-Mais-Muffins\/}}
}{
\begin{multicols}{2}

\textit{Teig}



200 g Maismehl

70 g Mehl

1 Pk. Backpulver

1 Ei, verquirlt

200 g Joghurt

375 ml Milch

1 TL Salz

\textit{Käsemasse}



50 g geriebener Gouda, mittelalt

3 EL geriebener Parmesan

5 EL Mozzarella

3 Frühlingszwiebeln, fein gehackt

1/2 roter Paprika, fein gehackt

1/2 grüner Paprika, fein gehackt

2 Scheiben Schinkenspeck, klein geschnitten

Salz


\end{multicols}

Backofen auf 200 Grad vorheizen. Maismehl, Mehl, Salz und Backpulver in einer großen Rührschüssel mischen. Ei, Joghurt und Milch gut vermischen, in die die Maismehlmischung geben, schnell verrühren, bis alles gleichmäßig glatt ist. Käse, Frühlingszwiebeln, Paprika und Schinkenspeck gut vermischen, salzen. Teigmischung in eingefettete Muffinförmchen gießen (ergibt 36 kleine oder 24 größere Muffins). Da hinein einen EL der Käsemischung geben. 13-15 Minuten backen. Leicht abkühlen lassen. Muffins aus den Förmchen nehmen, solange sie noch warm sind, und sofort servieren.


{\bfseries Menge:} 24 Stück

{\bfseries Quelle:} Anne Wilson: Party-Köstlichkeiten, abgewandelt von Frauke 

} 

%----------------------
\nopagebreak{ 
\subsection{Käserouladen-Grundmasse}

\index{Käseroulade@{Käseroulade\/}!Käserouladen-Grundmasse@{Käserouladen-Grundmasse\/}}
\index{vegetarisch@{vegetarisch\/}!Käserouladen-Grundmasse@{Käserouladen-Grundmasse\/}}
}{
\begin{multicols}{2}



60 g körniger Frischkäse

150 ml Sahne

4 Eier

200 g geriebener Greyerzer oder Emmentaler

3 EL fein geschnittene Kräuter

(Thymian, Majoran, Petersilie) nach Belieben

Salz

frisch gemahlener schwarzer Pfeffer

frisch geriebener Parmesan zum Bestreuen

Butter und frisch geriebener Parmesan zum Bestreuen des Backblechs


\end{multicols}

Die vegetarischen Rouladen in diesem Abschnitt sind aus dieser Grundmasse zubereitet.  Sie sind für 6 Personen als Vorspeise und 4 Personen als Hauptgericht gedacht.

Eier trennen. Ofen auf 200 Grad vorheizen, ein 22 x 32 cm großes Blech mit Backpapier auslegen. Das Papier mit etwas weicher Butter bestreichen und mit tockenem Parmesan bestreuen.

Den Frischkäse in einer großen Schüssel mit der Sahne glattrühren. Die Eigelbe nacheinander einarbeiten. Danach den geriebenen Käse einrühren und nach Belieben die Kräuter. Mit Salz und Pfeffer abschmecken.

In einer sauberen Schüssel das Eiweiß zu steifem, aber nicht zu trockenem Schnee schlagen. Den Eischnee mit einem Metalllöffel unter die Käsemasse heben. Die Masse auf das vorbereitete Blech streichen. Im vorgeheizten Ofen 12-15 Minuten backen, bis die Masse leicht aufgegangen und gestockt ist.

Ein ausreichend großes Stück Pergamentpapier auf die Arbeitsfläche breiten und mit Parmesan bestreuen. Die gebackene Teigplatte auf das Pergamentpapier stürzen. Das Backpapier vorsichtig abziehen.

Den Teig handwarm abkühlen lassen. Danach mit der gewünschten Füllung bestreichen und von der Schmalseite her aufrollen. Dabei die Teigplatte mit Hilfe des Papiers anheben. Die Roulade mit der Naht nach unten legen, damit sie sich nicht löst. Die Enden glattschneiden. Die Roulade nach Belieben mit Parmesan betreuen.

Rouladen können im Voraus zubereitet und wieder aufgewärmt werden. Dazu wickelt man sie in Alufolie und gibt sie 15 Minuten in einen 160 Grad heißen Ofen.


{\bfseries Menge:} 4 Portionen

{\bfseries Quelle:} Rose Elliot: Klassische vegetarische Küche 

} 

%----------------------
\nopagebreak{ 
\subsection{Linsencurry, indisch}

\index{Linsen@{Linsen\/}!Linsencurry, indisch@{Linsencurry, indisch\/}}
\index{Paprika@{Paprika\/}!Linsencurry, indisch@{Linsencurry, indisch\/}}
\index{Vegetarisch@{Vegetarisch\/}!Linsencurry, indisch@{Linsencurry, indisch\/}}
}{
\begin{multicols}{2}



250 g Linsen, buntgemischt

500 ml Wasser

1 Bund Lauchzwiebel(n)

3 Zehe/n Knoblauch

3 Chilischote(n), rote

1 Paprikaschote(n), rote

2 EL Butterschmalz oder Ghee (geklärte Butter)

1 kleine Dose geschälte Tomaten

1 TL Gewürzmischung (Garam Masala)

1 TL Kurkuma

1 TL Kreuzkümmel, (gemahlen)

200 g Joghurt

1 Prise Salz

Pfeffer, schwarzer (aus der Mühle)

\bild{image/cookbook_3.jpg}


\end{multicols}

Die Linsen waschen und verlesen. Iml Wasser aufkochen, salzen und pfeffern. Zugedeckt nicht zu weich kochen (rote, gelbe ca. 10 Min, Beluga, braune ca. 20 Min.).

Lauchzwiebeln abbrausen, putzen und in feine, schräge Ringe schneiden. Knoblauch abziehen und klein würfeln. Die Chilischoten entkernen, abbrausen und in hauchdünne Streifen schneiden. Die Paprika putzen, abbrausen, halbieren und in feine Streifen schneiden.
Ghee (oder das Butterschmalz) in einem Wok oder einer hochwandigen Pfanne erhitzen.
Lauchzwiebeln, Paprika, Knoblauch und Chili unterrühren, 2 Min braten.

Dann Garam Masala, Kurkuma und Kreuzkümmel einstreuen, kurz anschwitzen und die Linsen samt der Flüssigkeit hinzugeben, nach Belieben Dosentomaten zugeben. Alles verrühren und kurz erhitzen. Mit Salz und Pfeffer abschmecken. Vom Herd nehmen, den Joghurt darüber geben und heiß servieren.

Zubereitungszeit: 20 Minuten


{\bfseries Menge:} 4 Portionen

{\bfseries Quelle:} www.chefkoch.de, mit Anmerkungen von Frauke 

} 

%----------------------
\nopagebreak{ 
\subsection{Pilzrisotto}

\index{Pilze@{Pilze\/}!Pilzrisotto@{Pilzrisotto\/}}
\index{Reis@{Reis\/}!Pilzrisotto@{Pilzrisotto\/}}
\index{vegetarisch@{vegetarisch\/}!Pilzrisotto@{Pilzrisotto\/}}
}{
\begin{multicols}{2}



1 Zwiebel

1 Petersilienwurzel

2 Möhren

250 g Pilze

60 g Butter oder Öl

2 Tassen Reis

4 Tassen Brühe

4 Tomaten oder 2 EL Tomatenmark

Petersilie

Schnittlauch

Reibekäse


\end{multicols}

Die Zwiebel, die Petersilienwurzel und die Möhren in kleine Würfel schneiden. Die Pilze putzen, waschen und kleinschneiden. Alles in zerlassener Butter oder Öl andünsten, den Reis dazugeben und ebenfalls anrösten. Die Brühe zugießen. Das Gericht etwa 10 Minuten auf kleiner Flamme kochen lassen. Die geschälten und kleingeschnittenen Tomaten oder das Tomatenmark zugeben und alles auf kleiner Flamme fertiggaren. Zum Schluss die gewiegten Kräuter untermischen. Den Risotto mit Reibekäse bestreut servieren.


{\bfseries Menge:} 4 Portionen

{\bfseries Quelle:} Kochen 

} 

%----------------------
\nopagebreak{ 
\subsection{Sherry-Nudeln}

\index{Alkohol@{Alkohol\/}!Sherry-Nudeln@{Sherry-Nudeln\/}}
\index{Gurken@{Gurken\/}!Sherry-Nudeln@{Sherry-Nudeln\/}}
\index{Pasta@{Pasta\/}!Sherry-Nudeln@{Sherry-Nudeln\/}}
\index{vegetarisch@{vegetarisch\/}!Sherry-Nudeln@{Sherry-Nudeln\/}}
\index{Zucchini@{Zucchini\/}!Sherry-Nudeln@{Sherry-Nudeln\/}}
}{
\begin{multicols}{2}



500 g Spaghetti

1 1/2 l Brühe

2 Tassen trockener Sherry (Manzanilla oder Fino)

2 Knoblauchzehen, mit Schale, zerdrückt

1/2 grüne Gurke

1 Bund Frühlingszwiebeln

1 große Zucchini

8 kleine Cherry-Tomaten

2 EL Butter

Parmesankäse, frisch gerieben

Salz

Pfeffer aus der Mühle

1 Bund Petersilie

Olivenöl


\end{multicols}

Zwei Tassen Sherry in einem großen Topf um die Hälfte reduzieren, darauf die Brühe gießen und in diesem Sud die Gemüse garen. Das Ganze sieht am Schluss aus wie eine Suppe mit Gemüsenudeln und echten Nudeln.

Dazu Gurke und Zucchini längs vierteln, Kerngehäuse entfernen und in Streifen schneiden. Von den Frühlingszwiebeln das dunkle Grün  entfernen, den Rest in Streifen schneiden, Knoblauch mit Schale andrücken. Im Sud sechs bis sieben Minuten ganz leise gar ziehen lassen. Nudeln kochen und leicht feucht in den Sherrygemüsesud geben. Alles kräftig salzen und mit gemörsten Pfeffer würzen.

Nun von den Tomaten den Strunk entfernen, dann klein hacken und mit der gehackten Petersilie zu den Nudeln geben. Vor dem Servieren die Butter dazu und alles gut verrühren -- auf tiefe Teller verteilen, mit Parmesan bestreuen und mit feinstem Olivenöl begießen.

Dazu schmeckt gekühlter trockener Sherry und Mineralwasser.


{\bfseries Menge:} 4 Portionen

{\bfseries Quelle:} NDR Rezeptdatenbank - Das! Rezept von Rainer Sass 

} 

%----------------------
\nopagebreak{ 
\subsection{Spargelquiche mit Kräuterfrischkäse}

\index{Quiche@{Quiche\/}!Spargelquiche mit Kräuterfrischkäse@{Spargelquiche mit Kräuterfrischkäse\/}}
\index{Spargel@{Spargel\/}!Spargelquiche mit Kräuterfrischkäse@{Spargelquiche mit Kräuterfrischkäse\/}}
\index{vegetarisch@{vegetarisch\/}!Spargelquiche mit Kräuterfrischkäse@{Spargelquiche mit Kräuterfrischkäse\/}}
}{
\begin{multicols}{2}

\textit{Teig}



300 g Mehl

300 g Butter

1 Ei

Salz

\textit{Sud}



1 EL Zitronensaft

1 EL Öl

1 TL Zucker

\textit{Belag}



750 g Spargel

1 1/2 Becher Kräuterfrischkäse (225 g)

125 l süße Sahne

1 Ei

1 Bund Kerbel

scharzer Pfeffer

1 Prise geriebene Muskatnuß


\end{multicols}

Das Mehl, die in Stücke geschnittene Butter, das Ei und das Salz in eine Rührschüssel geben und mit dem Knethaken des Handrührgerätes gut miteinander verkneten, bis alle Zutaten bröselig sind. Dann den Teig mit den Händen zusammenkneten, in eine Frischhaltefolie einwickeln und im Kühlschrank 30 Minuten ruhen lassen (Wer keinen Mürbeteig mag, kann es auch mit Hefeteig probieren). Inzwischen den Spargel putzen und schälen und in einem Sud aus Salzwasser, Zitronensaft und Öl ca. 15 Minuten kochen (Für dieses Gericht eignet sich Dosenspargel (2 Dosen) ebenfalls hervorragend). Den Spargel herausnehmen und abtropfen lassen. Den Frischkäse, die Sahne, das Ei sowie den gewaschenen und kleingehackten Kerbel (einige Kerbelblättchen zum Garnieren beiseite legen) miteinander verrühren und mit Pfeffer und Muskatnuss abschmecken. Den Backofen auf 225 Grad Celsius vorheizen. Den Teig zu einem Kreis von ca. 30 cm Durchmesser ausrollen und den Boden einer Quiche-, Tarte- oder Springform damit auslegen. Dabei den Rand an der S
eite festdrücken. Mit einer Gabel ein paar Mal einstechen und 10 Minuten auf der untersten Schiene des Backofens vorbacken. Dann die Frischkäsemasse auf dem Teigboden verteilen und die Spargelstangen radspeichenförmig darauf anrichten. Die Quiche etwa 35 Minuten auf der unteren Schiene backen. Die Quiche etwas abkühlen lassen und vor dem Servieren mit den Kerbelblättchen garnieren.


{\bfseries Menge:} 4 Portionen

{\bfseries Quelle:} Besser essen: Gemüse 

} 

%----------------------
\nopagebreak{ 
\subsection{Spinat mit Pilzen im Wok}

\index{Nüsse@{Nüsse\/}!Spinat mit Pilzen im Wok@{Spinat mit Pilzen im Wok\/}}
\index{Pilze@{Pilze\/}!Spinat mit Pilzen im Wok@{Spinat mit Pilzen im Wok\/}}
\index{Spinat@{Spinat\/}!Spinat mit Pilzen im Wok@{Spinat mit Pilzen im Wok\/}}
\index{Vegetarisch@{Vegetarisch\/}!Spinat mit Pilzen im Wok@{Spinat mit Pilzen im Wok\/}}
}{
\begin{multicols}{2}



25 g Pinienkerne

500 g frische Spinatblätter

2 Knoblauchzehen

3 EL Pflanzenöl

425 g Strohpilze aus der Dose abgetropft (eratzweise Champignons)

25 g Rosinen

2 EL Sojasauce

Salz


\end{multicols}

Einen Wok oder eine große, schwere Pfanne erhitzen.

Die Pinienkerne darin (ohne Öl) leicht braun rösten. Mit einem Pfannenschaufler herausheben und für später beiseite stellen.

Den Spinat gründlich waschen, verlesen und lange Stiele entfernen. Gründlich abtropfen lassen und mit Küchenpapier trocken tupfen.

Die rote Zwiebel und den Knoblauch mit einem scharfen Küchenmesser in Scheiben schneiden. Das Pflanzenöl in dem Wok oder der Bratpfanne erhitzen. Die Zwiebel- und Knoblauchscheiben zugeben und 1 Minute anbraten, bis sie weich werden.

Den Spinat und die Pilze hinzufügen und weiter anbraten, bis die Blätter zusammenfallen. Überschüssige Flüssigkeit abgießen.

Die Rosinen, die gerösteten Pinien- kerne und die Sojasauce einrühren. Pfannenbraten, bis alles erhitzt ist und alle Zutaten gut vermischt sind.

Mit Salz abschmecken, auf einen vorgewärmten Servierteller füllen und servieren.

TIP: Weichen Sie die Rosinen vor Verwendung in 2 EL trockenem Sherry ein. Dabei quellen sie etwas
auf und geben ihnen beim Anbraten eine zusätzliche Geschmacksnote.


{\bfseries Bemerkung:} Eiweiß: 7 g, Fett: 15 g, Kohlenhydrate: 10 g, kcal: 201, BE: 1 

{\bfseries Menge:} 4 Portionen

{\bfseries Quelle:} Asiatisch kochen - die besten Wok- und Pfannengerichte 

} 

%----------------------
\nopagebreak{ 
\subsection{Spinatroulade mit Frischkäse und Paprikastreifen}

\index{Käseroulade@{Käseroulade\/}!Spinatroulade mit Frischkäse und Paprikastreifen@{Spinatroulade mit Frischkäse und Paprikastreifen\/}}
\index{Paprika@{Paprika\/}!Spinatroulade mit Frischkäse und Paprikastreifen@{Spinatroulade mit Frischkäse und Paprikastreifen\/}}
\index{vegetarisch@{vegetarisch\/}!Spinatroulade mit Frischkäse und Paprikastreifen@{Spinatroulade mit Frischkäse und Paprikastreifen\/}}
}{
\begin{multicols}{2}

\textit{Roulade}



500 g zarter entstielter Spinat

15 g Butter

4 Eier

frisch geriebene Muskatnuss

Salz

frisch geriebener Parmesan zum Bestreuen

\textit{Füllung}



1 große rote Parikaschote

250 g Frischkäse

etwas Milch


\end{multicols}

Diese Roulade ist für 6 Personen als Vorspeise und 4 Personen als Hauptgericht gedacht.

Die Paprikaschoten vierteln. Mit der Hautseite nach oben unter den heißen Grill legen, bis die Haut Blasen wirft und schwarz wird. Abkühlen lassen, die Haut abziehen, Stielansatz und Samenstand entfernen. In Streifen schneiden.

Den Spinat nur mit dem Wasser, das nach dem Waschen noch an ihm haftet, in einen Topf geben und über milder Hitze 7-10 Minuten zusammenfallen lassen. Danach kräftig ausdrücken.

Eier trennen. Ofen auf 200 Grad vorheizen, ein 22 x 32 cm großes Blech mit Backpapier auslegen. Das Papier mit etwas weicher Butter bestreichen und mit tockenem Parmesan bestreuen.

Den Spinat, die Butter und das Eigelb in einer großen Schüssel mit dem Mixer vermischen. Mit Muskatnuss, Salz und Pfeffer abschmecken.

In einer sauberen zweiten Schüssel das Eiweiß zu steifem, aber nicht zu trockenem Schnee schlagen. Den Eischnee mit einem Metalllöffel unter die Spinatmasse heben. Die Masse auf das vorbereitete Blech geben und glattstreichen. Die Teigplatte mit 2 EL Parmesan betreuen. Im vorgeheizten Ofen 12-15 Minuten backen, bis sie fest ist.

Ein ausreichend großes Stück Pergamentpapier auf die Arbeitsfläche breiten und mit Parmesan bestreuen. Die gebackene Teigplatte auf das Pergamentpapier stürzen. Das Backpapier vorsichtig abziehen.

Den Teig handwarm abkühlen lassen. Den Frischkäse mit etwas Milch glattrühren und auf die gebackene Teigplatte streichen. Die Paprikastreifen in weiten Zwischenräumen darauf legen und die Roulade von der Schmalseite her aufrollen. Dabei die Teigplatte mit Hilfe des Papiers anheben. Die Roulade mit der Naht nach unten legen, damit sie sich nicht löst. Die Enden glattschneiden. Die Roulade nach Belieben mit Parmesan betreuen.

Mit einer gelben Paprikasauce servieren.


{\bfseries Menge:} 4 Portionen

{\bfseries Quelle:} Rose Elliot: Klassische vegetarische Küche 

} 

%----------------------
\nopagebreak{ 
\subsection{Wirsingrouladen mit Graupenfüllung}

\index{Graupen@{Graupen\/}!Wirsingrouladen mit Graupenfüllung@{Wirsingrouladen mit Graupenfüllung\/}}
\index{vegetarisch@{vegetarisch\/}!Wirsingrouladen mit Graupenfüllung@{Wirsingrouladen mit Graupenfüllung\/}}
\index{Wirsing@{Wirsing\/}!Wirsingrouladen mit Graupenfüllung@{Wirsingrouladen mit Graupenfüllung\/}}
}{
\begin{multicols}{2}



100 g große Gerstengraupen

(über Nacht eingeweicht)

1 kleiner Wirsingkohl

1 mittelgroße Zwiebel

Salz, Muskat

3 Pfefferkörner

Hefewürze

1 Lorbeerblatt

1 Ei

1 mittelgroße Möhre

1 Stange Lauch

2 1/2 EL Semmelbrösel

1 Bund Petersilie

2 EL Crème fraîche

300 ml Gemüsebrühe

100 ml Sahne


\end{multicols}

Den Wirsingkohl im Salzwasser ca. 15 Minuten kochen, bis sich die Blätter lösen, abgießen und Blätter für 4 oder mehr Wirsingrouladen ausbreiten. Zwischenzeitlich die Graupen in der Gemüsebrühe (etwas Brühe für die Rouladen lassen) mit der geviertelten Zwiebel, dem Salz, der Hefewürze, dem Lorbeerblatt und den Pfefferkörnern garen. Die Masse dann abkühlen lassen. In die abgekühlte Masse Ei, grobgeraspelte Möhre, feingeschnittene Lauchstücke und Semmelbrösel einarbeiten. Mit Muskat, Salz, gehackter Petersilie und Crème fraîche abschmecken. Die fächerartig ausgebreiteten Wirsingblätter leicht mit Salz bestreuen und jeweils mit der Graupenmasse füllen. Die Seiten einschlagen und zusammenrollen, mit Bindfäden oder Holzstäbchen zusammenhalten. In heißem Pflanzenfett rundum anbraten und unter Zugabe von etwas Gemüsebrühe und Sahne ca. 30 Minuten schmoren. Zum Gericht Tomaten- oder Kräutersauce servieren.


{\bfseries Menge:} 4 Portionen

{\bfseries Quelle:} Packung Brüggen Gerstengraupen 

} 

%----------------------
\nopagebreak{ 
\subsection{Zucchiniküchlein (Frittelle di zucchini)}

\index{italienisch@{italienisch\/}!Zucchiniküchlein (Frittelle di zucchini)@{Zucchiniküchlein (Frittelle di zucchini)\/}}
\index{vegetarisch@{vegetarisch\/}!Zucchiniküchlein (Frittelle di zucchini)@{Zucchiniküchlein (Frittelle di zucchini)\/}}
\index{Zucchini@{Zucchini\/}!Zucchiniküchlein (Frittelle di zucchini)@{Zucchiniküchlein (Frittelle di zucchini)\/}}
}{
\begin{multicols}{2}



500 g frische, junge Zucchini (jede höchstens 175-250 g)

60 g Mehl, plus 2 EL bei Bedarf

60 g frisch geriebener Parmesan

1 extra großes Ei

1 Prise frisch geriebene Muskatnuss

Salz

frisch gemahlener Pfeffer

Olivenöl zum Frittieren


\end{multicols}

Ein Rezept aus Udine im nördlichen Friaul. Diese Küchlein eignen sich als Antipasto oder als Beilage zu Fisch oder Fleisch.

Die Zucchini waschen und die Enden abschneiden. Der Länge nach halbieren und das weiche Innere und die Kerne entfernen, falls sie sehr groß sind. Die Zucchini grob raspeln und in eine Schüssel geben. Mit dem Mehl, Parmesan, Ei, Muskat, einer kleinen Prise Salz und Pfeffer gründlich zu einem recht dicken Teig verrühren. Falls er zu dünn ist, noch etwas Mehl einrühren (bis zu 2 EL).

In eine Sauteuse 1 cm hoch Olivenöl einfüllen und erhitzen, bis etwas hinein getropfter Teig zischt. Den Teig esslöffelweise ins Öl geben und mit dem Löffel etwas flachdrücken. Auf jeder Seite etwa  2 Minuten ausbacken, bis die Küchlein goldbraun sind. Sie sollten innen gar, doch außen nicht verbrannt sein.

Die Küchlein auf Küchenpapier abtropfen lassen, auf eine vorgewärmte Platte geben, nach Bedarf salzen und servieren.

Abwandlung: Ein Teil der Zucchini kann auch durch Möhren ersetzt werden.


{\bfseries Menge:} 4 Portionen

{\bfseries Quelle:} Julia della Croce: Klassische italienische Küche 

} 

%----------------------
\nopagebreak{ 
\subsection{Zwiebel-Hokkaido-Kuchen - klein und vegetarisch}

\index{Kürbis@{Kürbis\/}!Zwiebel-Hokkaido-Kuchen - klein und vegetarisch@{Zwiebel-Hokkaido-Kuchen - klein und vegetarisch\/}}
\index{Vegetarisch@{Vegetarisch\/}!Zwiebel-Hokkaido-Kuchen - klein und vegetarisch@{Zwiebel-Hokkaido-Kuchen - klein und vegetarisch\/}}
\index{Zwiebeln@{Zwiebeln\/}!Zwiebel-Hokkaido-Kuchen - klein und vegetarisch@{Zwiebel-Hokkaido-Kuchen - klein und vegetarisch\/}}
}{
\begin{multicols}{2}

\textit{Teig}



250 g Mehl

125 g Margarine

1 Ei

1 Prise Backpulver

\textit{Belag}



Öl

500 g Zwiebeln

500 g Kürbis

1/4 l saure Sahne

2 Eier

Salz


\end{multicols}

Den Teig zubreiten und eine Springform damit auskleiden. Die Zwiebeln schälen und kleinschneiden; den Kürbis kleinschneiden und evtl. raspeln.

Öl in einer Pfanne erhitzen und die Zwiebeln ca. 20 Minuten andünsten, dann den Kürbis hinzufügen und weitere 15 Minuten dünsten.

Die saure Sahne mit den Eiern und Gewürzen verrühren und über das Gemüsegemisch gießen, aufstocken lassen. Belag auf dem Teig verteilen. Backzeit: Ca. 60 Minuten bei 175 Grad


{\bfseries Menge:} 4 Portionen

{\bfseries Quelle:} www.kuerbis-company.de 

} 

\pagebreak 
 
%----------------------------------------------------
\nopagebreak{ 
\section{SUPPE} 

\subsection{Apfel-Aprikosen-Suppe}

\index{Apfel@{Apfel\/}!Apfel-Aprikosen-Suppe@{Apfel-Aprikosen-Suppe\/}}
\index{Aprikosen@{Aprikosen\/}!Apfel-Aprikosen-Suppe@{Apfel-Aprikosen-Suppe\/}}
\index{Ingwer@{Ingwer\/}!Apfel-Aprikosen-Suppe@{Apfel-Aprikosen-Suppe\/}}
}{
\begin{multicols}{2}



125 g getrocknete Aprikosen, über Nacht eingeweicht

500 g Äpfel, geschält und in Würfel, geschnitten

1 klein. Zwiebel, gewürfelt

1 EL Zitronen- oder Limettensaft

700 ml frische Hühnerbrühe

1/4 TL gemahlener Ingwer

1 kräftige Prise gemahlener Piment

150 ml trockener Weißwein

Salz und Pfeffer

4 EL Sauerrahm oder Frischkäse

etwas gemahlener Ingwer oder Piment zum Garnieren


\end{multicols}

Die Aprikosen abtropfen lassen und klein hacken.

In einen Topf geben und Äpfel, Zwiebeln, Zitronen- oder Limettensaft und Brühe zugeben. Zum Kochen bringen, abdecken und ca. 20 Minuten köcheln lassen.

Die Suppe etwas abkühlen lassen, dann durch ein Sieb streichen oder pürieren.

In einen sauberen Topf gießen. Wein und Gewürze zugeben und mit Salz und Pfeffer abschmecken. Einmal aufkochen, dann abkühlen lassen. Ist die Suppe zu dick, etwas Brühe oder Wasser zugießen. Anschließend kalt stellen.

Jede Portion mit einem Löffel Sauerrahm oder Frischkäse garnieren und mit Piment oder Ingwer bestreuen.


{\bfseries Bemerkung:} Eiweiß: 3 g, Fett: 1 g, Kohlenhydrate: 29 g, kcal: 147 

{\bfseries Menge:} 6 Portionen

{\bfseries Inhalt pro Portion:} 842~kJ, 9~g Eiweiß, 6~g Fett, 21~g Kohlenhydrate, 201~kcal, 1.77~BE

{\bfseries Quelle:} Low-Fat -- Die besten Rezepte aus aller Welt 

} 

%----------------------
\nopagebreak{ 
\subsection{Apfel-Curry-Suppe mit Hähnchen}

\index{Apfel@{Apfel\/}!Apfel-Curry-Suppe mit Hähnchen@{Apfel-Curry-Suppe mit Hähnchen\/}}
\index{Huhn@{Huhn\/}!Apfel-Curry-Suppe mit Hähnchen@{Apfel-Curry-Suppe mit Hähnchen\/}}
\index{Vegetarisch@{Vegetarisch\/}!Apfel-Curry-Suppe mit Hähnchen@{Apfel-Curry-Suppe mit Hähnchen\/}}
}{
\begin{multicols}{2}



4 Äpfel, säuerlich, z.B. Boskop

1 Zwiebel

1 Hähnchenbrust

1/2 l Gemüsebrühe

1 EL Currypulver (Madrascurry)

Schlagsahne oder Crème fraiche

Salz und Pfeffer

Butter

evtl. Zucker


\end{multicols}

 Zwiebel und Hähnchenbrust (für Vegatarier: einfach weglassen) würfeln und in etwas Butter dünsten,
Äpfel schälen und in dünne Scheiben schneiden / hobeln. Zu den Zwiebeln geben, weiter dünsten.
Mit dem Currypulver bestäuben und anschließend mit der Gemüsebrühe aufgießen.
Das Ganze mit dem Stabmixer mixen. Mit Salz und Pfeffer, evtl. etwas Zucker abschmecken. Mit Crème fraiche oder geschlagener Sahne servieren.

Dazu schmecken sehr gut frische Croutons.


{\bfseries Menge:} 4 Portionen

{\bfseries Quelle:} www.chefkoch.de, abgewandelt von Frauke 

} 

%----------------------
\nopagebreak{ 
\subsection{Chili--Bohnensuppe}

\index{Bohnen@{Bohnen\/}!Chili--Bohnensuppe@{Chili--Bohnensuppe\/}}
\index{Hauptgericht@{Hauptgericht\/}!Chili--Bohnensuppe@{Chili--Bohnensuppe\/}}
\index{P1@{P1\/}!Chili--Bohnensuppe@{Chili--Bohnensuppe\/}}
\index{Vegetarisch@{Vegetarisch\/}!Chili--Bohnensuppe@{Chili--Bohnensuppe\/}}
}{
\begin{multicols}{2}



230 g weiße Bohnen aus der Dose (Abtropfgewicht, 1 kl. Dose)

1 Knoblauchzehe

90 g Dosentomaten mit Saft

1 Päckchen TK-Suppengrün

Salz

Chili-Gewürzmischung

Thymian oder TK-Kräuter der Provence

2 TL Olivenöl

1 Roggenbrötchen


\end{multicols}

Die Bohnen etwas abtropfen lassen. Den Knoblauch abziehen und hacken. Beides mit den Tomaten und dem Suppengrün in einen Topf füllen. Mit Salz und Chiligewürz würzen und 8 Minuten offen kochen. Kräuter unterrühren und weitere 2 Minuten kochen, dann das Öl zufügen und noch einmal abschmecken.
Das Brötchen toasten und dazu essen.

Tipp: Kochen Sie die doppelte Menge Suppe und frieren Sie sie
für später ein.


{\bfseries Bemerkung:} Fett: 12 g 

{\bfseries Menge:} 1 pro Portion

{\bfseries Quelle:} Brigitte: Jobdiät 2001 

} 

%----------------------
\nopagebreak{ 
\subsection{Erbseneintopf}

\index{Erbsen@{Erbsen\/}!Erbseneintopf@{Erbseneintopf\/}}
\index{vorbereiten@{vorbereiten\/}!Erbseneintopf@{Erbseneintopf\/}}
}{
\begin{multicols}{2}



250 g Erbsen, getrocknet

60 g Speck, durchwachsen

1 Zwiebel(n), mittelgroße, fein gehackt

100 g Karotten, gewürfelt

100 g Sellerie, gewürfelt

300 g Kartoffeln, gewürfelt

1 TL Majoran, gerebelt

Salz

Pfeffer, weiß aus der Mühle

4 Wiener Würstchen

1 EL Petersilie, gehackt


\end{multicols}

Die Erbsen waschen und über Nacht in 1,5 L Wasser einweichen. Den Speck würfeln und in einem Topf auslassen, Karotten und Sellerie dazu geben und im Speck anbraten. Erbsen mit dem Einweichwasser hineingeben und ca. 1 Stunde köcheln lassen. Dann die Kartoffeln und den Majoran zugeben, weiterkochen. Sind die Kartoffeln gar, den Eintopf würzen und die Würstchen darin heiß werden lassen. Zum Schluss die Petersilie darüber streuen.


{\bfseries Menge:} 4 Personen

{\bfseries Inhalt pro Portion:} 1997~kJ, 28~g Eiweiß, 21~g Fett, 40~g Kohlenhydrate, 477~kcal, 3.32~BE

{\bfseries Zeit:} Gesamtzeit 90 min

{\bfseries Quelle:} chefkoch.de 

} 

%----------------------
\nopagebreak{ 
\subsection{Kalte Kartoffel--Kohlrabi--Creme}

\index{Kartoffeln@{Kartoffeln\/}!Kalte Kartoffel--Kohlrabi--Creme@{Kalte Kartoffel--Kohlrabi--Creme\/}}
\index{Kohlrabi@{Kohlrabi\/}!Kalte Kartoffel--Kohlrabi--Creme@{Kalte Kartoffel--Kohlrabi--Creme\/}}
\index{Lachs@{Lachs\/}!Kalte Kartoffel--Kohlrabi--Creme@{Kalte Kartoffel--Kohlrabi--Creme\/}}
}{
\begin{multicols}{2}



500 g mehligkochende Kartoffeln

2 Kohlrabi (etwa 400 g)

1 Zwiebel

1 l Fleischbrühe

1 Zweig Thymian

Salz

1 TL gemahlener Kümmel

150 g saure Sahne

weißer Pfeffer

1 EL süßer Senf

1/2 Becher Schlagsahne (100 g)

200 g Räucherlachs

(zur Hälfte durch Krabben ersetzbar)


\end{multicols}

Kartoffeln, Kohlrabi und Zwiebel schälen und in Würfel schneiden. Alles zusammen in die Brühe geben, mit dem Thymianzweig, Salz und Kümmel würzen und zugedeckt etwa 30 Minuten kochen. Etwas abkühlen lassen und im Mixer pürieren. Saure Sahne darunterrühren und alles sehr gut durchkühlen lassen. Danach mit Salz, Pfeffer und Senf abschmecken. Den Dill abspülen, gut trockentupfen und die Blättchen von den Stielen zupfen. Zwei Drittel davon mit der Sahne im Mixer aufschäumen lassen. Die Suppe auf Tellern anrichten und die Dillsahne darauf verteilen. Den Lachs in feine Streifen schneiden und mit dem übrigen Dill auf die Suppe streuen.


{\bfseries Menge:} 4 Portionen

{\bfseries Quelle:} Kartoffeln -- einfach köstlich 

} 

%----------------------
\nopagebreak{ 
\subsection{Kartoffel-Rauke-Suppe}

\index{einfach@{einfach\/}!Kartoffel-Rauke-Suppe@{Kartoffel-Rauke-Suppe\/}}
\index{Kartoffeln@{Kartoffeln\/}!Kartoffel-Rauke-Suppe@{Kartoffel-Rauke-Suppe\/}}
\index{Rucola@{Rucola\/}!Kartoffel-Rauke-Suppe@{Kartoffel-Rauke-Suppe\/}}
\index{vegetarisch@{vegetarisch\/}!Kartoffel-Rauke-Suppe@{Kartoffel-Rauke-Suppe\/}}
}{
\begin{multicols}{2}



1 1/2 l Hühnerbrühe

1 1/4 kg mehlig kochende Kartoffeln, in kleine Stücke geschnitten

2 große Knoblauchzehen

250 g Rauke

1 EL Olivenöl

Rauke zum Garnieren

50 g gehobelter Parmesan

Salz

schwarzer Pfeffer


\end{multicols}

Die Brühe in einem großen Topf zum Kochen bringen. Kartoffeln und Knoblauch zufügen. Bei mittlerer Hitze etwa 15 min köcheln lassen, bis die Kartoffeln weich sind. Die Rauke zugeben
und die Mischung etwa 2 min weiterköcheln lassen. Das Olivenöl unterrühren.

Die Mischung pürieren, wieder in den Topf geben und bei mittlerer Temperatur unter Rühren erhitzen.
Mit Salz und zerstoßenem schwarzem Pfeffer nach Geschmack würzen und in vorgewärmten Schalen mit Rauke und Parmesan bestreut servieren.

Tipp: Zusätzlichen Geschmack erhält die Suppe durch Zufügen von kross gebratenem Speck.

Variation: Ala Garnitur für besondere Gelegenheiten kann man in Butter kurz gebratene Jakobsmuscheln und frittierte Raukeblätter verwenden.


{\bfseries Bemerkung:} Eiweiß: 8 g, Fett: 6 g, Kohlenhydrate: 30 g, kJ: 850, kcal: 205 

{\bfseries Menge:} 6 Portionen

{\bfseries Quelle:} Anne Wilson: Schnelle Suppen und Eintöpfe 

} 

%----------------------
\nopagebreak{ 
\subsection{Kichererbsensuppe}

\index{Kichererbsen@{Kichererbsen\/}!Kichererbsensuppe@{Kichererbsensuppe\/}}
\index{Klöße@{Klöße\/}!Kichererbsensuppe@{Kichererbsensuppe\/}}
\index{vorbereiten@{vorbereiten\/}!Kichererbsensuppe@{Kichererbsensuppe\/}}
}{
\begin{multicols}{2}



250 g Kichererbsen (getrocknet)

4 Schalotten

1 Bund Suppengemüse

40 g Butter

1/4 l trockener Weißwein

1 l Instant-Gemüsebrühe

1 Lorbeerblatt

500 g Tomaten in Stücken (Dose)

250 g Kalbsbratwurst

Salz

1/2 Bund glatte Petersilie

100 g Schlagsahne

weißer Pfeffer aus der Mühle


\end{multicols}

Kichererbsen über Nacht in 2 l Wasser einweichen.

Dann am nächsten Tag geschälte Schalotten und geputztes Suppengemüse würfeln, in Butter glasig braten. Mit Wein und Brühe ablöschen. Lorbeerblatt, Tomaten und die in einem Sieb abgebrausten Kichererbsen zufügen und in 50-60 Minuten weich kochen.

Wurstmasse in kleinen Klößchen aus der Pelle drücken und im nur simmernden Salzwasser garen.

Petersilie hacken, Sahne steif schlagen.

Suppe pürieren, durch ein Sieb streichen (wichtig!) und mit Salz und Pfeffer würzen.

Sahne unterziehen, die abgetropften Klößchen in die Suppe geben und mit Petersilie bestreut servieren.


{\bfseries Menge:} 4 Portionen

{\bfseries Quelle:} Winterliches Gemüse: Frisch und pfiffig 

} 

%----------------------
\nopagebreak{ 
\subsection{Kichererbsensuppe mit Spinat}

\index{Kichererbsen@{Kichererbsen\/}!Kichererbsensuppe mit Spinat@{Kichererbsensuppe mit Spinat\/}}
\index{Spinat@{Spinat\/}!Kichererbsensuppe mit Spinat@{Kichererbsensuppe mit Spinat\/}}
\index{vegetarisch@{vegetarisch\/}!Kichererbsensuppe mit Spinat@{Kichererbsensuppe mit Spinat\/}}
}{
\begin{multicols}{2}



4 EL Olivenöl

300 g Blattspinat, küchenfertig

1 Schalotte, fein gehackt

1 Möhre, kleingeschnitten

2 Knoblauchzehen, kleingeschnitten

30 g frische Petersilie, grob gehackt

1 TL Paprikapulver

800 ml Brühe

1 Dose Kichererbsen (a 400g Abtropfgewicht)

2 Gewürznelken

1 Lorbeerblatt

125 ml saure Sahne

Salz

Pfeffer


\end{multicols}

In einem Topf 1 EL Öl erhitzen, Spinat darin in 3-4 Min. zusammenfallen lassen, in einem Sieb auslecken lassen, mit einem Löffel ausdrücken und grob schneiden.

Topf mit Küchentuch auswischen und den Rest des Öls erhitzen. Schalotte, Möhre, Knoblauch, Petersilie und Paprikapulver 3 Min. anrösten. Brühe dazufügen und die Hälfte der Kichererbsen einrühren. Gewürznelken in das Lorbeerblatt stecken und in die Suppe geben. Suppe 5 Min. sachte kochen lassen.

Lorbeerblatt und Nelken herausnehmen und Suppe mit dem Stabmixer pürieren. Mit Salz und Pfeffer abschmecken. Spinat und den Rest der Kichererbsen in die Suppe rühren und kurz mit erwärmen. Suppe in vier tiefe Teller füllen. In die Mitte jeden Tellers einen Esslöffel saure Sahne setzen.

Dazu passt Vollkornbrot.


{\bfseries Bemerkung:} Eiweiß: 8 g, Fett: 18 g, Kohlenhydrate: 19 g, kcal: 235 

{\bfseries Menge:} 4 Portionen

{\bfseries Quelle:} AllerHande 01/2004 

} 

%----------------------
\nopagebreak{ 
\subsection{Klare Erbsensuppe mit Grießklößchen}

\index{Erbsen@{Erbsen\/}!Klare Erbsensuppe mit Grießklößchen@{Klare Erbsensuppe mit Grießklößchen\/}}
\index{Geflügel@{Geflügel\/}!Klare Erbsensuppe mit Grießklößchen@{Klare Erbsensuppe mit Grießklößchen\/}}
\index{Klöße@{Klöße\/}!Klare Erbsensuppe mit Grießklößchen@{Klare Erbsensuppe mit Grießklößchen\/}}
\index{vorbereiten@{vorbereiten\/}!Klare Erbsensuppe mit Grießklößchen@{Klare Erbsensuppe mit Grießklößchen\/}}
}{
\begin{multicols}{2}

\textit{Suppe}



500 g Hühnerklein

1 Zwiebel

150 g Knollensellerie

2 Lorbeerblätter

Salz

weißer Pfeffer aus der Mühle

300 g Erbsen (TK)

2 l Wasser

\textit{Klößchen}



4 Zweige Petersilie

65 g Weizengrieß

1 Ei

60 g Margarine

Salz

Pfeffer


\end{multicols}

Hühnerklein kalt waschen und gut abtrocknen. Zwiebel und Sellerie schälen und samt Hühnerklein, Lorbeer, Salz und Pfeffer in kaltem Wasser aufsetzen und aufkochen. Bei sanfter Hitze 60 Minuten köcheln lassen und zwischendurch mehrmals abschäumen. Petersilie abbrausen, trockentupfen und hacken. Grieß mit Ei, Margarine, Salz und Pfeffer verrühren, Petersilie zufügen und die Masse 1 Stunde quellen lassen. Dann Salzwasser erhitzen, bis es leise siedet (nicht sprudelnd kocht). Mit Hilfe von 2 Teelöffeln nacheinander 12 Klößchen aus der Grießmasse formen. Sie lösen sich beim Eintauchen ins Salzwasser vom Löffel. Die Klößchen in 15 Minuten gar ziehen lassen. Herausnehmen, abtropfen lassen und beiseite stellen. Die Hühnerbrühe durchsieben und erneut aufkochen. Die Erbsen darin in 8-10 Minuten garen. Die Suppe mit Salz und Pfeffer abschmecken und die Klößchen kurz darin erwärmen.


{\bfseries Menge:} 4 Portionen

{\bfseries Quelle:} Gemüse. Jung, frisch, unwiderstehlich 

} 

%----------------------
\nopagebreak{ 
\subsection{Käsesuppe mit Lauch}

\index{einfach@{einfach\/}!Käsesuppe mit Lauch@{Käsesuppe mit Lauch\/}}
\index{Käse@{Käse\/}!Käsesuppe mit Lauch@{Käsesuppe mit Lauch\/}}
\index{Lauch@{Lauch\/}!Käsesuppe mit Lauch@{Käsesuppe mit Lauch\/}}
\index{vegetarisch@{vegetarisch\/}!Käsesuppe mit Lauch@{Käsesuppe mit Lauch\/}}
}{
\begin{multicols}{2}



300 g Schmelzkäse

1 große gekochte Kartoffel

1 großer Apfel

1 l Brühe

1/4 l Milch

30 g Butter

2 Stangen Porree

2 Eier

1/2 TL Kümmel gemahlen

Muskat


\end{multicols}

Käse und Butter schaumig rühren und mit dem geriebenen Apfel und der zerkleinerten Kartoffel (ersetzbar durch 50 g Mehl) vermischen. Nach Zugabe von Milch, Kümmel und Brühe gut durchkochen. Die Lauchstangen in Ringe schneiden und in der Suppe gar kochen, so dass sie noch Biss haben. Nach Belieben mit Muskat abschmecken und mit Ei abziehen.


{\bfseries Menge:} 4 Portionen

{\bfseries Quelle:} Gerald und Frauke 

} 

%----------------------
\nopagebreak{ 
\subsection{Kürbis-Knoblauch-Suppe}

\index{Knoblauch@{Knoblauch\/}!Kürbis-Knoblauch-Suppe@{Kürbis-Knoblauch-Suppe\/}}
\index{Kürbis@{Kürbis\/}!Kürbis-Knoblauch-Suppe@{Kürbis-Knoblauch-Suppe\/}}
\index{Vegetarisch@{Vegetarisch\/}!Kürbis-Knoblauch-Suppe@{Kürbis-Knoblauch-Suppe\/}}
}{
\begin{multicols}{2}



750 g Kürbisfleisch (geputzt)

3 Zehen Knoblauch

200 g Sahne

Salz

Pfeffer

gekörnte Gemüsebrühe


\end{multicols}

Den Kürbis schälen, entkernen und würfeln. In einen Topf geben und knapp mit Wasser bedecken.
Aufkochen lassen und in ca. 20 Minuten weichkochen, dann im Topf fein pürieren.
Knoblauch schälen und fein hacken, mit der Sahne  in die Supper rühren und mit Salz, Pfeffer und gekörnter Brühe würzen. Nach Belieben mit einer Sahnehaube und mit Schnittlauchröllchen bestreut servieren.

Tipp: Für noch stärkeres Knoblaucharoma: 2-3 Knoblauchzehen schälen und in
dünne Scheiben schneiden. In 1 TL Bueter glasig dünsten und auf die Suppe streuen.


{\bfseries Bemerkung:} kcal: 205 

{\bfseries Menge:} 4 Portionen

{\bfseries Quelle:} Andrea Opperman: Kürbisglück 

} 

%----------------------
\nopagebreak{ 
\subsection{Kürbis-Riesling-Suppe}

\index{Alkohol@{Alkohol\/}!Kürbis-Riesling-Suppe@{Kürbis-Riesling-Suppe\/}}
\index{Kürbis@{Kürbis\/}!Kürbis-Riesling-Suppe@{Kürbis-Riesling-Suppe\/}}
\index{vegetarisch@{vegetarisch\/}!Kürbis-Riesling-Suppe@{Kürbis-Riesling-Suppe\/}}
\index{Wein@{Wein\/}!Kürbis-Riesling-Suppe@{Kürbis-Riesling-Suppe\/}}
}{
\begin{multicols}{2}



750 g Kürbisfleisch (geputzt)

100 g Sahne

100 g Schmand

100 ml Riesling (oder anderer trockener Weißwein)

Salz

Pfeffer

gekörnte Gemüsebrühe


\end{multicols}

Den Kürbis schälen, entkernen und würfeln. In einen Topf geben und knapp mit Wasser bedecken.
Aufkochen lassen und in ca. 20 Minuten weichkochen, dann im Topf fein pürieren.

Sahne, Schmand und Wein unterrühren und mit Salz, Pfeffer und gekörnter Brühe würzen. Nach Belieben mit einer Sahnehaube und etwas gehackter Petersilie bestreut servieren.

Tipp: Nach mehrmaligem Aufwärmen schmeckt die Suppe noch besser.


{\bfseries Bemerkung:} kcal: 205 

{\bfseries Menge:} 4 Portionen

{\bfseries Quelle:} Andrea Oppermann: Kürbisglück 

} 

%----------------------
\nopagebreak{ 
\subsection{Kürbissuppe klassisch}

\index{Kürbis@{Kürbis\/}!Kürbissuppe klassisch@{Kürbissuppe klassisch\/}}
\index{vegetarisch@{vegetarisch\/}!Kürbissuppe klassisch@{Kürbissuppe klassisch\/}}
}{
\begin{multicols}{2}



1 kg Kürbis

60 g Butter

1 Zwiebel, fein gehackt

Pfeffer und Salz nach Geschmack

200 ml Sahne

Basilikum zum Garnieren


\end{multicols}

Kürbis schälen und in mittelgroße Stücke schneiden. Butter in einem Topf erhitzen und die Zwiebeln 15 Minuten schwach andünsten. Hühnerbrühe und Kürbisstücke zufügen. 20 Minuten zugedeckt köcheln lassen, bis der Kürbis gar ist. Portionsweise pürieren. Mit Salz und Pfeffer abschmecken, Sahne einrühren. Die Suppe auf kleiner Flamme unter Rühren nochmals erhitzen. Mit Basilikum garnieren.


{\bfseries Menge:} 4 Portionen

{\bfseries Quelle:} Anne Wilson: Klassieke Soepen 

} 

%----------------------
\nopagebreak{ 
\subsection{Kürbissuppe scharf}

\index{Apfel@{Apfel\/}!Kürbissuppe scharf@{Kürbissuppe scharf\/}}
\index{Kürbis@{Kürbis\/}!Kürbissuppe scharf@{Kürbissuppe scharf\/}}
\index{Vegetarisch@{Vegetarisch\/}!Kürbissuppe scharf@{Kürbissuppe scharf\/}}
}{
\begin{multicols}{2}



750 g Kürbis

1 Apfel

3 EL Öl

1 Zwiebel

1 kleines Stück Ingwer

2 TL Currypaste (oder Curry und Chilipulver)

Wasser

Gemüsebrühe

Pfeffer und Salz nach Geschmack

200 ml Sahne


\end{multicols}

Kürbis und Apfel putzen und in Stücke schneiden. Ingwer und Zwiebel schälen und fenin würfeln.
Öl in einem Topf erhitzen und die Zwiebeln, den Ingwer und die Currypaster schwach andünsten. Kürbisstücke zufügen.

Nach etwas 2 Minuten mit Wasser bedecken und Gemüsebrühe hinzufügen. 20 Minuten zugedeckt köcheln lassen, bis der Kürbis gar ist. Portionsweise pürieren. Mit Salz und Pfeffer abschmecken, Sahne einrühren.
Die Suppe auf kleiner Flamme unter Rühren nochmals erhitzen.


{\bfseries Menge:} 4 Portionen

{\bfseries Quelle:} Frauke und Gerald 

} 

%----------------------
\nopagebreak{ 
\subsection{Linsensuppe mit Orange und Curry}

\index{Linsen@{Linsen\/}!Linsensuppe mit Orange und Curry@{Linsensuppe mit Orange und Curry\/}}
\index{vegetarisch@{vegetarisch\/}!Linsensuppe mit Orange und Curry@{Linsensuppe mit Orange und Curry\/}}
}{
\begin{multicols}{2}



300 g rote Linsen

1 Gemüsezwiebel

1 Knoblauchzehe

1 walnussgroßes Stück Ingwer

Zeste von einer Orange

1 l Brühe

100 ml Sahne

1 1/2 EL Curry


\end{multicols}

Die Zwiebel und den Knoblauch schälen, klein schneiden und in Olivenöl anschwitzen. Die Orangenschale hinzufügen. Den fein geriebenen Ingwer und das Currypulver hinzugeben und ebenfalls mit anschwitzen. Die Linsen dazugeben und mit der Brühe auffüllen. Das Ganze kochen bis die Linsen gar sind. (10-15 min). Sahne dazugeben, nochmals aufkochen und ggf. mit Salz und Zitrone abschmecken.


{\bfseries Menge:} 4 Portionen

{\bfseries Quelle:} www.vox.de: Schmeckt nicht, gibt's nicht 

} 

%----------------------
\nopagebreak{ 
\subsection{Litauische Rote-Bete-Suppe (kalt)}

\index{Rote@{Rote\/}!Litauische Rote-Bete-Suppe (kalt)@{Litauische Rote-Bete-Suppe (kalt)\/}}
\index{Bete@{Bete\/}!Litauische Rote-Bete-Suppe (kalt)@{Litauische Rote-Bete-Suppe (kalt)\/}}
\index{einfach@{einfach\/}!Litauische Rote-Bete-Suppe (kalt)@{Litauische Rote-Bete-Suppe (kalt)\/}}
\index{vegetarisch@{vegetarisch\/}!Litauische Rote-Bete-Suppe (kalt)@{Litauische Rote-Bete-Suppe (kalt)\/}}
}{
\begin{multicols}{2}



1 l Kefir, (ersatzweise Buttermilch, oder griechischer Jogurt),, gekühlt

4 Gurken, klein, in dünne Stiffte, geschnitten

2 Rote Bete, kleine, am besten die in, Apfelsaft eingelegten

1 Bund Dill, klein gehackt

1 Bund Frühlingszwiebeln, in dünne Ringe geschnitten

2 EL Saure Sahne, oder Schmand/Creme fraiche

2 Eier, hart gekocht

Weißweinessig

Salz

Pfeffer


\end{multicols}

Ale Zutaten (außer Sahne und Eier) zusammenmischen,  je nach Konsistenz und Wunsch mit kaltem Wasser verdünnen, mit saurer Sahne verfeinern, mit Salz, weißem Pfeffer  und einem Schuss Weinessig abschmecken, 5 bis 10 Minuten am besten im Kühlschrank ruhen lassen, dann, wenn nötig, noch mal salzen und mit Eierspalten dekorieren.
Dazu gekochte neue Kartoffeln oder Ofenkartoffeln oder auch Pommes frites reichen.


{\bfseries Menge:} 4 Personen

{\bfseries Inhalt pro Portion:} 866~kJ, 13~g Eiweiß, 7~g Fett, 17~g Kohlenhydrate, 208~kcal, 1.42~BE

{\bfseries Zeit:} Gesamtzeit 20 min

{\bfseries Quelle:} chefkoch.de 

} 

%----------------------
\nopagebreak{ 
\subsection{Möhren--Ingwer--Suppe}

\index{Möhren@{Möhren\/}!Möhren--Ingwer--Suppe@{Möhren--Ingwer--Suppe\/}}
\index{Orangen@{Orangen\/}!Möhren--Ingwer--Suppe@{Möhren--Ingwer--Suppe\/}}
\index{vegetarisch@{vegetarisch\/}!Möhren--Ingwer--Suppe@{Möhren--Ingwer--Suppe\/}}
}{
\begin{multicols}{2}



1 Zwiebel

20 g Ingwerwurzel

400 g Möhren

80 g Speckschwarte

2 TL Curry

Zucker

3/4 l Gemüsebrühe

1 Orange (Saft)

Salz

weißer Pfeffer

4 Scheiben mild geräucherter Schinken (100 g)

1 EL gehackte Petersilie


\end{multicols}

Zwiebel, Ingwer und Möhren schälen. Zwiebel und Ingwer würfeln, die Möhren in dünne Scheiben schneiden. Schwarte bei sanfter Hitze langsam auslassen. Zwiebel und Ingwer im Bratfett 2 Minuten anbraten. Möhren, Curry und 1 Prise Zucker dazugeben und weitere 2 Minuten sanft braten, dann mit der Brühe ablöschen. Etwa 30 Minuten köcheln lassen, die Schwarte entfernen und die Möhren mit dem Kartoffelstampfer zerkleinern (oder pürieren). Nochmals erhitzen, den Orangensaft zufügen und die Suppe mit Salz und Pfeffer abschmecken. Den Schinken vom Fettrand befreien, in dünne Streifen schneiden und mit der Petersilie unter die Suppe mischen. Als vegetarische Variante ohne Speck und Schinken zubereiten.


{\bfseries Menge:} 4 Portionen

{\bfseries Quelle:} Leicht genießen 

} 

%----------------------
\nopagebreak{ 
\subsection{Ochsenschwanz-Suppe}

\index{Rindfleisch@{Rindfleisch\/}!Ochsenschwanz-Suppe@{Ochsenschwanz-Suppe\/}}
\index{vorbereiten@{vorbereiten\/}!Ochsenschwanz-Suppe@{Ochsenschwanz-Suppe\/}}
}{
\begin{multicols}{2}



1 1/2 kg Ochsenschwanz, in Stücke, zerteilt

3 Zwiebeln

1/2 kleine Sellerieknolle

2 Möhren

4 EL Pflanzenöl

Salz

Pfeffer

1 EL Rosenpaprika

2 Lorbeerblätter

1/2 TL Thymian

2 EL Tomatenmark

1/4 l Rotwein

4 cl Cognac


\end{multicols}

Die Ochsenschwanz-Stücke waschen und abtrocknen. Zwiebeln, Sellerie und Möhren schälen und in grobe Würfel schneiden.

In einem großen Topf die Fleischstücke in Öl anbraten, bis sie auf allen Seiten Farbe angenommen haben. Dann die Gemüsewürfel dazugeben und weiter unter Rühren anschmoren. Mit Salz, Pfeffer und dem Rosenpaprika würzen, etwa 1,5 l Wasser aufgießen. Die Lorbeerblätter, den Thymian und das Tomatenmark zugeben, langsam offen aufkochen lassen und den Schaum abschöpfen. Dann den Deckel auflegen. Bei geringer Hitze 3, besser noch 4 Stunden leise simmern lassen. Die Ochsenschwanz-Stücke herausnehmen und mit einem spitzen Messer das Fleisch ablösen. Den restlichen Inhalt des Topfes durch ein Sieb passieren, die Rückstände kräftig ausdrücken.

Die Suppe wieder erhitzen und den Wein dazugießen, ohne Deckel etwas einkochen lassen, dann das ausgelöste Fleisch darin erwärmen. Mit Salz, Pfeffer und dem Cognac abschmecken. Dazu serviert man Weißbrot und den Rest des Rotweins.


{\bfseries Menge:} 4 Portionen

{\bfseries Quelle:} Alfred Biolek: Meine Rezepte 

} 

%----------------------
\nopagebreak{ 
\subsection{Pikante Pastinakensuppe}

\index{Pastinaken@{Pastinaken\/}!Pikante Pastinakensuppe@{Pikante Pastinakensuppe\/}}
\index{Porree@{Porree\/}!Pikante Pastinakensuppe@{Pikante Pastinakensuppe\/}}
\index{Vegetarisch@{Vegetarisch\/}!Pikante Pastinakensuppe@{Pikante Pastinakensuppe\/}}
}{
\begin{multicols}{2}



1 1/4 l Hühner- oder Gemüsebrühe

1 weiße Zwiebel, geviertelt,

1 Lauchstange

500 g Pastinaken

30 g Butter

1 EL Madras-Curry

1 TL gemahlener Kreuzkümmel

300 ml Sahne

10 g frische Korianderblätter


\end{multicols}

Die Brühe in einem Topf zum Kochen bringen, während das Gemüse vorbereitet wird. Zwiebel, Lauch und Pastinaken in dünne Scheiben schneiden.

Die Butter in einem großen Topf bei mittlerer Hitze zerlassen. Das Gemüse zufügen und zugedeckt etwa 5 Minuten schmoren. Currypulver und Kreuzkümmel zugeben und eine Minute mit-schmoren. Die Brühe zugiessen und alle Zutaten mit aufgelegtem Deckel bei mittlerer Hitze etwa 10 Minuten garen.

Die Suppe portionsweise in Mixer oder Küchenmaschine glatt pürieren, wieder in den Topf geben und die Sahne unterrühren. Bei niedriger Temperatur noch einmal erhitzten, mit Salz und zerstoßenem schwarzen Pfeffer abschmecken und mit Koriander bestreut servieren.


{\bfseries Bemerkung:} Eiweiß: 3 g, Fett: 25 g, Kohlenhydrate: 12 g, kcal: 295, BE: 1 

{\bfseries Menge:} 6 Portionen

{\bfseries Quelle:} Anne Wilson: Schnelle Suppen und Eintöpfe 

} 

%----------------------
\nopagebreak{ 
\subsection{Rote Linsensuppe}

\index{Garnelen@{Garnelen\/}!Rote Linsensuppe@{Rote Linsensuppe\/}}
\index{Linsen@{Linsen\/}!Rote Linsensuppe@{Rote Linsensuppe\/}}
}{
\begin{multicols}{2}



20 g rote Linsen

1 Bund Suppengemüse

2 Schalotten

1 Knoblauchzehen

25 g Butter

1 TL Paprika rosenscharf

1 EL Paprika edelsüß

400 ml Geflügelbrühe

1/8 l tockener Weißwein

1/8 l Wasser

200 g Schlagsahne

Salz

150 g Tiefseegarnelen


\end{multicols}

Linsen waschen und abtropfen lassen. Suppengemüse putzen, Schalotten und Knoblauch schälen. Alles würfeln und in Butter glasig braten und das Paprikapulver darunterrühren. Mit Brühe, Wein und Wasser ablöschen. Die Linsen zufügen und alles 15 Minuten kochen. Die Suppe leicht abgekühlt mit dem Mixer pürieren. 100 g Sahne steif schlagen. Die übrige Sahne unter die Suppe rühren und diese salzen. Die abgebrausten und abgetropften Garnelen auf Suppentassen verteilen, mit Suppe auffüllen, mit je einer Sahnehaube krönen und mit Paprikapulver bestäuben.


{\bfseries Menge:} 4 Portionen

{\bfseries Quelle:} Winterliches Gemüse. Frisch und pfiffig 

} 

%----------------------
\nopagebreak{ 
\subsection{Rote Zwiebelsuppe}

\index{Alkohol@{Alkohol\/}!Rote Zwiebelsuppe@{Rote Zwiebelsuppe\/}}
\index{vegetarisch@{vegetarisch\/}!Rote Zwiebelsuppe@{Rote Zwiebelsuppe\/}}
\index{Wein@{Wein\/}!Rote Zwiebelsuppe@{Rote Zwiebelsuppe\/}}
\index{Zwiebeln@{Zwiebeln\/}!Rote Zwiebelsuppe@{Rote Zwiebelsuppe\/}}
}{
\begin{multicols}{2}



2 Platten Blätterteig (a 90 g)

300 g rote Zwiebeln

2 Knoblauchzehen

30 g Butter

200 ml Gemüsebrühe

100 ml Weißwein

Salz, Pfeffer

1 Ei, getrennt

1 EL Milch

1 TL grobes Meersalz

1/2 TL Kümmel, ganz


\end{multicols}

Blätterteigplatten nebeneinander auf Backpapier auftauen lassen.

Zwiebeln in feine Ringe schneiden. Knoblauch fein würfeln. Butter in einem Topf zerlassen, Zwiebeln und Knoblauch darin glasig dünsten. Mit Brühe und Weißwein ablöschen und mit Salz und Pfeffer würzen. Zugedeckt bei milder Hitze ca. 20 Minuten kochen lassen.

Blätterteigplatten auf einer bemehlten Arbeitsfläche einzeln zu je einem Quadrat von 24 x 24 cm ausrollen. Die Ränder der ofenfesten Portionssuppenschalen (a ca 250 ml Inhalt) mit Eiweiß bestreichen.

Zwiebelsuppe abschmecken und in die Suppenschalen füllen. Schalen jeweils mit einer Scheibe Blätterteig abdecken, dabei die Ränder gut andrücken. Eigelb und Milch verrühren. Blätterteig vorsichtig mit Eiermilch bestreichen und mit Meersalz und Kümmel bestreuen. Im heißen Ofen bei 200 Grad (Umlauft 200 Grad) auf der mittleren Schiene ca. 12-15 Minuten goldbraun überbacken.


{\bfseries Bemerkung:} Eiweiß: 9 g, Fett: 35 g, Kohlenhydrate: 40 g, kJ: 2264, kcal: 542 

{\bfseries Menge:} 2 Portionen

{\bfseries Quelle:} essen und trinken für jeden Tag 01/08 

} 

%----------------------
\nopagebreak{ 
\subsection{Rote-Bete-Mango-Suppe}

\index{Mango@{Mango\/}!Rote-Bete-Mango-Suppe@{Rote-Bete-Mango-Suppe\/}}
\index{Rote@{Rote\/}!Rote-Bete-Mango-Suppe@{Rote-Bete-Mango-Suppe\/}}
\index{Bete@{Bete\/}!Rote-Bete-Mango-Suppe@{Rote-Bete-Mango-Suppe\/}}
}{
\begin{multicols}{2}



60 g Ingwerwurzel

600 g Rote Bete

6 EL Olivenöl

800 ml Geflügelfond

3 reife Mangos (ca. 350 g)

4 EL Zitronensaft

Salz

Pfeffer

50 g Nordseekrabben

40 g Meerrettichwurzel


\end{multicols}

Für die Suppe den Ingwer schälen und grob schneiden. Rote Bete schälen und auf Backpapier (Farbschutz für die Oberfläche) in ca. 1 cm große Stücke schneiden.

2 EL Olivenöl in einem Topf erhitzen und einen Teil der Ingwerstücke darin kurz dünsten. Rote Bete zugeben, mit Fond auffüllen und abgedeckt 20 Minuten bei mittlerer Hitze leise kochen lassen.

In der Zwischenzeit die Mangos schälen, das Fruchtfleisch von den Kernen schneiden und in grobe Stücke schneiden. 80 g Fruchtfleisch in einem Blitzhacker mit 2 EL Olivenöl und 1 EL Zitronensaft fein pürieren.
Die Rote-Bete-Suppe mit der restlichen Mango im Küchenmixer sehr fein
pürieren, dann zurück in den Topf gießen und mit dem restlichen Ingwer bei milder Hitze 20 Minuten ziehen lassen (gibt einen frischen Ingwergeschmack).
In der Zwischenzeit das restliche Olivenöl in einer Pfanne erhitzen und die Krabben darin knusprig braten. Anschließend auf Küchenpapier abtropfen lassen.

Die Suppe mit dem restlichen Zitronensaft und Salz und Pfeffer abschmecken, durch ein feines Küchensieb gießen und nochmals kurz aufkochen.
Die Meerrettichwurzel schälen und fein mit dem Sparschäler in die Suppe hobeln.

Die Suppe in tiefen Tellern anrichten. Mit dem Mangopüree dekorieren und mit Meerrettich und Krabben bestreuen.

Zubereitungszeit: 1 Stunde


{\bfseries Menge:} 6 Portionen

{\bfseries Quelle:} www.vox.de: ganz und gar henssler vom 23.10.2007 

} 

%----------------------
\nopagebreak{ 
\subsection{Sauer-scharfe Suppe (Pekingart)}

\index{asiatisch@{asiatisch\/}!Sauer-scharfe Suppe (Pekingart)@{Sauer-scharfe Suppe (Pekingart)\/}}
\index{Huhn@{Huhn\/}!Sauer-scharfe Suppe (Pekingart)@{Sauer-scharfe Suppe (Pekingart)\/}}
\index{Pilze@{Pilze\/}!Sauer-scharfe Suppe (Pekingart)@{Sauer-scharfe Suppe (Pekingart)\/}}
\index{vorbereiten@{vorbereiten\/}!Sauer-scharfe Suppe (Pekingart)@{Sauer-scharfe Suppe (Pekingart)\/}}
}{
\begin{multicols}{2}



2 EL Stärkemehl

4 EL Wasser

2 EL helle Sojasauce

3 EL Reisweinessig oder Balsamico

1/2 TL gemahlener schwarzer Pfeffer

1 frischer, kleiner, roter, fein geschnittener Chili

1 Ei

2 EL Pflanzenöl

1 klein geschnittene Zwiebel

850 ml klare Hühner- oder Rindfleischbrühe

1 Pilz mit offenem breitem Hut, in Scheiben geschnitten

50 g Hähnchenbrustfilet, in dünne Streifen geschnitten

1 TL Sesamöl


\end{multicols}

In einer kleinen Schüssel das Stärkemehl mit dem Wasser glatt rühren.

Die Sojasauce, den Reisweinessig und den schwarzen Pfeffer untermischen.

Den roten Chili klein schneiden und zu den Zutaten in der Schüssel hinzufügen. Gut durchmischen.

Das Ei in einer anderen Schüssel aufschlagen und gut verquirlen. Beiseite stellen.

Das Öl in einen vorher erhitzen Wok (oder anderen Topf) gießen, erhitzen und die Zwiebel 1-2 Minuten anbraten, bis sie weich und glasig ist.

Die klare Brühe, Pilze und das Hähnchenfleisch einrühren und zum Kochen bringen. 15 Minuten garen oder so lange, bis das Hähnchenfleisch zart ist.

Die Stärkemehlmischung langsam in die Suppe gießen und unter ständigem Rühren kochen, bis sie dickflüssig wird.

Während des Rührens das Ei in die Suppe träufeln. So entstehen fadenartige Strukturen aus Ei.

Die sauer-scharfe Suppe in eine vorgewärmte Suppenterrine oder in einzeln Suppentassen füllen, mit dem Sesamöl beträufeln und heiß servieren.


{\bfseries Bemerkung:} Eiweiß: 5 g, Fett: 8 g, Kohlenhydrate: 8 g, kcal: 124 

{\bfseries Menge:} 4 Portionen

{\bfseries Quelle:} Asiatisch kochen -- Die besten Wok- und Pfannengerichte 

} 

%----------------------
\nopagebreak{ 
\subsection{Scharfe Tomaten- und Erbsensuppe}

\index{Erbsen@{Erbsen\/}!Scharfe Tomaten- und Erbsensuppe@{Scharfe Tomaten- und Erbsensuppe\/}}
\index{Tomaten@{Tomaten\/}!Scharfe Tomaten- und Erbsensuppe@{Scharfe Tomaten- und Erbsensuppe\/}}
\index{vegetarisch@{vegetarisch\/}!Scharfe Tomaten- und Erbsensuppe@{Scharfe Tomaten- und Erbsensuppe\/}}
}{
\begin{multicols}{2}



5 große Tomaten

500 ml Wasser

1 große Zwiebel

1/2 TL zerdrückter Knoblauch

2 EL Butter

2 TL Korianderpulver

2 TL Kreuzkümmelpulver

1/2 TL Fenchelsamen

2 Lorbeerblätter

1 grüne Chilischote

400 ml Kokosmilch

275 g tiefgefrorene Erbsen

1 EL Zucker

gemahlener Pfeffer

1 EL gehackte frische Minze


\end{multicols}

Tomaten klein schneiden, im Wasser köcheln lassen, bis sie sehr weich sind. Mit dem Mixer pürieren. (Oder: Tomaten und Wasser durch fertiges Tomatenpüree (0.75-1 l) ersetzen.) Zwiebel in dünne Scheiben schneiden und zusammen mit dem Knoblauch in der Butter glasig braten. Gewürze, Lorbeerblätter und Chili dazugeben und 1-2 Minuten kochen lassen. Kokosmilch und die pürierten Tomaten dazurühren und aufkochen lassen. Erbsen hinzufügen und bei schwacher Hitze so lange kochen, bis die Erbsen weich sind. Mit Zucker und Pfeffer würzen und die gehackte Minze unterrühren. Heiß servieren und dazu geröstetes Fladenbrot reichen.


{\bfseries Menge:} 4 Portionen

{\bfseries Quelle:} Anne Wilson: Indische Küche 

} 

%----------------------
\nopagebreak{ 
\subsection{Schwedische Gurkensuppe mit Lachs}

\index{Gurken@{Gurken\/}!Schwedische Gurkensuppe mit Lachs@{Schwedische Gurkensuppe mit Lachs\/}}
\index{Lachs@{Lachs\/}!Schwedische Gurkensuppe mit Lachs@{Schwedische Gurkensuppe mit Lachs\/}}
}{
\begin{multicols}{2}



2 Kartoffeln

6 Gurken

2 Zwiebeln

3 Bund Dill

100 g Butter

2 Zitronen

Salz

Pfeffer

3/4 l Fleischbrühe

3 Becher Sahne

1 Becher Crème fraîche

500 g Lachs


\end{multicols}

Kartoffeln schälen, kochen. Gurken schälen, halbieren, Kerne mit einem Löffel entfernen und in ca. 3 cm große Stücke schneiden. Zwiebeln sehr klein schneiden und in Butter andünsten, wegstellen.

Die Gurkenstücke in ca. 5 Portionen in Butter gut andünsten. Anschließend alles im Mixer pürieren. Eine Zitrone, Salz, Pfeffer und 3/4 l schwache Fleischbrühe dazugeben und etwa 30 Minuten ziehen lassen. Vorsicht, brennt leicht an!

Den Lachs in dünne Streifen schneiden und den Dill in kleine Stücke zerteilen. Kurz vor dem Anrichten die Sahne und Crème fraîche in die Suppe geben und mit Salz, Pfeffer und Zitrone abschmecken. In die vorgewärmten Teller je 4-5 Streifen Lachs legen, die Suppe darauf geben und mit Dill bestreuen. Schnell servieren.


{\bfseries Menge:} 10 Portionen

{\bfseries Quelle:} Internet, Bonner Generalanzeiger 

} 

%----------------------
\nopagebreak{ 
\subsection{Tomatierte Hühnerherzensuppe}

\index{Geflügel@{Geflügel\/}!Tomatierte Hühnerherzensuppe@{Tomatierte Hühnerherzensuppe\/}}
\index{Huhn@{Huhn\/}!Tomatierte Hühnerherzensuppe@{Tomatierte Hühnerherzensuppe\/}}
\index{Tomaten@{Tomaten\/}!Tomatierte Hühnerherzensuppe@{Tomatierte Hühnerherzensuppe\/}}
}{
\begin{multicols}{2}



500 g Hühnerherzen

Tomatenmark

1 l Brühe (Klare Gemüsebrühe oder Hühnerbrühe)


\end{multicols}

Man nehme sich die Hühnerherzen und schneide überflüssige Arterien und Venen einfach ab (so dass fast nur noch Muskelfleisch übrigbleibt). dann gebe man sie in einen Topf mit Wasser und erhitze diesen bis das Wasser den Siedepunkt erreicht (und darüber hinaus). nach etwa 5 bis 10 inuten kann man den braunen Schaum abschöpfen oder einfach das Wasser mit im Wasserkocher vorbereitetem heissem Wasser tauschen. dann lässt man den opf einfach in Ruhe köcheln nochmal etwa 10 bis 15 Minuten. nachdem man sich vergewissert hat, dass die Herzen durch sind, fülle man den Topf mit weiterem Wasser auf und rühre die Brühe ein, bis es geschmacklich passt. dann kommt noch mindestens eine kleine Dose Tomatenmark dazu. Zwischendurch kosten, rühren, kurz aufkochen lassen (Thymian dazu) und fertig.


{\bfseries Menge:} 4 Portionen

{\bfseries Quelle:} Goldbroiler Rostock 

} 

%----------------------
\nopagebreak{ 
\subsection{Ukrainische Soljanka}

\index{einfach@{einfach\/}!Ukrainische Soljanka@{Ukrainische Soljanka\/}}
\index{Resteverwertung@{Resteverwertung\/}!Ukrainische Soljanka@{Ukrainische Soljanka\/}}
}{
\begin{multicols}{2}



2 Zwiebeln

2 EL Margarine

1 l Brühe

2 EL Tomatenmark

4 kleine Gewürzgurken

300 g gegartes Fleisch oder Würstchen

1 EL Kapern

Salz

1 Lorbeerblatt

saure Sahne

1 Zitrone

Petersilie oder Dill


\end{multicols}

Zwiebeln hacken, in der Margarine goldgelb anrösten. Mit etwas Fleischbrühe ablöschen und mit dem Tomatenmark dünsten. Fleisch und Gurken in Stücke schneiden und zu den Zwiebeln geben. Kapern, Salz, Lorbeerblatt zufügen, mit Brühe auffüllen und 10 Minuten auf kleiner Flamme kochen. Beim Servieren mit etwas saurer Sahne, 1 Zitronenscheibe und gehackter Petersilie oder Dill garnieren.


{\bfseries Menge:} 4 Portionen

{\bfseries Quelle:} Muttis Kochbuch 

} 

%----------------------
\nopagebreak{ 
\subsection{Wildkraftbrühe mit Steinpilzen und Gemüsejuliennes}

\index{Pilze@{Pilze\/}!Wildkraftbrühe mit Steinpilzen und Gemüsejuliennes@{Wildkraftbrühe mit Steinpilzen und Gemüsejuliennes\/}}
\index{Wild@{Wild\/}!Wildkraftbrühe mit Steinpilzen und Gemüsejuliennes@{Wildkraftbrühe mit Steinpilzen und Gemüsejuliennes\/}}
}{
\begin{multicols}{2}



2 kg gehackte Wildknochen (am besten im Oktober)

1/2 l Rotwein

2 Bund

Wasser

Salz

Portwein oder Sherry

Möhre

Lauch

Sellerie

Steinpilze


\end{multicols}

Die gehackten Wildknochen anbraten, bis sie schön braun sind, dann mit dem Rotwein ablöschen. Das Suppengemüse dazu, mit Wasser auffüllen, salzen und ca. 2 Stunden leicht kochen lassen. Anschließend durch ein sehr feines Sieb passieren und durch ein Tuch filtern. Die Brühe  erkalten lassen.

Die Brühe entfetten. Mit etwas Portwein oder Sherry abschmecken. Möhre, Lauch, Sellerie und Stenpilze in ganz feine Streifen schneiden und in der Brühe ziehen lassen.


{\bfseries Menge:} 4 Portionen

{\bfseries Quelle:} NDR Rezeptdatenbank (Schleswig-Holstein-Magazin) 

} 

\pagebreak 
 
%----------------------------------------------------
\nopagebreak{ 
\section{NACHTISCH} 

\subsection{Apfelcrumble}

\index{Apfel@{Apfel\/}!Apfelcrumble@{Apfelcrumble\/}}
\index{Dessert@{Dessert\/}!Apfelcrumble@{Apfelcrumble\/}}
}{
\begin{multicols}{2}



4 mittelgroße Äpfel

4 TL Zitronensaft

6 Messerspitzen Zimtpulver

8 EL Ahornsirup

4 EL getr. Cranberries

10 g weiche Butter

60 g kalte Butter

2 gehäufte EL Mehl

2 EL kernige Haferflocken

2 gestrichene EL Zucker

2 Prisen Salz


\end{multicols}

Die Äpfel vierteln, schälen, entkernen und in 2 cm große Würfel schneiden. Mit Zitronensaft, Zimtpulver, Ahornsirup und den Cranberries vermischen und in vier gefettete Tassen füllen.

Die Butter in kleine Stücke schneiden. Mehl, Haferflocken, Butter, Zucker und Salz mit dem Knethaken des Handrührers oder den Händen zu groben Streuseln kneten. Über die Äpfel krümeln und im vorgeheizten Ofen auf der zweiten Schiene von unten bei 190$^\circ$C (Umluft 170 $^\circ$C) ca. 30 Minuten backen.

Zubereitungszeit: 40 Minuten


{\bfseries Menge:} 4 Portionen

{\bfseries Quelle:} www.vox.de: Schmeckt nicht, gibt's nicht 

} 

%----------------------
\nopagebreak{ 
\subsection{Bratäpfel mit Nuss-Gelee-Füllung}

\index{Apfel@{Apfel\/}!Bratäpfel mit Nuss-Gelee-Füllung@{Bratäpfel mit Nuss-Gelee-Füllung\/}}
\index{Vegetarisch@{Vegetarisch\/}!Bratäpfel mit Nuss-Gelee-Füllung@{Bratäpfel mit Nuss-Gelee-Füllung\/}}
}{
\begin{multicols}{2}



2 große Äpfel

2 TL Butter

2 TL Zucker

2 EL Quitten- oder Apfelgelee

2 EL gehackte Walnüsse

2 EL Rosinen

2 EL Grappa

2 Prisen Zimt


\end{multicols}

Äpfel in der Mitte mit einem Apfelausstecher ausstechen und das Kerngehäuse vorsichtig entfernen.

Butter mit Zucker, Gelee, Nüssen, Rosinen, Schnaps und Zimt vermischen. Diese Masse in die Apfelhöhlung drücken und auf ein Blech oder in eine Form setzen.
Im Ofen bei 180 C/Umluft 160 C 40 Minuten backen. Dazu schmeckt Vanillesauce oder -eis.


{\bfseries Menge:} 2 Portionen

{\bfseries Quelle:} www.daskochrezept.de, abgewandelt von Gerald und Frauke 

} 

%----------------------
\nopagebreak{ 
\subsection{Kirschauflauf}

\index{Auflauf@{Auflauf\/}!Kirschauflauf@{Kirschauflauf\/}}
\index{Kirschen@{Kirschen\/}!Kirschauflauf@{Kirschauflauf\/}}
\index{vegetarisch@{vegetarisch\/}!Kirschauflauf@{Kirschauflauf\/}}
}{
\begin{multicols}{2}



500 g Schattenmorellen

3 EL Mehl

60 g Zucker

1 Prise Salz

4 Eier

1/4 l Milch

4 EL Cognac oder Kirschwasser

Puderzucker zum Bestäuben


\end{multicols}

Den Backofen auf 175 Grad vorheizen. Eine flache Auflaufform (1.5l) mit Butter einstreichen und mit Zucker ausstreuen. Die Kirschen gegebenenfalls entsteinen und in der Form verteilen.

Für den Teig das Mehl mit dem Zucker und dem Salz in eine Schüssel geben. Die Eier hineinschlagen und gut verquirlen. Die Hälfte der Milch unterrühren, bis ein glatter Teig entstanden ist. Zum Schluss die restliche Milch unterrühren.

Den Teig über die Kirschen geben und 34 bis 45 Minuten backen. Den heißen Auflauf mit Cognac oder Kirschwasser beträufeln. Der Auflauf wird beim Abkühlen etwas zusammenfallen. Unmittelbar vor dem Auftragen mit Puderzucker bestäuben und heiß oder noch warm servieren.


{\bfseries Menge:} 6 Portionen

{\bfseries Quelle:} Anne Willan: Auf den Punkt 

} 

%----------------------
\nopagebreak{ 
\subsection{Pikante Karamelsauce mit Pfirsichen und Vanilleeis}

\index{kalorienreich@{kalorienreich\/}!Pikante Karamelsauce mit Pfirsichen und Vanilleeis@{Pikante Karamelsauce mit Pfirsichen und Vanilleeis\/}}
\index{Pfirsiche@{Pfirsiche\/}!Pikante Karamelsauce mit Pfirsichen und Vanilleeis@{Pikante Karamelsauce mit Pfirsichen und Vanilleeis\/}}
\index{vegetarisch@{vegetarisch\/}!Pikante Karamelsauce mit Pfirsichen und Vanilleeis@{Pikante Karamelsauce mit Pfirsichen und Vanilleeis\/}}
}{
\begin{multicols}{2}

\textit{Pfirsiche}



6 Pfirsiche

3 EL brauner Zucker

3 EL Butter

Vanilleeis

\textit{Sauce}



200 g Zucker

1/8 l Wasser

60 g gesalzene Butter

250 g Crème fraîche


\end{multicols}

Den Backofengrill vorheizen. Die Pfirsiche halbieren, entsteinen und mit der Schnittfläche nach oben in eine gebutterte Auflaufform legen. Mit dem braunen Zucker bestreuen und mit Butterflöckchen besetzen. Den Grillrost etwa 7,5 cm unterhalb des Grills einschieben und die Pfirsiche 7 bis 10 Minuten grillen, bis sie weich sind und braun zu werden beginnen.

In der Zwischenzeit einen dunklen Karamelsirup zubereiten. Dazu den Zucker mit dem Wasser in einem Topf mit schwerem Boden erhitzen, bis sich der Zucker aufgelöst hat. Dabei ein- oder zweimal umrühren. Dann rasch kochen, bis der Sirup am Rand braun zu werden beginnt. Dabei nicht rühren. Anschließend 1 bis 2 Minuten bei mittlerer Hitze weiterkochen und den Topf dabei gelegentlich schwenken, damit der Sirup gleichmäßig bräunt.

Sobald der Karamel die richtige Farbe hat, den Topf vom Herd nehmen und warten, bis der Karamel aufhört zu kochen. Dann die Butter unterrühren und die Crème fraîche hineingießen. Dabei vom Herd zurücktreten, denn der Karamel spritzt und wirft Blasen.

Die heißen Pfirsiche auf vier Teller verteilen und mit dem Saft begießen. Je eine Kugel Vanilleeis daneben anrichten. Die Sauce gegebenenfalls nochmal erhitzen, bis sich der Karamel vollständig verflüssigt hat, die Pfirsiche damit begießen und sofort servieren.


{\bfseries Menge:} 4 Portionen

{\bfseries Quelle:} Anne Willan: Auf den Punkt 

} 

%----------------------
\nopagebreak{ 
\subsection{Pochierte Pfirsiche mit Dickmilch-Kaltschale}

\index{Dickmilch@{Dickmilch\/}!Pochierte Pfirsiche mit Dickmilch-Kaltschale@{Pochierte Pfirsiche mit Dickmilch-Kaltschale\/}}
\index{Pfirsiche@{Pfirsiche\/}!Pochierte Pfirsiche mit Dickmilch-Kaltschale@{Pochierte Pfirsiche mit Dickmilch-Kaltschale\/}}
\index{vegetarisch@{vegetarisch\/}!Pochierte Pfirsiche mit Dickmilch-Kaltschale@{Pochierte Pfirsiche mit Dickmilch-Kaltschale\/}}
}{
\begin{multicols}{2}



1 l Dickmilch

3 EL Kokossirup (ersatzweise Honig)

200 ml Schlagsahne

abgeriebene Schale und Saft von 1

Zitrone (unbehandelt)

4 Pfirsiche

125 ml trockener Weißwein

100 g Zucker

1 Vanilleschote

8 Hobbitkekse


\end{multicols}

Pfirsiche oben und unten leicht einschneiden und in kochendem Wasser für ca.
10-15 Sekunden blanchieren. Dann die Haut entfernen und das Fruchtfleisch in kleine Würfel schneiden.

Dickmilch, Sahne, Zitronenschale, Zitronensaft und Kokossirup glatt rühren und kalt stellen.

Weißwein, die gleiche Menge Wasser und Zucker aufkochen. Vanilleschote halbieren und in die Flüssigkeit geben. Pfirsichwürfel darin bei mittlerer Hitze 4 Minuten sanft kochen lassen. Anschließend auf Zimmertemperatur oder lauwarm abkühlen lassen.

Zum Servieren die Kaltschale in tiefe Teller geben, etwas von den Pfirsichen in die Mitte geben und mit den zerbröselten Keksen bestreut servieren.


{\bfseries Menge:} 4 Portionen

{\bfseries Quelle:} www.vox.de: Schmeckt nicht, gibt's nicht 2006-05-24 

} 

%----------------------
\nopagebreak{ 
\subsection{Zwiebackauflauf}

\index{kalorienreich@{kalorienreich\/}!Zwiebackauflauf@{Zwiebackauflauf\/}}
\index{vegetarisch@{vegetarisch\/}!Zwiebackauflauf@{Zwiebackauflauf\/}}
\index{Zwieback@{Zwieback\/}!Zwiebackauflauf@{Zwiebackauflauf\/}}
}{
\begin{multicols}{2}



Zwieback

Butter

4 Eier

1 l Milch

Soßenpulver

Vanillinzucker

Mandeln

Rosinen


\end{multicols}

Zwieback mit Butter bestreichen, in eine Schale einschichten, Eier mit Zucker schlagen, Milch und Soßenpulver zugeben, evtl. auch Vanillinzucker, Mandeln oder Rosinen. Den Zwieback damit begießen und die verschlossene Schale  mit Inhalt im Wasserbad kochen.


{\bfseries Menge:} 4 Portionen

{\bfseries Quelle:} Muttis Kochbuch 

} 

\pagebreak 
 
%----------------------------------------------------
\nopagebreak{ 
\section{BEILAGE} 

\subsection{Apfelrotkohl}

\index{Apfel@{Apfel\/}!Apfelrotkohl@{Apfelrotkohl\/}}
\index{Rotkohl@{Rotkohl\/}!Apfelrotkohl@{Apfelrotkohl\/}}
\index{vegetarisch@{vegetarisch\/}!Apfelrotkohl@{Apfelrotkohl\/}}
}{
\begin{multicols}{2}



800 g Rotkohl

50 g Schweineschmalz oder

3 EL Öl

1 Zwiebel

2 Nelken

1 TL Zucker

1/2 TL Salz

1/4 l Wasser oder Brühe

3 EL milder Essig oder Rotwein

2 herbe Äpfel (z.B. Boskop)


\end{multicols}

Den gehobelten Rotkohl in dem siedenden Fett so lange umwenden, bis er zusammenfällt. Dann die kleingeschnittenen Äpfel (ohne Schale und Kernhaus), die mit Nelken besteckte Zwiebel, Salz, Zucker, Essig sowie das heiße Wasser zufügen und das Kraut zugedeckt gar dünsten. Mit Essig, Zucker und Salz süßsauer abschmecken.


{\bfseries Menge:} 4 Portionen

{\bfseries Quelle:} Wir kochen gut (variiert von Gerald und Frauke) 

} 

%----------------------
\nopagebreak{ 
\subsection{Blechgemüse}

\index{Feta@{Feta\/}!Blechgemüse@{Blechgemüse\/}}
\index{Möhren@{Möhren\/}!Blechgemüse@{Blechgemüse\/}}
\index{vegetarisch@{vegetarisch\/}!Blechgemüse@{Blechgemüse\/}}
\index{Zucchini@{Zucchini\/}!Blechgemüse@{Blechgemüse\/}}
}{
\begin{multicols}{2}



1 kg Zucchini

800 g Möhren

1 rote Pfefferschote

8 EL Öl

4 TL Zimt

Salz

Pfeffer

1 Prise Zucker

4 TL Zitronensaft

300 g Schafsmilch-Feta


\end{multicols}

Zucchini und Möhren putzen. Zucchini quer halbieren, das Kerngehäuse entfernen und anschließend längs vierteln. Möhren längs vierteln, Pfefferschote in Scheiben schneiden.

Das Gemüse in einer Schüssel mit dem Öl, Zimt, Salz, Pfeffer, Zucker und Zitronensaft mischen und dann auf einem Backblech verteilen. Das Ganze nun im vorgeheizten Ofen bei 200$^\circ$C (Umluft nicht empfehlenswert) 15 Minuten backen. Anschließend den Schafsmilch-Feta über das Gemüse bröseln und weitere 10 Minuten backen.


{\bfseries Menge:} 4 Portionen

{\bfseries Quelle:} www.vox.de: Schmeckt nicht, gibt's nicht 2006-05-24 

} 

%----------------------
\nopagebreak{ 
\subsection{Couscous mit Dörrobst}

\index{Couscous@{Couscous\/}!Couscous mit Dörrobst@{Couscous mit Dörrobst\/}}
\index{vegetarisch@{vegetarisch\/}!Couscous mit Dörrobst@{Couscous mit Dörrobst\/}}
}{
\begin{multicols}{2}



200 g gemischtes Dörrobst

120 g Nüsse nach Wahl, z.B.: Walnüsse, Pecannüsse, Pinienkerne o.a.

250 g Couscous

300 ml Gemüsebrühe

3 EL Zitronensaft

1 EL Olivenöl


\end{multicols}

Das Couscous in einer Schüssel mit einem Esslöffel Öl vermengen. Das Dörrobst klein schneiden. Die Nüsse zermahlen und einer Pfanne ohne Fett goldbraun rösten. Beides mit dem Couscous mischen und mit der heißen Brühe übergießen. 5 Minuten quellen lassen, dann den Couscous mit einer Gabel auflockern, mit Salz, Pfeffer und Zitronensaft abschmecken und warm halten.


{\bfseries Menge:} 6 Portionen

{\bfseries Quelle:} www.vox.de: Schmeckt nicht, gibt's nicht 

} 

%----------------------
\nopagebreak{ 
\subsection{Ganzer Reis nach arabischer Art}

\index{arabisch@{arabisch\/}!Ganzer Reis nach arabischer Art@{Ganzer Reis nach arabischer Art\/}}
\index{Reis@{Reis\/}!Ganzer Reis nach arabischer Art@{Ganzer Reis nach arabischer Art\/}}
\index{vegetarisch@{vegetarisch\/}!Ganzer Reis nach arabischer Art@{Ganzer Reis nach arabischer Art\/}}
}{
\begin{multicols}{2}



3 Tassen Reis

Butter, Öl

1 TL Salz

1/2 TL Curry

5 Tassen Wasser


\end{multicols}

Reis in ein Sieb geben, viel spülen, bis das Wasser klar ist. In einem Topf etwas Butter, Öl oder Margarine warmmachen und Reis dazugeben, würzen mit Salz und Curry, Wasser daraufgeben. 10 Minuten kochen, 10 Minuten wenig kochen. Wenn der Reis zu nass ist -- Deckel abnehmen, ist er zu trocken -- etwas Wasser zugeben.


{\bfseries Menge:} 4 Portionen

{\bfseries Quelle:} Muttis Kochbuch 

} 

%----------------------
\nopagebreak{ 
\subsection{Gnocchi alla romana}

\index{Mais@{Mais\/}!Gnocchi alla romana@{Gnocchi alla romana\/}}
\index{P6@{P6\/}!Gnocchi alla romana@{Gnocchi alla romana\/}}
\index{vegetarisch@{vegetarisch\/}!Gnocchi alla romana@{Gnocchi alla romana\/}}
\index{vorbereiten@{vorbereiten\/}!Gnocchi alla romana@{Gnocchi alla romana\/}}
}{
\begin{multicols}{2}



1 l Milch (ggf. etwas mehr)

1 Zwiebeln mit 1 Nelke gespickt

1 Lorbeerblatt

1 Messerspitze gemahlene Muskatnuss

Salz

Pfeffer

150 g grobes gelbes Maismehl oder Maisgrieß

3 Eigelb

1 TL Dijonsenf

125 g geriebener Parmesan

45 g Butter, zerlassen


\end{multicols}

Ein Backblech dick mit Butter einstreichen. Die Milch in einem Topf mit der Zwiebel, dem Lorbeerblatt, den Pfefferkörnern, dem Muskat und etwas Salz erhitzen. Den Topf zudecken, die Milch kurz aufkochen lassen und dann 10 bis 15 Miunter bei schwacher Hitze ziehen lassen. Anschließend durchseihen und in den Topf zurückgießen.

Das Maismehl oder den Grieß langsam in die heiße Milch einrieseln lassen, zum Kochen bringen und dann 8 bis 10 Minuten unter Rühren köcheln lassen, bis die Masse dick ist und sich von den Topfwänden löst.

Den Topf vom Herd nehmen und die Eigelbe einzeln unterschlagen. Den Senf und die Hälfte des Käses untermischen und mit Salz und Pfeffer abschmecken (die Polenta sollte kräftig gewürzt sein).

Die Masse 1cm dick auf das Backblech streichen, mit der zerlassenen Butter bestreichen und 1 bis 2 Stunden kalt stellen, bis die Polenta fest ist. Den Backofen auf 23$^\circ$C (Gas Stufe 4-5) vorheizen.

Sobald die Polenta fest ist, mit einem Glas oder einem Teigausstecher 5 cm große Kreise ausstechen oder die Masse in 5 cm große Quadrate zerteilen. Diese schuppenförmig in eine Auflaufform schichten und mit dem restlichen Käse bestreuen. Im vorgeheizten Ofen 5 bis 10 Minuten überbacken, bis sie sehr heiß und schön braun sind, und sofort servieren.


{\bfseries Menge:} 6 Portionen

{\bfseries Quelle:} Anne Willan: Auf den Punkt 

} 

%----------------------
\nopagebreak{ 
\subsection{Graupen-Risotto}

\index{Graupen@{Graupen\/}!Graupen-Risotto@{Graupen-Risotto\/}}
\index{Käse@{Käse\/}!Graupen-Risotto@{Graupen-Risotto\/}}
\index{Lauch@{Lauch\/}!Graupen-Risotto@{Graupen-Risotto\/}}
\index{Möhren@{Möhren\/}!Graupen-Risotto@{Graupen-Risotto\/}}
\index{Sellerie@{Sellerie\/}!Graupen-Risotto@{Graupen-Risotto\/}}
\index{vegetarisch@{vegetarisch\/}!Graupen-Risotto@{Graupen-Risotto\/}}
}{
\begin{multicols}{2}



75 g Möhre

75 g Lauch

75 g Sellerie

1 Zwiebel

1 EL Öl

50 g Graupen

325 ml Gemüsefond

1 EL gehackte Petersilie

50 ml Sahne

25 g geriebener Parmesan

Salz

Pfeffer


\end{multicols}

Möhre, Lauch und Knollensellerie putzen und fein würfeln. Zwiebel fein hacken. Öl in einem Topf erhitzen und das Gemüse darin andünsten. Graupen dazugeben und kurz mitdünsten. Gemüsefond dazugießen und aufkkochen. Mit halb geschlossenem Deckel 35 min unter häufigem Rühren garen. Am Ende der Garzeit die gehackte Petersilie, Sahne und Parmesan unterrühren. Salzen und Pfeffern.


{\bfseries Bemerkung:} Eiweiß: 17 g, Fett: 9 g, Kohlenhydrate: 23 g, kJ: 1190, kcal: 284 

{\bfseries Menge:} 2 Portionen

{\bfseries Quelle:} essen und trinken für jeden Tag 10/2005 

} 

%----------------------
\nopagebreak{ 
\subsection{Grüne Bohnen italienisch}

\index{Bohnen@{Bohnen\/}!Grüne Bohnen italienisch@{Grüne Bohnen italienisch\/}}
\index{vegetarisch@{vegetarisch\/}!Grüne Bohnen italienisch@{Grüne Bohnen italienisch\/}}
}{
\begin{multicols}{2}



350 g grüne Bohnen

3 Knoblauchzehen

10 Kirschtomaten, halbiert

2 EL Olivenöl

Salz, Pfeffer

1 EL gehacktes Basilikum


\end{multicols}

Grüne Bohnen putzen. Knoblauchzehen fein hacken. Olivenöl in einer Pfanne erhitzen, Knoblauch darin bei mittlerer Hitze andünsten. Bohnen dazugeben, kurz anschwitzen und mit Wasser abgießen. Die Bohnen 8-10 Minuten zugedeckt dünsten.

Zwei Minuten vor Ende der Garzeit die Tomaten dazugeben, salzen und pfeffern. Zum Schluss gehacktes Basilikum dazugeben.

Passt gut zu Schweinemedaillons auf Rosmarinspieß.


{\bfseries Menge:} 4 Portionen

{\bfseries Quelle:} www.vox.de. Schmeckt nicht, gibt's nicht 

} 

%----------------------
\nopagebreak{ 
\subsection{Indischer Zitronenreis}

\index{Reis@{Reis\/}!Indischer Zitronenreis@{Indischer Zitronenreis\/}}
}{
\begin{multicols}{2}



375 g Reis (weißer Langkornreis)

600 ml Wasser

1/2 TL Kurkuma

2 EL Kokosraspel

2 EL Milch oder Kokosmilch

2 EL Cashewnüsse, gehackte und geröstete oder Mandeln

4 Curryblätter

1/3 TL Senfkörner

1 Chilischote(n), grüne, entkernt und gehackt

100 g Butterschmalz (Ghee)

2 Zitronen


\end{multicols}

Den Reis mit Kurkuma im Wasser aufkochen lassen. Zugedeckt weitere 12 Minuten köcheln lassen.

Die Milch oder Kokosmilch über die Kokosraspeln gießen und beiseite stellen.

Die Nüsse, Curryblätter, Senfkörner und Chili im Ghee rösten, bis die Senfkörner allmählich aufzuplatzen beginnen.
Diese Mischung und die Kokosraspeln in den Reis einrühren und mit dem Saft einer Zitrone beträufeln. Nochmals zugedeckt 7-8 Minuten kochen lassen, bis der Reis gar ist.

Abschmecken und  je nach Geschmackden Saft der zweiten Zitrone unterrühren oder vor dem Servieren den Reis mit Zitronenscheiben garnieren.


{\bfseries Menge:} 6 Portionen

{\bfseries Quelle:} Anne Wilson: Indische Küche 

} 

%----------------------
\nopagebreak{ 
\subsection{Polenta}

\index{Mais@{Mais\/}!Polenta@{Polenta\/}}
\index{vegetarisch@{vegetarisch\/}!Polenta@{Polenta\/}}
}{
\begin{multicols}{2}



250 g Maisgrieß

1 l Wasser (evtl. die Hälfte Milch)

1 TL Salz

2 EL Öl oder

50 g Butter


\end{multicols}

Flüssigkeit zum Kochen bringen, Polentagrieß unter stetem Rühren in die Flüssigkeit einrieseln lassen, zu einem dicken Brei koch (ca. 15 min), evtl. etwas Flüssigkeit zufügen. Butter oder Öl zugeben, dann 10 min ziehen lassen.


{\bfseries Menge:} 4 Portionen

{\bfseries Quelle:} Polentapackung 

} 

%----------------------
\nopagebreak{ 
\subsection{Rosmarinkartoffeln}

\index{Kartoffeln@{Kartoffeln\/}!Rosmarinkartoffeln@{Rosmarinkartoffeln\/}}
\index{Rosmarin@{Rosmarin\/}!Rosmarinkartoffeln@{Rosmarinkartoffeln\/}}
\index{vegetarisch@{vegetarisch\/}!Rosmarinkartoffeln@{Rosmarinkartoffeln\/}}
}{
\begin{multicols}{2}



500 g möglichst kleine Kartoffeln

einige Knoblauchzehen

frischer und/oder getrockneter Rosmarin

Salz

2 EL Olivenöl

\bild{image/cookbook_4.jpg}


\end{multicols}

Die Kartoffeln abwaschen und bürsten, etwas trocken tupfen und halbieren.

Eine Auflaufform mit etwas Öl ausstreichen und die Kartoffelhälften mit der Schnittfläche nach unten hineinsetzen.
Es sollten möglichst alle Kartoffelhälften nebeneinander passen. Mit reichlich Salz bestreuen, Rosmarin dazulegen sowie ganze ungeschälte Knoblauchzehen; Menge ganz nach Geschmack.

Zum Schluss nochmal wenig Öl über die Kartoffeln träufeln und in den Ofen schieben. Bei 180 Grad Umluft ca. 30 Minuten im Ofen backen. Als Garprobe vorsichtig mit einer Gabel in die Kartoffel stechen, es sollte sich nicht mehr hart anfühlen. Nötigenfalls die Backzeit etwas erhöhen.

Man kann auch gekochte Kartoffeln vom Vortag nehmen. Dann verringert sich die Backzeit auf etwa 20 Minuten.

Mit einem leckeren Dip werden zwei Leute von den Rosmarinkartoffeln satt. Mit entsprechenden Beilagen wie Salat oder Gemüse reicht es für mehr Esser.


{\bfseries Menge:} 4 Portionen

{\bfseries Quelle:} www.veggiekochbuch.de 

} 

%----------------------
\nopagebreak{ 
\subsection{Rote Bete-Gratin}

\index{P4@{P4\/}!Rote Bete-Gratin@{Rote Bete-Gratin\/}}
\index{Rote@{Rote\/}!Rote Bete-Gratin@{Rote Bete-Gratin\/}}
\index{Bete@{Bete\/}!Rote Bete-Gratin@{Rote Bete-Gratin\/}}
\index{Vegetarisch@{Vegetarisch\/}!Rote Bete-Gratin@{Rote Bete-Gratin\/}}
\index{Vorbereiten@{Vorbereiten\/}!Rote Bete-Gratin@{Rote Bete-Gratin\/}}
}{
\begin{multicols}{2}



800 g gekochte Rote Bete (möglichst frisch gekauft und selbst gekocht)

4 EL Olivenöl

Zucker

Salz

Pfeffer

6 EL Semmelbrösel

2 TL grober Senf

2 EL gemahlene Haselnüsse

2 TL gehackter Thymian

\bild{image/cookbook_5.jpg}


\end{multicols}

Rote Bete in Spalten schneiden, in einer Schüssel mit 2 EL Öl, etwas Salz, Zucker und Pfeffer mischen und in eine ofenfeste Form geben.

Semmelbrösel in einer Schüssel mit Senf, Haselnüssen und Thymian mischen.
Restliches Öl zugeben.

Brösel über die Bete streuen. Im heißen Ofen bei 200$^\circ$C goldbraun ca. 12 Minuten überbacken.


{\bfseries Menge:} 4 Portionen

{\bfseries Quelle:} www.vox.de:Schmeckt nicht, gibt's nicht 2006-09-15 

} 

%----------------------
\nopagebreak{ 
\subsection{Scharfes Möhrengemüse}

\index{Möhren@{Möhren\/}!Scharfes Möhrengemüse@{Scharfes Möhrengemüse\/}}
\index{Paprika@{Paprika\/}!Scharfes Möhrengemüse@{Scharfes Möhrengemüse\/}}
\index{vegetarisch@{vegetarisch\/}!Scharfes Möhrengemüse@{Scharfes Möhrengemüse\/}}
}{
\begin{multicols}{2}



500 g Möhren

1 Bund Frühlingszwiebeln

(ersatzweise Porree)

2 Peperoni

1 TL Apfeldicksaft

2 EL Butter

1 Handvoll Sojasprossen

1 Bund Kerbel oder Petersilie


\end{multicols}

Die Möhren putzen, waschen, in Scheiben schneiden, in wenig Salzwasser knapp garen. Die Frühlingszwiebeln putzen, in Ringe schneiden und zu den Möhren geben. Peperoni waschen, putzen und in winzig kleine Würfel schneiden, mit Salz, dem Apfeldicksaft und der Butter unter die Möhren geben. Die Sojasprossen abspülen und mit den gewürzten Möhren 5 Minuten auf kleiner Flamme dünsten, mit gezupften Kerbelblättchen bestreuen.

Passt hervorragend zu Rotbarsch in Sesamkruste.


{\bfseries Menge:} 4 Portionen

{\bfseries Quelle:} Internet 

} 

%----------------------
\nopagebreak{ 
\subsection{Senf-Kartoffelpüree}

\index{Kartoffeln@{Kartoffeln\/}!Senf-Kartoffelpüree@{Senf-Kartoffelpüree\/}}
\index{vegetarisch@{vegetarisch\/}!Senf-Kartoffelpüree@{Senf-Kartoffelpüree\/}}
\index{einfach@{einfach\/}!Senf-Kartoffelpüree@{Senf-Kartoffelpüree\/}}
}{
\begin{multicols}{2}



500 g Kartoffeln

125 ml Milch

1 EL Butter

150 g Mandeln, gehackt

1 EL Zitronensaft

Salz

Pfeffer

3 EL Senf, grob


\end{multicols}

Kartoffeln waschen, schälen und in kochendem Salzwasser 20 min. garen. Abgießen, mit einem Kartoffelstampder fein zerdrücken.
Milch und Butter kurz aufkochen und untermischen.

Gehackte Mandeln und Zitronensaft untermischen. Mit Salz und Pfeffer und grobem Senf würzen


{\bfseries Menge:} 2 Portionen

{\bfseries Inhalt pro Portion:} 2947~kJ, 22~g Eiweiß, 47~g Fett, 44~g Kohlenhydrate, 704~kcal, 3.72~BE

{\bfseries Zeit:} Gesamtzeit 35 min, Kochzeit 20 min, Vorbereitungszeit: 15 min

{\bfseries Quelle:} essen und trinken für jeden Tag 12/08 

} 

%----------------------
\nopagebreak{ 
\subsection{Senfsoßen-Kartoffeln}

\index{Kartoffeln@{Kartoffeln\/}!Senfsoßen-Kartoffeln@{Senfsoßen-Kartoffeln\/}}
\index{Senf@{Senf\/}!Senfsoßen-Kartoffeln@{Senfsoßen-Kartoffeln\/}}
\index{vegetarisch@{vegetarisch\/}!Senfsoßen-Kartoffeln@{Senfsoßen-Kartoffeln\/}}
}{
\begin{multicols}{2}



500 g kleine Kartoffeln

100 ml Gemüsefond

100 ml Schlagsahne

3 TL scharfer Senf

2 TL Senfkörner

Salz


\end{multicols}

Die Kartoffeln mit Wasser bedeckt aufkochen und 10 bis 15 Minuten garen.
Dann abgießen und pellen.
Gemüsefond und Sahne aufkochen und offen 3 Minuten einkochen lassen.
Senf und Senfkörner einrühren. Mit Salz abschmecken.
Die gepellten Kartoffeln in der Soße erwärmen und servieren.


{\bfseries Menge:} 4 Portionen

{\bfseries Quelle:} www.vox.de: Schmeckt nicht, gibt's nicht 

} 

%----------------------
\nopagebreak{ 
\subsection{Servietten-Olivenknödel}

\index{Brot@{Brot\/}!Servietten-Olivenknödel@{Servietten-Olivenknödel\/}}
\index{Oliven@{Oliven\/}!Servietten-Olivenknödel@{Servietten-Olivenknödel\/}}
\index{P6@{P6\/}!Servietten-Olivenknödel@{Servietten-Olivenknödel\/}}
\index{vegetarisch@{vegetarisch\/}!Servietten-Olivenknödel@{Servietten-Olivenknödel\/}}
}{
\begin{multicols}{2}



1/2 Kastenweißbrot oder

2 Laugenbrezel oder

anderes Weißbrot nach Wahl

100 ml Milch

2 Eier

1 kleine Zwiebel (gewürfelt)

1 Bund Petersilie gehackt

20 grüne Oliven ohne Stein

Olivenöl

Salz, Pfeffer

etwas Öl


\end{multicols}

Das Brot in kleine Würfel schneiden. Zwiebelwürfel in Öl anschwitzen, Salz, Pfeffer und Milch dazugeben, kurz aufkochen lassen und mit den Brotwürfeln vermischen. Eier verrühren und ebenfalls über die Masse geben.
Petersilie und klein gehackte Oliven dazu. 5 Minuten ziehen lassen. Alufolie ausbreiten. In die Mitte die Brötchenmasse geben und fest zusammenrollen; luftdicht verschließen und in Salzwasser schwimmend ca. 10 Minuten leicht kochen lassen. Die Masse aus der Folie nehmen und in 0,5 cm dicke Scheiben schneiden. Diese mit Butter in der Pfanne kurz anbraten.


{\bfseries Menge:} 6 Portionen

{\bfseries Quelle:} www.vox.de: Schmeckt nicht, gibt's nicht 

} 

%----------------------
\nopagebreak{ 
\subsection{Waldpilze mit Majoran}

\index{Pilze@{Pilze\/}!Waldpilze mit Majoran@{Waldpilze mit Majoran\/}}
\index{vegetarisch@{vegetarisch\/}!Waldpilze mit Majoran@{Waldpilze mit Majoran\/}}
}{
\begin{multicols}{2}



400 g gemischte Pilze (Champignons, Pfifferlinge, Kräuterseitlinge)

2 Knoblauchzehen

2 Stiele Majoran

Salz

4 EL Öl

150 ml Geflügelfond oder Brühe

Pfeffer

1 rote Zwiebel


\end{multicols}

Die Pilze putzen und in kleinere Stücke schneiden. Die Zwiebel pellen und fein würfeln, Knoblauch hacken. Majoran zupfen und hacken.
Inzwischen Öl in die Pfanne geben und die Schalotten, Knoblauch und Pilze darin unter Rühren bei mittlerer Hitze 5 bis 6 Minuten sanft dünsten. Die Brühe angießen und kurz aufkochen. Alles mit Salz und Pfeffer würzen und den Majoran untermischen.


{\bfseries Menge:} 4 Portionen

{\bfseries Quelle:} www.vox.de: Schmeckt nicht, gibt's nicht 

} 

%----------------------
\nopagebreak{ 
\subsection{Zerdrückte Pellkartoffeln}

\index{Kartoffeln@{Kartoffeln\/}!Zerdrückte Pellkartoffeln@{Zerdrückte Pellkartoffeln\/}}
}{
\begin{multicols}{2}



750 g kleine Kartoffeln

Meersalz

1 TL Kümmel

4 Zweige Majoran

3 EL weiche Butter

schwarzer Pfeffer


\end{multicols}

Die Kartoffeln gründlich waschen. In Salzwasser zugedeckt 20 Minuten kochen.

Den Kümmel im Mörser mahlen und mit dem klein geschnittenen Majoran und der weichen Butter vermengen. Mit Salz und Pfeffer abschmecken. Kartoffeln abgießen, auf dem Herd ausdämpfen lassen und grob zerstampfen, Mit der Butter vermengen und mit Meersalz bestreuen.

Passt gut zum Schweinefilet mit roten Zwiebeln.


{\bfseries Menge:} 4 Portionen

} 

\pagebreak 
 
%----------------------------------------------------
\nopagebreak{ 
\section{DRINK} 

\subsection{Aufgesetzter}

\index{Alkohol@{Alkohol\/}!Aufgesetzter@{Aufgesetzter\/}}
\index{Eingelegt@{Eingelegt\/}!Aufgesetzter@{Aufgesetzter\/}}
\index{vegetarisch@{vegetarisch\/}!Aufgesetzter@{Aufgesetzter\/}}
}{
\begin{multicols}{2}



Früchte

Teekandis

neutraler Schnaps (mindestens 40\% Alkohol)


\end{multicols}

Früchte waschen, putzen und kleinschneiden. Mit Teekandis im Verhältnis 3:1 bis 2:1 mischen.
In Schraubgläser füllen, mit Schnaps auffüllen und zuschrauben.
An einem hellen Ort ca. 4. Wochen stehenlassen. Dann in Flaschen abfiltern, z.B. mit Teefilterbeuteln.

Für Mirabellen und Pflaumen passt als Schnaps sehr gut Zwetschenwasser.


{\bfseries Quelle:} Frauke mit Anregungen aus dem Internet 

} 

%----------------------
\nopagebreak{ 
\subsection{Ayran 1}

\index{Getränke@{Getränke\/}!Ayran 1@{Ayran 1\/}}
\index{Joghurt@{Joghurt\/}!Ayran 1@{Ayran 1\/}}
\index{Türkei@{Türkei\/}!Ayran 1@{Ayran 1\/}}
}{
\begin{multicols}{2}



150 ml Naturjoghurt

150 ml Wasser (bis zu)

Salz

etwas saure Sahne

Zitronensaft


\end{multicols}

Ayran ist ein mit Wasser verdünnter und gesalzener Jogurt. Er soll bei Magenbeschwerden positiv wirken und den Flüssigkeits- und Salzhaushalt ausgleichen. Ayran ist ein typischer türkischer Sommerdrink. Er passt gut zu würzigen türkischen Spezialitäten, besonders zu Kebap. Dass Ayran den Blutdruck erhöht, ist eine weitere Eigenschaft des Getränks. Sie kann bei Hitze eine große Hilfe sein. Nicht zuletzt deshalb gilt Ayran als absoluter Durstlöscher.

Alle Zutaten müssen gekühlt sein.  Den Joghurt mit der Sahne, etwas Salz (nach Geschmack) und ein paar Tropfen Zitronensaft in eine Schüssel geben. Das Ganze mit einem Mixer und unter Zugabe von Wasser schaumig schlagen. Zwischendurch immer wieder abschmecken und nach Geschmack Zitronensaft, Wasser oder Salz zugeben. Sobald die gewünschte Schaumigkeit erreicht ist, den Ayran auf Gläser verteilen.
 Dekoration: Eine Zitronenscheibe ans Glas hängen. Zwei
getrocknete Pfefferminzblätter klein schneiden und auf das Getränk streuen.

Alternative Zubereitung: Wenn man neben den genannten Zutaten noch
Gurken nimmt, sie dünn schneidet, zum Jogurt gibt  und insgesamt weniger Wasser verwendet, entsteht kein Ayran, sondern die türkische Sauce *Gaci''. Sie eignet sich als Dipp.


{\bfseries Menge:} 1 Portionen

{\bfseries Quelle:} daheimunterwegs 2005-07-05 (www.kochfreunde.de) 

} 

%----------------------
\nopagebreak{ 
\subsection{Ayran 2}

\index{Getränke@{Getränke\/}!Ayran 2@{Ayran 2\/}}
\index{Joghurt@{Joghurt\/}!Ayran 2@{Ayran 2\/}}
\index{Kalt@{Kalt\/}!Ayran 2@{Ayran 2\/}}
\index{P4@{P4\/}!Ayran 2@{Ayran 2\/}}
}{
\begin{multicols}{2}



500 g Vollmilchjoghurt

200 g Saure Sahne

500 ml Wasser

Salz

Zitronensaft


\end{multicols}

Alle Zutaten gekühlt:   Joghurt, Saure Sahne, etwas
Salz nach Geschmack und ein paar Tropfen Zitronensaft zum Säuern des Ayran in eine Schüssel geben und möglichst mit einem Mixer unter Zugabe des Wassers schaumig rühren (schlagen mit Schneebesen geht auch) Das Wasser bis zur Maximalmenge zugeben. Lieber aber weniger wie mehr, je nach gewünschter Konsistenz des Ayrans, zwischendurch abschmecken und je nach Geschmack mehr Salz, Zitrone oder Wasser zugeben.
Nachdem der Ayran schön schaumig ist in Gläser verteilen und darauf achten das alle etwas vom Schaum abbekommen. Sehr erfrischend an heißen Tagen.


{\bfseries Menge:} 4 Portionen

{\bfseries Quelle:} www.kochfreunde.de 

} 

\pagebreak 
 
%----------------------------------------------------
\nopagebreak{ 
\section{PASTA} 

\subsection{Bärlauch--Pesto}

\index{Bärlauch@{Bärlauch\/}!Bärlauch--Pesto@{Bärlauch--Pesto\/}}
\index{Pasta@{Pasta\/}!Bärlauch--Pesto@{Bärlauch--Pesto\/}}
\index{vegetarisch@{vegetarisch\/}!Bärlauch--Pesto@{Bärlauch--Pesto\/}}
}{
\begin{multicols}{2}



1 Bund Bärlauch (ca. 80g)

70 ml Olivenöl

1 TL Salz

etwas weißer Pfeffer

50 g ungeschälte Mandeln oder Pinienkerne

20 g geriebener Pecorino


\end{multicols}

Bärlauch waschen, trockentupfen und grob hacken, mit Olivenöl, Salz und Pfeffer im Mixer pürieren. Mandeln oder Pinienkerne fein hacken und mit dem geriebenen Pecorino unterheben. Pesto in ein sauberes Marmeladenglas füllen, mit Olivenöl bedecken und verschlossen im Kühlschrank aufbewahren. Hält etwa 14 Tage.


{\bfseries Menge:} 4 Portionen

{\bfseries Quelle:} Bio Eck, Hannover, Alte Döhrener Straße 

} 

%----------------------
\nopagebreak{ 
\subsection{Cannelloni di carne (Cannelloni mit Fleischfüllung)}

\index{Hackfleisch@{Hackfleisch\/}!Cannelloni di carne (Cannelloni mit Fleischfüllung)@{Cannelloni di carne (Cannelloni mit Fleischfüllung)\/}}
\index{italienisch@{italienisch\/}!Cannelloni di carne (Cannelloni mit Fleischfüllung)@{Cannelloni di carne (Cannelloni mit Fleischfüllung)\/}}
\index{Pasta@{Pasta\/}!Cannelloni di carne (Cannelloni mit Fleischfüllung)@{Cannelloni di carne (Cannelloni mit Fleischfüllung)\/}}
\index{Rindfleisch@{Rindfleisch\/}!Cannelloni di carne (Cannelloni mit Fleischfüllung)@{Cannelloni di carne (Cannelloni mit Fleischfüllung)\/}}
\index{vorbereiten@{vorbereiten\/}!Cannelloni di carne (Cannelloni mit Fleischfüllung)@{Cannelloni di carne (Cannelloni mit Fleischfüllung)\/}}
}{
\begin{multicols}{2}



Pastateig aus 2 Eiern oder

250 g Fertig-Cannelloni

60 g Butter

3 EL fein gewürfelte Zwiebeln

350 g mageres Rinderhackfleisch

Salz

frisch gemahlener schwarzer Pfeffer

250 g geschälte Dosentomaten mit Saft, grob geschnitten

100 g sehr fein gehackte Mortadella

1 Eigelb

1/8 TL frisch geriebene Muskatnuss

275 g Ricotta

150 g frisch geriebener Parmesan

1 Rezeptmenge Béchamelsauce


\end{multicols}

Je 30g Butter in 2 Pfannen geben und über mittlerer Hitze schmelzen lassen. Je eine Hälfte der Zwiebelwürfel hineingeben und glasig dünsten. Je eine Hälfte Rinderhackfleisch in die Pfanne geben und anbraten. Dabei das Hackfleisch mit einem Holzlöffel zerzupfen und mit Salz und Pfeffer würzen.

Die Tomaten in eine der Pfannen schütten und mit dem Fleisch verrühren. Zum Kochen bringen und auf kleiner Flamme 35-45 Minuten kochen lassen, bis das Fett sich absetzt. Dies ist die Fleischsauce.

Für die Fleischfüllung das angebratene Hackfleisch aus der zweiten Pfanne in eine Schüssel schütten und abkühlen lassen. Danach mit der Mortadella, dem Eigelb, der geriebenen Muskatnuss, dem Ricotta und 125g geriebenem Parmesan mit einer Gabel vermischen.

(Sauce und Füllung können bereits am Vortag zubereitet und im Kühlschrank aufbewahrt werden.)

Eine Béchamelsauce zubereiten, jedoch nicht so dick einkochen lassen.

Den Pastateig so dünn wie möglich ausrollen und in 7.5 x 10 cm große Rechtecke schneiden. In einem großen Topf 4l Wasser zum Kochen bringen. Eine große Schüssel mit leicht gesalzenem kalten Wasser und einigen Eiswürfeln daneben stellen. Auf der Arbeitsfläche mehrere saubere Küchentücher ausbreiten. Wenn das Wasser kocht, 1 EL Salz hinein schütten und nur so viele Pastarechtecke hinein legen, dass sie bequem im Topf schwimmen können. Nur etwa 30 Sekunden garen, mit einem Sieblöffel heraus heben und im geeisten Wasser schwenken. Auf den Küchentüchern ausbreiten und trockentupfen.

Etwa 6 EL Béchamelsauce mit der Fleischfüllung verrühren. Damit die Rechtecke dünn bestreichen und dabei ringsrum einen Rand von 1/2 cm frei lassen. Die Cannelloni aufrollen.
Den Boden einer passenden ofenfesten Form mit etwas Béchamelsauce ausstreichen. Die Cannelloni in einer Schicht in die Form legen (falls nötig 2 Formen nehmen). Die Cannelloni mit der Fleischsauce und der restlichen Béchamelsauce überziehen. Mit dem restlichen Käse bestreuen.

Im 200 Grad heißen Ofen etwa 15-20 Minuten überbacken, bis sich eine hellbraune Kruste gebildet hat. Aus dem Ofen nehmen und vor dem Servieren 10 Minuten ruhen lassen.


{\bfseries Menge:} 4 Portionen

{\bfseries Quelle:} Giuliano Hazan: Klassische Pastaküche 

} 

%----------------------
\nopagebreak{ 
\subsection{Canneloni mit Pilzfüllung}

\index{Pasta@{Pasta\/}!Canneloni mit Pilzfüllung@{Canneloni mit Pilzfüllung\/}}
\index{Pilze@{Pilze\/}!Canneloni mit Pilzfüllung@{Canneloni mit Pilzfüllung\/}}
\index{Spinat@{Spinat\/}!Canneloni mit Pilzfüllung@{Canneloni mit Pilzfüllung\/}}
\index{vegetarisch@{vegetarisch\/}!Canneloni mit Pilzfüllung@{Canneloni mit Pilzfüllung\/}}
\index{vorbereiten@{vorbereiten\/}!Canneloni mit Pilzfüllung@{Canneloni mit Pilzfüllung\/}}
}{
\begin{multicols}{2}

\textit{Teig}



70 g Hartweizengrieß

70 g Weizenmehl

1 Ei

1 Eigelb

Salz

\textit{Füllung}



80 g Zwiebeln

1 Knoblauchzehe

300 g gemischte Pilze

100 g geputzter Spinat

30 g Butter

40 g feingewürfelter Lauch

100 ml Sahne

1/2 TL Salz

frisch gemahlener weißer Pfeffer

1 Stangen Lauch

\textit{Sauce}



250 ml Sahne

1 Eigelb

Salz

frisch gemahlener weißer Pfeffer

1 EL frisch gehackte Petersilie

\textit{Außerdem}



4 feuerfeste Förmchen

Butter für die Förmchen


\end{multicols}

Für den Nudelteig Grieß und Mehl gut vermischen und mit den restlichen Zutaten zu einem glatten Teig verkneten. In Klarsichtfolie wickeln und eine Stunde im Kühlschrank ruhen lassen.

Zwiebeln und Knoblauch schälen und fein hacken. Die Pilze sorgfältig putzen und in kleine Würfel schneiden. Spinat waschen, gut abtropfen lassen und fein hacken.

Die Butter in der Pfanne zerlassen. Zwiebeln und Knoblauch darin glasig anschwitzen. Die Pilze 3 Minuten mitbraten, anschließend den Spinat und den gewürfelten Lauch zufügen und durchschwenken. Sahne angießen, salzen und pfeffern und 3 bis 4 Minuten einkochen. Die Mischung vom Herd nehmen und auskühlen lassen.

Die Lauchstangen putzen und vom unteren hellen Teil ein etwa 12 cm langes Stück abschneiden. Von diesem vorsichtig 8 Blätter ablösen und diese in siedendem Salzwasser 4 Minuten kochen. Herausnehmen, kalt abschrecken und gut abtropfen lassen.

Den Teig auf einer bemehlten Arbeitsfläche dünn ausrollen (am besten durch die Nudelmaschine drehen) und 8 Rechtecke von 9 x 14 cm ausschneiden. Jedes Teigstück mit einem Lauchblatt belegen, die Pilzfüllung darauf verteilen und zu Canneloni aufrollen.

Für die Sauce die Sahne in einen Topf geben und so lange köcheln, bis die Flüssigkeit um 1/3 reduziert ist. In einer kleinen Schüssel das Eigelb verquirlen und 1 EL heiße Sahne unterrühren. Die Mischung mit der heißen Sahne vermengen, dabei darf die Sauce nicht mehr kochen. Salzen und pfeffern und die gehackte Petersilie einstreuen.

Je zwei Canneloni in die gebutterten feuerfesten Förmchen legen und mit der Sauce übergießen. Bei 20$^\circ$C im vorgeheizten Ofen ca. 20 bis 25 Minuten backen, die letzte Minute unter den Grill stellen, damit die Oberfläcje schön bräunt.


{\bfseries Menge:} 4 Portionen

{\bfseries Quelle:} Christian Teubner: Die 100 besten Rezepte aus aller Welt: Vegetarisch 

} 

%----------------------
\nopagebreak{ 
\subsection{Fettuccine all'Alfredo (Fettuccine mit Alfredos Käse-Sahne-Sauce)}

\index{italienisch@{italienisch\/}!Fettuccine all'Alfredo (Fettuccine mit Alfredos Käse-Sahne-Sauce)@{Fettuccine all'Alfredo (Fettuccine mit Alfredos Käse-Sahne-Sauce)\/}}
\index{kalorienreich@{kalorienreich\/}!Fettuccine all'Alfredo (Fettuccine mit Alfredos Käse-Sahne-Sauce)@{Fettuccine all'Alfredo (Fettuccine mit Alfredos Käse-Sahne-Sauce)\/}}
\index{Pasta@{Pasta\/}!Fettuccine all'Alfredo (Fettuccine mit Alfredos Käse-Sahne-Sauce)@{Fettuccine all'Alfredo (Fettuccine mit Alfredos Käse-Sahne-Sauce)\/}}
\index{vegetarisch@{vegetarisch\/}!Fettuccine all'Alfredo (Fettuccine mit Alfredos Käse-Sahne-Sauce)@{Fettuccine all'Alfredo (Fettuccine mit Alfredos Käse-Sahne-Sauce)\/}}
}{
\begin{multicols}{2}



für Pasta aus 3 Eiern oder

500 g Fertignudeln (z.B. Fettuccine, Tagliatelle)

45 g Butter

250 ml Sahne

etwas frisch geriebene Muskatnuss

Salz

frisch gemahlener schwarzer Pfeffer

60 g frisch geriebener Parmesan


\end{multicols}

Die Butter und die Sahne in einer großen Sauteuse auf mittlere Flamme setzen. Unter ständigem Rühren die Sahne auf etwa die Hälfte eibkochen lassen. Mit Muskatnuss, reichlich schwarzem Pfeffer und etwas Salz würzen und vom Feuer nehmen.

Inzwischen die Pasta mit 1 EL Salz in 4l kochendes Wasser schütten, gut umrühren und al dente kochen. Danach abtropfen lassen und in die Sauteuse schütten. Die Pasta mit Käse bestreuen und in der Sauteuse gründlich wenden. Abschmecken und sofort servieren.


{\bfseries Menge:} 4 Portionen

{\bfseries Quelle:} Giuliano Hazan: Klassische Pastaküche 

} 

%----------------------
\nopagebreak{ 
\subsection{Kartoffelgnocchi}

\index{Kartoffeln@{Kartoffeln\/}!Kartoffelgnocchi@{Kartoffelgnocchi\/}}
\index{vegetarisch@{vegetarisch\/}!Kartoffelgnocchi@{Kartoffelgnocchi\/}}
}{
\begin{multicols}{2}



500 g gekochte, geschälte Kartoffeln (z.B. vom Vortag)

150 g Mehl

2 Eier

Salz und Pfeffer

Muskatnuss

\bild{image/cookbook_6.jpg}


\end{multicols}

Die gekochten Kartoffeln auf der Kartoffelreibe reiben oder alternativ durch Kartoffelpresse pressen und mit Mehl und Eiern vermengen bis ein geschmeidiger, fast trockener Teig entsteht. Bei Bedarf noch Mehl dazu geben. Mit Salz, Pfeffer und Muskatnuß abschmecken. Aus dem Teig Rollen mit einem Durchmesser von ca.1,5 cm formen. Die Rollen in 1,5 cm breite Stücke schneiden. die Stücke mit dem Handballen zu kleinen Kugeln drehen oder über die Gabel ziehen. Die Kugeln in kochendes Wasser geben und kochen bis sie an der Oberfläche schwimmen. Abtropfen lassen.


{\bfseries Menge:} 4 Portionen

{\bfseries Quelle:} www.vox.de: Schmeckt nicht, gibt's nicht 

} 

%----------------------
\nopagebreak{ 
\subsection{Kürbisgnocchi}

\index{Käse@{Käse\/}!Kürbisgnocchi@{Kürbisgnocchi\/}}
\index{Kürbis@{Kürbis\/}!Kürbisgnocchi@{Kürbisgnocchi\/}}
\index{Vegetarisch@{Vegetarisch\/}!Kürbisgnocchi@{Kürbisgnocchi\/}}
}{
\begin{multicols}{2}



1.2 kg Hokkaido-Kürbis (geputzt gewogen)

Salz

200 g Mehl

2 Eigelbe

120 g Parmesan, frisch gerieben

Pfeffer

2 Prisen frisch geriebene Muskatnuss

\bild{image/cookbook_7.jpg}


\end{multicols}

Den Backofen auf 20$^\circ$ vorheizen. Den Kürbis gründlich waschen, in sechs Spalten schneiden und entkernen. Auf ein mit Backpapier ausge-
legtes Backblech legen und im heißen Backofen (Mitte, Umluft 180$^\circ$) ca. 35-40 Min. backen. Herausnehmen und fein pürieren.

In einem großen Topf reichlich Wasser zum Kochen bringen, salzen. Das warme Kürbismus mit dem Mehl, den Eigelben, dem Parmesan, Salz, Pfeffer und Muskat mit den Knethaken des Handmixers zu einem glatten Teig verkneten.

Aus dem Teig mit zwei Teelöffeln Gnocchi ausstechen und formen. Ins siedende Wasser gleiten lassen und bei schwacher Hitze ca. 10 Min. ziehen lassen, bis die Gnocchi an die Oberfläche steigen. Mit einem Schaumlöffel herausheben und heiß servieren.

Tip: Bis alle Gnocchi fertig sind, dauert es eine Weile, darum bereits fertige Gnocchi
in eine flache feuerfeste Form geben, mit Butterflöckchen bestreuen und bei 120$^\circ$ im Backofen (Mitte, Umluft 100$^\circ$ warm halten.

Zubereitungszeit: ca. 30 Min.
Backzeit: ca. 40 Min.


{\bfseries Bemerkung:} kcal: 395 

{\bfseries Menge:} 4 Portionen

{\bfseries Quelle:} Andrea Oppermann: Kürbisglück 

} 

%----------------------
\nopagebreak{ 
\subsection{Lasagne alla Bolognese (Lasagne mit Bologneser Sauce)}

\index{Hackfleisch@{Hackfleisch\/}!Lasagne alla Bolognese (Lasagne mit Bologneser Sauce)@{Lasagne alla Bolognese (Lasagne mit Bologneser Sauce)\/}}
\index{italienisch@{italienisch\/}!Lasagne alla Bolognese (Lasagne mit Bologneser Sauce)@{Lasagne alla Bolognese (Lasagne mit Bologneser Sauce)\/}}
\index{Pasta@{Pasta\/}!Lasagne alla Bolognese (Lasagne mit Bologneser Sauce)@{Lasagne alla Bolognese (Lasagne mit Bologneser Sauce)\/}}
\index{Rindfleisch@{Rindfleisch\/}!Lasagne alla Bolognese (Lasagne mit Bologneser Sauce)@{Lasagne alla Bolognese (Lasagne mit Bologneser Sauce)\/}}
\index{vorbereiten@{vorbereiten\/}!Lasagne alla Bolognese (Lasagne mit Bologneser Sauce)@{Lasagne alla Bolognese (Lasagne mit Bologneser Sauce)\/}}
}{
\begin{multicols}{2}



Pastateig von 2 Eiern oder

250 g fertige Lasagneblätter

1 Rezeptmenge Bologneser Sauce

1 1/2 Rezeptmenge Béchamelsauce

Salz

90 g frisch geriebener Parmesan

30 g Butter


\end{multicols}

Die Bologneser Sauce in eine Schüssel geben. Eine Béchamelsauce zubereiten. Den Boden einer Backform mit etwas Béchamelsauce ausgießen und mit Pastarechtecken belegen. Die restliche Béchamelsauce mit der Bologneser Sauce in der Schüssel verrühren. Eine Schicht dieser Saucenmischung auf die Pastaschicht in der Form streichen. Weitere Pastablätter darüberlegen. Fortfahren, bis mindestens 5 Schichten in der Form sind. Die Deckschicht ebenfalls dünn mit Sauce bestreichen. Mit dem Käse bestreuen und mit Butterflöckchen belegen.

So weit kann das Gericht im Voraus zubereitet und bis zu zwei Tagen im Kühlschrank aufbewahrt werden.

Die Form im 200 Grad heißen Ofen 15-20 Minuten backen, bis sich eine hellbraune Kruste gebildet hat. Vor dem Servieren 10 Minuten ruhen lassen.


{\bfseries Menge:} 4 Portionen

{\bfseries Quelle:} Giuliano Hazan: Klassische Pastaküche 

} 

%----------------------
\nopagebreak{ 
\subsection{Linsenbolognese}

\index{Linsen@{Linsen\/}!Linsenbolognese@{Linsenbolognese\/}}
\index{Pasta@{Pasta\/}!Linsenbolognese@{Linsenbolognese\/}}
\index{vegetarisch@{vegetarisch\/}!Linsenbolognese@{Linsenbolognese\/}}
}{
\begin{multicols}{2}



1 Bund Suppengrün (ca. 400 g)

200 g Zwiebeln

4 EL Olivenöl

Salz

Pfeffer

Zucker

5 EL Paprikamark (z.B. Ajvar)

1 EL Tomatenmark

1 TL getrockneter Oregano

100 ml Rotwein

125 g Beluga-Linsen

500 ml Gemüsebrühe

400 g Bandnudeln

30 g gehobelter Parmesan

1 EL gehackte Petersilie

\bild{image/cookbook_8.jpg}


\end{multicols}

Suppengrün putzen, waschen und ca. 1/2 cm groß würfeln. Zwiebeln fein würfeln. Alles in heißem Olivenöl andünsten. Kräftig mit Salz, Pfeffer und etwas Zucker würzen. Paprika- und Tomatenmark und Oregano kurz mitrösten. Mit Rotwein und Brühe aufgießen, Linsen zugeben. Aufkochen, zugedeckt bei mittlerer Hitze 20 bis 25 Minuten garen. Dabei ab und zu umrühren.

Bandnudeln in kochendem Salzwasser nach Packungsanweisung garen. Soße eventuell nachwürzen, Nudeln abgießen, unter die Soße mischen und mit dem Parmesan und der Petersilie bestreut servieren.

Tipp: Statt der Beluga-Linsen können Sie auch rote Linsen verwenden, die zerkochen jedoch etwas stärker.


{\bfseries Menge:} 4 Portionen

{\bfseries Quelle:} www.vox.de: Schmeckt nicht, gibt's nicht 

} 

%----------------------
\nopagebreak{ 
\subsection{Pasta-Grundteig}

\index{Grundteig@{Grundteig\/}!Pasta-Grundteig@{Pasta-Grundteig\/}}
\index{italienisch@{italienisch\/}!Pasta-Grundteig@{Pasta-Grundteig\/}}
\index{Pasta@{Pasta\/}!Pasta-Grundteig@{Pasta-Grundteig\/}}
\index{vegetarisch@{vegetarisch\/}!Pasta-Grundteig@{Pasta-Grundteig\/}}
}{
\begin{multicols}{2}



3 Eier

300 g Weizenmehl


\end{multicols}

Das Mehl auf ein Holzbrett oder eine andere glatte, warme Arbeitsfläche häufen und mit den Fingern eine Mulde in die Mitte drücken. Die Eier nacheinander in die Mulde aufschlagen. Nur zimmerwarme Eier verwenden. Die Eier vorsichtig mit einer Gabel verrühren, bis Eigelb und Eiweiß vermischt sind. Mit der Gabel das Mehl vom Muldeninneren her in die Eimischung einarbeiten, bis diese nicht mehr flüssig ist. Dabei den äußeren Mehlrand nicht durchbrechen, sonst läuft das Ei aus. Mit beiden Händen die Eimischung schnell mit dem Mehl bedecken. Die Masse mit den Händen bearbeiten, bis alles Mehl mit den Eiern vermischt ist. Jetzt muss man entscheiden, ob noch mehr Mehl nötig ist: Der Teig soll sich feucht anfühlen, jedoch nicht kleben. Wenn er die richtige Kosistenz hat, wird er in Klarsichtfolie gewickelt.

Alle Teigreste von der Arbeitsfläche abschaben und die Hände waschen. Den Teig auswickeln und kneten. Dabei den Teig mit einer Hand halten und mit den Fingern der anderen hand zusamenfalten. Mit dem Handballen den Teig zusammendrücken und von sich stoßen. Danach den Teig um eine Vierteldrehung drehen. Das Ganze wiederholen, bis der Teig glatt und geschmeidig ist. Den Teig sofort in Klarsichtfolie wickeln und vor dem Ausrollen mindestens 20 Minuten ruhen lassen.

Einen Teig aus drei Eiern in mindestens 6 Stücke teilen. Ein Stück mit den Fingern glatt ziehen, den Rest in Folie wickeln. Den Teig zuerst bei der breitesten Walzeneinstellung durch die Maschine lassen. Den Teig beim Herauskommen halten, jedoch nicht herausziehen. Den Teig dreifach zusammenschlagen,  so drehen, dass die Falten an den Seiten sind und nochmals durch die Maschine treiben. Dies noch drei bis vier Mal wiederholen, bis der Teig ganz glatt ist. Mit den restlichen Teigstücken ebenso verfahren. Die Walzen eine Stufe enger stellen. Alle Teigstücke durchlaufen lassen und auf Küchentüchern ausbreiten. Die Walzen stufenweise bis zur letzten Einstellung enger stellen und jedes Mal alle Teigstücke durchlaufen lassen.

Den Teig stufenweise mit der Maschine auswalzen. Wenn man dabei auf der Einstellskala Stufen überspringt, bekommt die Pasta eine weniger gute Konsistenz. Pasta muss stufenweise ausgewalzt werden, damit sie elastisch wird. Um alle Teigstücke auf einmal auszurollen, benötigt man eine große Arbeitsfläche. Eventuell sollte man zuerst nur 3 Teigstücke ausrollen und  den Rest in Klarsichtfolie wickeln. Wenn der Teigstreifen beim Ausrollen zu lang wird, sollte man ihn entzwei schneiden.

Mit der Maschine kann man auch Fettuccine schneiden. Zum Aufbewahren um die Hnd zu lockeren Nestern wickeln und auf einem Küchentuch trocknen lassen (getrocknete Nudeln halten sich monatelang). Nudeln zum Sofortverbrauch ausgebreitet trocknen lassen.

In einem großen Topf viel Wasser zum Kochen bringen (1EL Salz auf 4l Wasser). Alle Nudeln auf einmal hineinschütten, sofort umrühren, damit sie nicht kleben. Die Pasta wiederholt umrühren und probieren, ob sie al dente und damit gar ist (frische Nudeln sind in weniger als 1 Minute fertig).

Die Pasta sofort zum Abtropfen in ein Sieb schütten. Leicht schütteln, damit das restlich Wasser abläuft. Niemals kalt abspülen -- das kühlt die Pasta ab und entfernt die Stärkeschicht, an der die Sauce haften bleibt. Die abgetropfte Pasta in eine vorgewärmte Schüssel geben und die Sauce zufügen oder die Pasta gleich in den Topf zu der Sauce schütten. Die Pasta mit Gabel und Löffel wenden, bis sie mit der Sauce überzogen ist. Nicht nur einfach mit der Sauce übergießen!


{\bfseries Menge:} 4 Portionen

{\bfseries Quelle:} Giuliano Hazan: Klassische Pastaküche 

} 

%----------------------
\nopagebreak{ 
\subsection{Pesto di basilico alla Genovese (Basilikumsauce auf Genuese}

\index{italienisch@{italienisch\/}!Pesto di basilico alla Genovese (Basilikumsauce auf Genuese@{Pesto di basilico alla Genovese (Basilikumsauce auf Genuese\/}}
\index{Pasta@{Pasta\/}!Pesto di basilico alla Genovese (Basilikumsauce auf Genuese@{Pesto di basilico alla Genovese (Basilikumsauce auf Genuese\/}}
\index{vegetarisch@{vegetarisch\/}!Pesto di basilico alla Genovese (Basilikumsauce auf Genuese@{Pesto di basilico alla Genovese (Basilikumsauce auf Genuese\/}}
}{
\begin{multicols}{2}



für Pasta aus drei Eiern oder

500 g Fertignudeln (z.B. Spaghetti, Fettuccine)

60 g frisches Basilikum

8 EL Olivenöl

2 EL Pinienkerne

2 Knoblauchzehen

60 g frisch geriebener Parmesan

2 EL frisch geriebener Pecorino

45 g weiche Butter


\end{multicols}

Die Bsilikumblätter, das Olivenöl, die Pinienkerne, den Knoblauch und 1 TL Salz in der Küchenmaschine zu einer cremigen Mischung verarbeiten.

Diese Mischung kann bereits vorbereitet werden und im Kühlschrank aufbewahrt oder eingefroren werden. Dabei die Oberfläche mit etwas Olivenöl bedecken, damit sich das Basilikum nicht verfärbt.

Die Mischung in eine große Schüssel geben und die beiden geriebenen Käsesorten einrühren.
Die Pasta und 1 EL Salz in 4 l kochendes Wasser schütten, gut umrühren und al dente kochen. Danach abtropfen lassen und mit der Sauce sowie 2 EL heißem Wasser und der weichen Butter vermischen.


{\bfseries Menge:} 4 Portionen

{\bfseries Quelle:} Giuliano Hazan: Klassische Pastaküche 

} 

%----------------------
\nopagebreak{ 
\subsection{Pilz-Puten-Sauce für Pasta}

\index{Geflügel@{Geflügel\/}!Pilz-Puten-Sauce für Pasta@{Pilz-Puten-Sauce für Pasta\/}}
\index{Plize@{Plize\/}!Pilz-Puten-Sauce für Pasta@{Pilz-Puten-Sauce für Pasta\/}}
\index{einfach@{einfach\/}!Pilz-Puten-Sauce für Pasta@{Pilz-Puten-Sauce für Pasta\/}}
}{
\begin{multicols}{2}



500 g Champignons

500 g Putenbrustfilet

1 Paprika

2 EL Öl

2 TL Paprikapaste

200 g Saure Sahne

2 TL Saucenbinder


\end{multicols}

Pilze und Paprika putzen, Pilze, Paprika und Fleisch in Öl anbraten. Paprikapaste und saure Sahne zufügen, mit etwas Sauchenbinder abbinden.

{\bfseries Menge:} 4 Personen

{\bfseries Quelle:} Gerald 

} 

%----------------------
\nopagebreak{ 
\subsection{Ravioli alla Potentina (Ravioli mit Ricotta, Prociutto und Pecorino)}

\index{italienisch@{italienisch\/}!Ravioli alla Potentina (Ravioli mit Ricotta, Prociutto und Pecorino)@{Ravioli alla Potentina (Ravioli mit Ricotta, Prociutto und Pecorino)\/}}
\index{Pasta@{Pasta\/}!Ravioli alla Potentina (Ravioli mit Ricotta, Prociutto und Pecorino)@{Ravioli alla Potentina (Ravioli mit Ricotta, Prociutto und Pecorino)\/}}
\index{vorbereiten@{vorbereiten\/}!Ravioli alla Potentina (Ravioli mit Ricotta, Prociutto und Pecorino)@{Ravioli alla Potentina (Ravioli mit Ricotta, Prociutto und Pecorino)\/}}
}{
\begin{multicols}{2}

\textit{Pasta und Sauce}



1 Pastateig von 3 Eiern

1 Eiweiß

Tonatensauce

geriebener Pecorino oder Parmesan

\textit{Füllung}



1 kg Ricotta

2 Eigelb

90 g frisch geriebener Pecorino

60 g dünn geschnittener roher Schinken, fein gehackt

1/2 TL Salz

1/4 TL frisch geriebener weißer Pfeffer

2 EL frische glatte Petersilie, fein gehackt


\end{multicols}

Für die Füllung den Ricotta 3-4 Stunden im Kühlschrank in einem Sieb über einer Schüssel abtropfen lassen, bis er fest ist. Gut mit den anderen Zutaten vermischen.

Die Pasta ausrollen, zwei Teigstreifen auf einmal verarbeiten; den Rest bis zur Verarbeitung zudecken. Die Streifen nebeneinander legen.
Die Füllung in kleinen Häufchen auf einen Streifen setzen (Teelöffel oder Spritzbeutel) und den Teig rundherum mit Eiweiß bestreichen.
Den zweiten Pstastreifen auf den ersten legen, andrücken und in Stücke schneiden.

Die Ravioli werden auf Tabletts ausgelegt, die mit bemehlten Küchentüchern bedeckt sind. Etwa 4 Stunden trocknen lassen, gelegentlich wenden.

In einem großen Topf Salzwasser zum Kochen bringen. Die Ravioli hineingeben, nach dem Aufkochen offen und unter gelgentlichem Umrühren 3-5 Minuten garen. In eine vorgewärmte Schüssel geben.
Die Sause über die Ravioli geben und falls gewünscht, geriebenen Pecorino oder Parmesan dazu reichen.


Abwandlung: Alternativ reicht die Füllung für 250 g (gekaufte) Canneloni, dann 20 Minuten bei 200 Grad in der Tomatensauce backen.


{\bfseries Menge:} 4 Portionen

{\bfseries Quelle:} Julia della Croce: Klassische italienische Küche 

} 

%----------------------
\nopagebreak{ 
\subsection{Ricotta-Spinat-Gnocchi mit Steinpilzrahm (Malfatti)}

\index{Pasta@{Pasta\/}!Ricotta-Spinat-Gnocchi mit Steinpilzrahm (Malfatti)@{Ricotta-Spinat-Gnocchi mit Steinpilzrahm (Malfatti)\/}}
\index{Spinat@{Spinat\/}!Ricotta-Spinat-Gnocchi mit Steinpilzrahm (Malfatti)@{Ricotta-Spinat-Gnocchi mit Steinpilzrahm (Malfatti)\/}}
\index{vegetarisch@{vegetarisch\/}!Ricotta-Spinat-Gnocchi mit Steinpilzrahm (Malfatti)@{Ricotta-Spinat-Gnocchi mit Steinpilzrahm (Malfatti)\/}}
}{
\begin{multicols}{2}



25 g getrocknete Steinpilze

250 g Ricotta

Salz, weißer Pfeffer

2 Eier

100 g Parmesan frisch gerieben

3 EL Mehl

3 EL Semmelbrösel

400 g Spinat

Muskatnuss

2 Schalotten

2 EL Öl

3 EL Portwein

250 ml Schlagsahne

1 EL Schnittlauchröllchen

\bild{image/cookbook_9.jpg}


\end{multicols}

Die Steinpilze in 125 ml warmem Wasser einweichen. Den Ricotta in eine Schüssel geben. Salzen und pfeffern, mit Ei, Parmesan, Mehl und Semmelbröseln mischen, mit Muskatnuss würzen und kurz quellen lassen  Inzwischen den Spinat putzen, gründlich waschen und in kochendem Salzwasser 1 Minute blanchieren. Abgießen, abschrecken und gut ausdrücken. Anschließend den Spinat hacken und zur Ricottamischung geben.

Die Masse zu kleinen Kugeln formen und in einen Topf kochendes Salzwasser geben. Hitze reduzieren und die Knödel darin knapp unter dem Siedepunkt etwa 12 Minuten ziehen lassen.

Inzwischen die Schalotten pellen und fein würfeln. In heißem Öl 1 Minute farblos anschwitzen. Die Steinpilze ausdrücken und klein hacken, dazugeben und eine weitere Minute anschwitzen. Das Einweichwasser durch ein feines Sieb und in einen Topf geben. Vollständig einkochen lassen und dann mit dem Portwein ablöschen.

Die Sahne angießen und cremig einkochen lassen, mit Salz und Pfeffer würzen und Schnittlauch dazugeben.

Die Knödel aus dem Wasser nehmen, abtropfen lassen und mit der Soße servieren.


{\bfseries Menge:} 4 Portionen

{\bfseries Quelle:} www.vox.de: Schmeckt nicht, gibt's nicht 

} 

%----------------------
\nopagebreak{ 
\subsection{Rosa Tomatensauce}

\index{italienisch@{italienisch\/}!Rosa Tomatensauce@{Rosa Tomatensauce\/}}
\index{Tomaten@{Tomaten\/}!Rosa Tomatensauce@{Rosa Tomatensauce\/}}
\index{vegetarisch@{vegetarisch\/}!Rosa Tomatensauce@{Rosa Tomatensauce\/}}
}{
\begin{multicols}{2}



für Pasta aus 2 Eiern (z.B. Tortelloni)

1/2 Rezeptmenge Sugo burro e pomodoro (Tomaten-Butter-Sauce)

8 EL Sahne

60 g frisch geriebener Parmesan


\end{multicols}

Die vorbereitete Tomaten-Butter-Sauce über mittlerer Hitze zum Kochen bringen. Die Sahne einrühren und 2-3 Minuten einkochen, bis die Sauce so dick ist, dass sie einen Kochlöffel überzieht.

Die heiße Sauce abschmecken und mit den fertigen Tortelloni und dem Käse mischen. Sofort servieren.


{\bfseries Menge:} 4 Portionen

{\bfseries Quelle:} Giuliano Hazan: Klassische Pastaküche 

} 

%----------------------
\nopagebreak{ 
\subsection{Sugo al burro e pomodoro (Tomaten-Butter-Sauce)}

\index{italienisch@{italienisch\/}!Sugo al burro e pomodoro (Tomaten-Butter-Sauce)@{Sugo al burro e pomodoro (Tomaten-Butter-Sauce)\/}}
\index{Pasta@{Pasta\/}!Sugo al burro e pomodoro (Tomaten-Butter-Sauce)@{Sugo al burro e pomodoro (Tomaten-Butter-Sauce)\/}}
\index{Tomaten@{Tomaten\/}!Sugo al burro e pomodoro (Tomaten-Butter-Sauce)@{Sugo al burro e pomodoro (Tomaten-Butter-Sauce)\/}}
\index{vegetarisch@{vegetarisch\/}!Sugo al burro e pomodoro (Tomaten-Butter-Sauce)@{Sugo al burro e pomodoro (Tomaten-Butter-Sauce)\/}}
}{
\begin{multicols}{2}



für Pasta aus 3 Eiern oder

500 g Fertignudeln (z.B. Spaghetti, Penne, Torteloni)

1 kg vollreife Tomaten, geschält, entkernt und grob geschnitten oder

2 Dosen (je 400g) geschälte Tomaten mit Saft

100 g Butter

1 mittelgroße Zwiebel, geschält und halbiert

Salz

4 EL geriebener Parmesan


\end{multicols}

Alle Saucenzutaten außer dem Käse, in eine Sauteuse geben und etwa 30 Minuten auf kleiner Flamme kochen, bis die Tomaten zerfallen sind. Vom Feuer nehmen und die Zwiebelhälften entfernen.

(So weit kann die Sauce vorbereitet und 3-4 Tage im Kühlschrank aufbewahrt oder eingefroren werden.)

Die Pasta mit 1 EL Salz in 4l kochendes Wasser schütten, gut umrühren und al dente kochen. Danach abtropfen lassen und in der wieder erhitzten Sauce wenden, dabei den frisch geriebenen Käse zufügen.


{\bfseries Menge:} 4 Portionen

{\bfseries Quelle:} Giuliano Hazan: Klassische Pastaküche 

} 

%----------------------
\nopagebreak{ 
\subsection{Tagliatelle al ragù (Tagliatelle mit Bologneser Sauce)}

\index{Hackfleisch@{Hackfleisch\/}!Tagliatelle al ragù (Tagliatelle mit Bologneser Sauce)@{Tagliatelle al ragù (Tagliatelle mit Bologneser Sauce)\/}}
\index{italienisch@{italienisch\/}!Tagliatelle al ragù (Tagliatelle mit Bologneser Sauce)@{Tagliatelle al ragù (Tagliatelle mit Bologneser Sauce)\/}}
\index{Pasta@{Pasta\/}!Tagliatelle al ragù (Tagliatelle mit Bologneser Sauce)@{Tagliatelle al ragù (Tagliatelle mit Bologneser Sauce)\/}}
\index{Tomaten@{Tomaten\/}!Tagliatelle al ragù (Tagliatelle mit Bologneser Sauce)@{Tagliatelle al ragù (Tagliatelle mit Bologneser Sauce)\/}}
}{
\begin{multicols}{2}



für Pasta aus 3 Eiern oder

500 g Fertignudeln (z.B. Tagliatelle, Rigatoni, Pappardelle)

3 EL Olivenöl

75 g Butter

2 EL fein gewürfelte Zwiebeln

2 EL fein gewürfelte Karotten

2 EL fein gewürfelter Sellerie

350 g gehacktes mageres Rindfleisch

Salz

250 ml trockener Weißwein

120 ml Vollmilch

1/8 TL frisch geriebene Muskatnuss

500 g geschälte Dosentomaten mit Saft, grob geschnitten

60 g frisch geriebener Parmesan


\end{multicols}

In einem Topf mit schwerem Boden das Olivenöl und etwas mehr als die Hälfte der Butter erhitzen und darin die Zwiebelwürfel goldbraun dünsten. Die Karotten- und Selleriewürfel einstreuen und dünsten, bis sie etwas Farbe bekommen. Das Hackfleisch zufügen, leicht salzen, mit einem Holzlöffel auseinander zupfen und unter Rühren anbraten.

Anschließend den Wein einrühren und unter ständigem Rühren vollständig einkochen lassen. Die Milch zugießen und die geriebene Muskatnuss einstreuen. Unter ständigem Rühren die Milch fast vollständig einkochen lassen.

Zuletzt die Dosentomaten zugeben. Sobald sie aufkochen, die Hitze auf die niedrigste Stufe zurückschalten. Die Sauce im offenen Topf mindestens 3 Stunden kochen lassen, dabei gelegentlich umrühren.

(Die Sauce kann jetzt abgekühlt, im Kühlschrank aufbewahrt oder eingefroren werden. Beim Erwärmen etwas Wasser zufügen.)

Die Pasta mit 1 EL Salz in 4l kochendes Wasser schütten, gut umrühren und al dente kochen. Danach abtropfen lassen und in der wieder erhitzten Sauce wenden, dabei die restliche Butter und den frisch geriebenen Käse zufügen. Abschmecken und sofort servieren.


{\bfseries Menge:} 4 Portionen

{\bfseries Quelle:} Giuliano Hazan: Klassische Pastaküche 

} 

%----------------------
\nopagebreak{ 
\subsection{Tortelloni bi biete  (Tortelloni mit Mangoldfüllung)}

\index{italienisch@{italienisch\/}!Tortelloni bi biete  (Tortelloni mit Mangoldfüllung)@{Tortelloni bi biete  (Tortelloni mit Mangoldfüllung)\/}}
\index{Mangold@{Mangold\/}!Tortelloni bi biete  (Tortelloni mit Mangoldfüllung)@{Tortelloni bi biete  (Tortelloni mit Mangoldfüllung)\/}}
\index{Pasta@{Pasta\/}!Tortelloni bi biete  (Tortelloni mit Mangoldfüllung)@{Tortelloni bi biete  (Tortelloni mit Mangoldfüllung)\/}}
\index{Spinat@{Spinat\/}!Tortelloni bi biete  (Tortelloni mit Mangoldfüllung)@{Tortelloni bi biete  (Tortelloni mit Mangoldfüllung)\/}}
}{
\begin{multicols}{2}

\textit{Tortelloni}



Pastateig aus 2 Eiern

1 kg entstielter Mangold oder Spinat oder

625 g tiefgefrorener Blattspinat

Salz

60 g Butter

4 EL feingewürfelte Zwiebeln

60 g feingehackter Rohschinken

200 g Ricotta

1 Eigelb

60 g frisch geriebener Parmesan

1/8 TL geriebene Muskatnuss

\textit{Sauce}



Sugo burro e pomodoro (Tomaten-Butter-Sauce) oder

90 g Butter in kleinen Flocken

60 g frisch geriebener Parmesan


\end{multicols}

Die Mangold- oder Spinatblätter in mehrfach gewechseltem kaltem Wasser gründlich waschen. In einen Topf geben, mit 1/2 TL Salz bestreuen und über mittlerer Hitze nur in dem Wasser, das noch vom Waschen an den Blättern ist, in etwas 8-12 Minuten weich dünsten. Tiefgefrorener Spinat wird etwa 3 Minuten in kochendem Salzwasser gegart.
Die Blätter abtropfen lassen. Nachdem sie handwarm abgekühlt sind, das restliche Wasser ausdrücken und die Blätter fein hacken.

Die Butter in einer Pfanne über mittlerer Hitze schmelzen und darin die Zwiebelstücke glasig dünsten. Den Rohschinken beigeben und 1 Minute anbraten. Den gehackten Mangold oder Spinat unterheben und etwa 3 Minuten dünsten. Danach in eine Schüssel schütten und abkühlen lassen.

Mit dem Ricotta, dem Eigelb, dem geriebenen Käse und der geriebenen Muskatnuss gründlich vermischen. Mit etwas Salz abschmecken.

Den Pastateig so dünn wie möglich ausrollen. Tortelloni füllen und auf einem sauberen Küchentuch ausbreiten.

In einem großen Topf 4 l Wasser zum Kochen bringen. Je 1 EL Salz und Olivenöl einrühren und die gefüllten Tortelloni vorsichtig vom Küchentuch ins kochende Wasser gleiten lassen. Sie sind gar, wenn die fest zusammengepressten Ränder al dente sind. Danach herausheben und abtropfen lassen.

Die Tortelloni vorsichtig mit der heißen Burro-e-pomodoro-Sauce oder nur mit der zerlassenen Butter und dem geriebenen Käse vermischen. Sofort servieren.

Abwandlung: Alternativ reicht die Füllung für 250 g (gekaufte) Canneloni, dann 20 Minuten bei 200 Grad in der Tomatensauce backen.


{\bfseries Menge:} 4 Portionen

{\bfseries Quelle:} Giuliano Hazan: Klassische Pastaküche 

} 

%----------------------
\nopagebreak{ 
\subsection{Tortelloni di ricotta e prezzemolo (Tortelloni mit Ricotta-Petersilienfüllung)}

\index{italienisch@{italienisch\/}!Tortelloni di ricotta e prezzemolo (Tortelloni mit Ricotta-Petersilienfüllung)@{Tortelloni di ricotta e prezzemolo (Tortelloni mit Ricotta-Petersilienfüllung)\/}}
\index{Pasta@{Pasta\/}!Tortelloni di ricotta e prezzemolo (Tortelloni mit Ricotta-Petersilienfüllung)@{Tortelloni di ricotta e prezzemolo (Tortelloni mit Ricotta-Petersilienfüllung)\/}}
\index{Petersilie@{Petersilie\/}!Tortelloni di ricotta e prezzemolo (Tortelloni mit Ricotta-Petersilienfüllung)@{Tortelloni di ricotta e prezzemolo (Tortelloni mit Ricotta-Petersilienfüllung)\/}}
\index{vegetarisch@{vegetarisch\/}!Tortelloni di ricotta e prezzemolo (Tortelloni mit Ricotta-Petersilienfüllung)@{Tortelloni di ricotta e prezzemolo (Tortelloni mit Ricotta-Petersilienfüllung)\/}}
}{
\begin{multicols}{2}



rosa Tomatensauce

Pastateig aus 2 Eiern

300 g Ricotta

60 g fein geschnittene glattblättrige Petersilie

1 Eigelb

1/8 TL frisch geriebene Muskatnuss

125 g frisch geriebener Parmesan

Salz

frisch gemahlener schwarzer Pfeffer


\end{multicols}

Eine rosa Tomatensauce zubereiten.
In einer Schüssel den Ricotta, die Petersilie, das Eigleb, die geriebene Muskatnuss und die Hälfte des geriebenen Käses mit einer Gabel vermischen. Mit Salz und Pfeffer abschmecken.

Den Pastateig so dünn wie möglich ausrollen. Tortelloni füllen und auf einem sauberen Küchentuch ausbreiten.
In einem großen Topf 4 l Wasser zum Kochen bringen. Je 1 EL Salz und Olivenöl einrühren und die gefüllten Tortelloni vorsichtig vom Küchentuch ins kochende Wasser gleiten lassen. Sie sind gar, wenn die fest zusammengepressten Ränder al dente sind. Danach herausheben und abtropfen lassen.
Die Tortelloni vorsichtig mit der heißen Sauce und dem restlichen Parmesan vermischen. Sofort servieren.

Abwandlung: Alternativ reicht die Füllung für 250 g (gekaufte) Canneloni, dann 20 Minuten bei 200 Grad in der Tomatensauce backen.


{\bfseries Menge:} 4 Portionen

{\bfseries Quelle:} Giuliano Hazan: Klassische Pastaküche 

} 

\pagebreak 
 
%----------------------------------------------------
\nopagebreak{ 
\section{SAUCE} 

\subsection{Béchamelsauce}

\index{vegetarisch@{vegetarisch\/}!Béchamelsauce@{Béchamelsauce\/}}
}{
\begin{multicols}{2}



1/2 l Vollmilch

60 g Butter

4 EL Mehl (gestrichen)

Salz

frisch gemahlener weißer Pfeffer


\end{multicols}

Die Milch aufkochen und vom Feuer nehmen. Inzwischen die Butter in einer Kasserolle mit schwerem Boden auf mittlerer Flamme schmelzen. Mit einem Rührbesen das Mehl einrühren. Unter ständigem Rühren 1-2 Minuten hell anschwitzen. Nicht bräunen lassen!

Zuerst die Milch nur esslöffelweise zufügen. Die Sauce immer erst glattrühren, bevor weitere Milch zugegossen wird. Erst wenn die Sauce dünnflüssig geworden ist, kann die restliche Milch zügig zugegossen werden.

Unter ständigem Rühren mit dem Rührbesen die Sauce über milder Hitze weiterkochen, bis sie so sämig ist, dass sie den Rührbesen dicklich überzieht. Die Sauce mit Salz und weißem Pfeffer abschmecken, bevor sie vom Feuer genommen wird. Béchamelsauce sollte zwar stets am gleichen Tag verwendet werden, sie kann aber auch über Nacht im Kühlschrank aufbewahrt werden.


{\bfseries Menge:} 4 Portionen

{\bfseries Quelle:} Giuliano Hazan: Klassische Pastaküche 

} 

%----------------------
\nopagebreak{ 
\subsection{Curry-Ketchup}

\index{Curry@{Curry\/}!Curry-Ketchup@{Curry-Ketchup\/}}
\index{Tomaten@{Tomaten\/}!Curry-Ketchup@{Curry-Ketchup\/}}
\index{vegetarisch@{vegetarisch\/}!Curry-Ketchup@{Curry-Ketchup\/}}
}{
\begin{multicols}{2}



750 g frische Tomaten

oder

1 kleine Dose Tomaten

2 Schalotten

2 rote Chilischoten

2 TL scharfes Currypulver

oder

2 TL Currypaste (statt Chili und Curry)

3 EL Olivenöl

1 Knoblauchzehe

2 TL gelbe Senfsaat

5 EL Geflügelbrühe

3 EL Orangensaft

3 EL Cola

oder

1 TL brauner Zucker und

3 EL Orangensaft


\end{multicols}

Tomaten blanchieren, abschrecken, häuten und entkernen. Alternativ: aus der Dose nehmen und abtrofen lassen. Fruchtfleisch würfeln.

Schalotten fein würfeln, Chilischoten hlbieren, entkernen und fein würfeln.

Schalotten und Chili in heißen Olivenöl glasig dünsten, knoblauch dazu pressen.

Senfsaat und Currypulver bzw. Currypaste unterrühren. Mit Geflügelbrühe, Orangensaft und Cola oder Zucker aufkochen.

Tomaten unterrühren, offen aufkochen und bei milder Hitze 5 Min. kochen lassen. Sauce kräftig abschmecken.


{\bfseries Menge:} 4 Portionen

{\bfseries Quelle:} essen und trinken für jeden tag 01/08, Tim Mälzer: Ham'se noch Hack 

} 

%----------------------
\nopagebreak{ 
\subsection{Einfache Tomatensauce}

\index{Tomaten@{Tomaten\/}!Einfache Tomatensauce@{Einfache Tomatensauce\/}}
\index{vegetarisch@{vegetarisch\/}!Einfache Tomatensauce@{Einfache Tomatensauce\/}}
}{
\begin{multicols}{2}



1 EL Olivenöl oder Butter

1 Zwiebel, fein gehackt

1 gehackte Knoblauchzehe nach Belieben

400 g geschälte Tomaten mit Saft aus der Dose oder

500 ml passierte Tomaten

Salz

1 EL Zucker


\end{multicols}

Ergibt mindestens 300 ml Sauce, passt vor allem zu Pasta, z.B. für Canneloni zum Überbacken.

Das Öl auf mittlerer Flamme erhitzen. Die Zwiebel und nach Belieben den Knoblauch zufügen. Zugedeckt etwa 4 Minuten weich dünsten.

Die Tomaten mit ihrem Saft oder passierte Tomaten einrühren und im offenen Topf etwa 15 Minuten auf kleiner Flamme sämig einkochen. Falls nötig pürieren. Mit Salz und Zucker abschmecken.


{\bfseries Menge:} 4 Portionen

{\bfseries Quelle:} Rose Elliot: Klassische vegetarische Küche, abgewandelt von Frauke 

} 

%----------------------
\nopagebreak{ 
\subsection{Fetakäseaufstrich}

\index{Feta@{Feta\/}!Fetakäseaufstrich@{Fetakäseaufstrich\/}}
\index{Ricotta@{Ricotta\/}!Fetakäseaufstrich@{Fetakäseaufstrich\/}}
\index{vegetarisch@{vegetarisch\/}!Fetakäseaufstrich@{Fetakäseaufstrich\/}}
}{
\begin{multicols}{2}



100 g Ricotta

175 g Fetakäse

3 EL Olivenöl

1/4 TL frisch gemahlener schwarzer Pfeffer


\end{multicols}

Alle Zutaten in eine Schüssel geben. Mit einer Gabel gut vermischen und den Käse zerdrücken. Die Mischung sollte jedoch nicht völlig glattgerührt werden. Kann bis zu drei Tagen im voraus zubereitet werden und in einem luftdichten Behälter im Kühlschrank aufbewahrt werden.


{\bfseries Menge:} 250 Portionen

{\bfseries Quelle:} Anne Wilson: Mediterran genießen 

} 

%----------------------
\nopagebreak{ 
\subsection{Hummus}

\index{Kichererbsen@{Kichererbsen\/}!Hummus@{Hummus\/}}
\index{vegetarisch@{vegetarisch\/}!Hummus@{Hummus\/}}
\index{vorbereiten@{vorbereiten\/}!Hummus@{Hummus\/}}
}{
\begin{multicols}{2}



450 g Kichererbsen aus der Dose

1 Zwiebel

60 ml Olivenöl

1 TL Kreuzkümmel

2 zerdrückte Knoblauchzehen

1 Prise Cayennepfeffer

2 EL Sesampaste

30 ml Zitronensaft

30 ml Wasser

Salz, Pfeffer


\end{multicols}

Kichererbsen aus der Dose abspülen, Zwiebel pellen und hacken, Knoblauchzehen zerdrücken. Alles zusammen mit Kreuzkümmel und Cayennpfeffer in Olivenöl auf höchster Stufe eine Minute braten, bis es aromatisch duftet.

In eine Küchenmaschine geben und zusammen mit Sesampaste, Zitronensaft und Wasser zerkleinern und glattrühren. Evtl. Wasser oder Olivenöl zugeben. Mit Salz, Pfeffer und Zitronensaft abschmecken.

Kann bis zu 5 Tagen im Voraus zubereitet werden und in einem luftdichten Behälter im Kühlschrank aufbewahrt werden (mit Olivenöl abdecken).


{\bfseries Menge:} 8 Portionen

{\bfseries Quelle:} Anne Wilson: Ausgefallene Appetizer / Mediterran genießen, kombiniert von Frauke 

} 

%----------------------
\nopagebreak{ 
\subsection{Lachsdip}

\index{Fisch@{Fisch\/}!Lachsdip@{Lachsdip\/}}
\index{Lachs@{Lachs\/}!Lachsdip@{Lachsdip\/}}
}{
\begin{multicols}{2}



200 g Räucherlachs

60 ml Sahne

100 g Rahmfrischkäse

30 ml Zitronensaft

2 EL frischer Dill

1 EL gehackter Schnittlauch

Salz, Pfeffer


\end{multicols}

Alle Zutaten in der Küchenmaschine gut vermischen, bis die Masse glatt ist. Mit Salz und frisch gemahlenem schwarzen Pfeffer würzen. Im Kühlschrank bis 1 Tag haltbar.


{\bfseries Menge:} 4 Portionen

{\bfseries Quelle:} Anne Wilson: Ausgefallene Appetizer 

} 

%----------------------
\nopagebreak{ 
\subsection{Orangensauce}

\index{Orangen@{Orangen\/}!Orangensauce@{Orangensauce\/}}
\index{vegetarisch@{vegetarisch\/}!Orangensauce@{Orangensauce\/}}
}{
\begin{multicols}{2}



2 unbehandelte Orangen

2 EL brauner Zucker

1 EL kalte, gewürfelte Butter


\end{multicols}

Für die Soße die Orangen heiß waschen und abtrocknen. Die Schale von einer fein abreiben, den Saft von beiden auspressen. Den Zucker in einer Pfanne hellgelb karamellisieren, dann mit dem Orangensaft ablöschen und einköcheln lassen. Die Butter dazugeben und unterrühren. Mit einer Prise Salz und weißem Pfeffer abschmecken.

Da immer zuwenig von der Sauce da ist, sollte man sich nicht scheuen, ruhig die doppelte Menge zuzubereiten.


{\bfseries Menge:} 6 Portionen

{\bfseries Quelle:} www.vox.de: Schmeckt nicht, gibt's nicht 

} 

%----------------------
\nopagebreak{ 
\subsection{Paprikasauce}

\index{Paprika@{Paprika\/}!Paprikasauce@{Paprikasauce\/}}
\index{vegetarisch@{vegetarisch\/}!Paprikasauce@{Paprikasauce\/}}
}{
\begin{multicols}{2}



1 große Paprikaschote, entkernt

300 ml Gemüsebrühe oder Wasser

1 geschälte Knoblauchzehe oder

1 Zweig Thymian

15 g Butter oder

1 EL Olivenöl

Salz

frisch gemahlener schwarzer Pfeffer

1 Prise Cayennepfeffer nach Belieben


\end{multicols}

Je nach Farbe der Paprikaschote erhält man eine rote, gelbe oder grüne Paprikasauce.

Die Parikaschote in gleichmäßige Stücke schneiden und zusammen mit der Brühe oder dem Wasser, dem Thymian oder dem Knoblauch zum Kochen bringen. Über reduzierter Hitze im geschlossenen Topf etwa 10 Minuten kochen lassen, bis der Paprika weich ist. Den Thymian oder Knoblauch entfernen.

Sauce pürieren. Anschließend Butter oder Öl einarbeiten. Mit Salz, Pfeffer und nach Belieben mit Cayennepfeffer würzen. Die Sauce wieder sanft erhitzen, wenn sie warm serviert wird.


{\bfseries Menge:} 4 Portionen

{\bfseries Quelle:} Rose Elliot: Klassische vegetarische Küche 

} 

%----------------------
\nopagebreak{ 
\subsection{Rotweinsauce}

\index{Alkohol@{Alkohol\/}!Rotweinsauce@{Rotweinsauce\/}}
\index{Rotwein@{Rotwein\/}!Rotweinsauce@{Rotweinsauce\/}}
\index{vegetarisch@{vegetarisch\/}!Rotweinsauce@{Rotweinsauce\/}}
}{
\begin{multicols}{2}



60 g Butter

2 Schalotten, feingehackt

2 TL fein gehackter, frischer Thymian

1 gehackte Knoblauchzehe nach Belieben

300 ml Rotwein

3 EL Portwein

1/2 TL Gemüsebrühenpulver oder -würfel

Salz

frisch gemahlener schwarzer Pfeffer


\end{multicols}

Die Hälfte der Butter im Kühlschrank gut durchkühlen.

Die andere Hälfte der Butter in einem mittelgroßen Topf über mittlerer Hitze schmelzen.

Die Schalotten, den Thymian und nach Belieben den Knoblauch zufügen und 5 Minuten dünsten. Wein und Portwein zugießen und Brühpulver einrühren. Zum Kochen bringen und die Flüssigkeit auf die Hälfte einkochen.

Unmittelbar vor dem Servieren die kalte Butter in kleinen Flocken abseits vom Herd in die heiße Sauce einrühren (montieren).


{\bfseries Menge:} 4 Portionen

{\bfseries Quelle:} Rose Elliot: Klassische vegetarische Küche 

} 

%----------------------
\nopagebreak{ 
\subsection{Schnelle Kräutersauce}

\index{vegetarisch@{vegetarisch\/}!Schnelle Kräutersauce@{Schnelle Kräutersauce\/}}
}{
\begin{multicols}{2}



4 EL geschnittene Kräuter (Schnittlauch, Petersilie)

200 ml Crème fraîche oder Joghurt

Salz

frisch gemahlener schwarzer Pfeffer


\end{multicols}

Die Kräuter mit der Crème fraîche oder dem Joghurt verrühren. Die Sauce abschmecken und sofort servieren oder, im Kühlschrank gekühlt, noch am selben Tag verwenden.


{\bfseries Menge:} 4 Portionen

{\bfseries Quelle:} Rose Elliot: Klassische vegetarische Küche 

} 

%----------------------
\nopagebreak{ 
\subsection{Tomaten und Ricotta - Sauce}

\index{kalorienreich@{kalorienreich\/}!Tomaten und Ricotta - Sauce@{Tomaten und Ricotta - Sauce\/}}
\index{Pasta@{Pasta\/}!Tomaten und Ricotta - Sauce@{Tomaten und Ricotta - Sauce\/}}
\index{Ricotta@{Ricotta\/}!Tomaten und Ricotta - Sauce@{Tomaten und Ricotta - Sauce\/}}
\index{Tomaten@{Tomaten\/}!Tomaten und Ricotta - Sauce@{Tomaten und Ricotta - Sauce\/}}
\index{vegetarisch@{vegetarisch\/}!Tomaten und Ricotta - Sauce@{Tomaten und Ricotta - Sauce\/}}
}{
\begin{multicols}{2}



10 reife Tomaten

Olivenöl

150 g Ricotta

1 Bund Basilikum

Salz und Pfeffer


\end{multicols}

Die Sauce passt am besten zu Kartoffelgnocchi.

Die Tomaten vierteln, entkernen und in kleine Würfel schneiden. Olivenöl in der Pfanne erhitzen und die Tomaten darin schmelzen. Mit Salz und Pfeffer abschmecken.

Das Basilikum in Streifen schneiden.

Die Gnocchi zu den Tomaten hinzugeben und erwärmen. Unmittelbar vor dem Servieren die Basilikumstreifen hinzugeben und Ricotta unterheben.

Nochmals abschmecken und servieren.


{\bfseries Menge:} 4 Portionen

{\bfseries Quelle:} www.vox.de: Schmeckt nicht, gibt's nicht 

} 

%----------------------
\nopagebreak{ 
\subsection{Tomatensauce}

\index{Tomaten@{Tomaten\/}!Tomatensauce@{Tomatensauce\/}}
\index{vegetarisch@{vegetarisch\/}!Tomatensauce@{Tomatensauce\/}}
}{
\begin{multicols}{2}



450 g grob gehackte Tomaten

50 g fein gehackte Zwiebeln

1 EL Oliven- oder Sonnenblumenöl

1 EL Weizenschrotmehl

150 ml Wasser

1 Lorbeerblatt

1 Knoblauchzehe, geschält und fein gehackt (nach Wunsch)

Salz

frisch gemahlener Pfeffer


\end{multicols}

Zwiebeln in Öl glasig dünsten, Mehl, dann Wasser einrühren. Tomaten mit Lorbeerblatt und, nach Wunsch, Knoblauch zugeben. Unter Rühren zum Kochen bringen, 30 Minuten simmern lassen, dabei von Zeit zu Zeit umrühren. Sauce durch ein Sieb streichen, nachwürzen und vor dem Servieren erhitzen.

Für eine pikantere Sauce 1 EL Wein- oder Apfelweinessig zum Wasser geben und nach dem Passieren 1/2 EL fein gehackte Kapern zufügen.


{\bfseries Bemerkung:} Eiweiß: 1 g, kJ: 247 

{\bfseries Menge:} 4 Portionen

{\bfseries Quelle:} Paul Southey: Die Feinheiten der vegetarischen Küche 

} 

%----------------------
\nopagebreak{ 
\subsection{Zaziki}

\index{Joghurt@{Joghurt\/}!Zaziki@{Zaziki\/}}
\index{Knoblauch@{Knoblauch\/}!Zaziki@{Zaziki\/}}
\index{vegetarisch@{vegetarisch\/}!Zaziki@{Zaziki\/}}
}{
\begin{multicols}{2}



1 kleine Salatgurke

250 ml Naturjoghurt

2 Zehen Knoblauch

1 EL Rotweinessig

1 EL Olivenöl

1/4 TL Salz

1/3 TL gemahlener schwarzer Pfeffer


\end{multicols}

Die Gurke waschen und ungeschält reiben, austropfen lassen, Knoblauch zerdrücken, mit den übrigen Zutaten gut vermengen und servieren. Kann bis zu 3 Tage im voraus zubereitet und in einem luftdichten Behälter im Kühlschrank gelagert werden.


{\bfseries Menge:} 4 Portionen

{\bfseries Quelle:} Anne Wilson: Mediterran genießen 

} 

\pagebreak 
 
%----------------------------------------------------
\nopagebreak{ 
\section{AUFLAUF} 

\subsection{Belgischer Rosenkohlauflauf}

\index{Auflauf@{Auflauf\/}!Belgischer Rosenkohlauflauf@{Belgischer Rosenkohlauflauf\/}}
\index{Hackfleisch@{Hackfleisch\/}!Belgischer Rosenkohlauflauf@{Belgischer Rosenkohlauflauf\/}}
\index{Käse@{Käse\/}!Belgischer Rosenkohlauflauf@{Belgischer Rosenkohlauflauf\/}}
\index{Rosenkohl@{Rosenkohl\/}!Belgischer Rosenkohlauflauf@{Belgischer Rosenkohlauflauf\/}}
\index{Schweinefleisch@{Schweinefleisch\/}!Belgischer Rosenkohlauflauf@{Belgischer Rosenkohlauflauf\/}}
\index{vorbereiten@{vorbereiten\/}!Belgischer Rosenkohlauflauf@{Belgischer Rosenkohlauflauf\/}}
}{
\begin{multicols}{2}

\textit{Rosenkohl}



1 kg Rosenkohl

60 g roher durchwachsener Speck

1 EL Öl

250 g Hackfleisch vom Schwein

150 g Tomaten (evtl. aus der Dose)

Salz

frisch gemahlener schwarzer Pfeffer

1 EL edelsüßes Paprikapulver

1 EL gehackte Petersilie

\textit{Sauce}



25 g Butter

30 g Mehl

1/2 l Milch

Salz

frisch gemahlener Pfeffer

frisch geriebene Muskatnuss

1 Eigelb

1 EL Sahne, geschlagen

60 g frisch geriebener Passendale (Käse, mild bis dezent-pikant)

100 ml Sahne, geschlagen


\end{multicols}

Den Rosenkohl putzen, es sollten etwa 800 g übrig bleiben. Die äußeren Blättchen entfernen und den Strunk eines jeden Rosenkohl-Köpfchens frisch anschneiden und kreuzweise anschneiden. Salzwasser in einem entsprechend großen Topf zum Kochen bringen und den Rosenkohl 10 bis 12 Minuten kochen. Herausnehmen und längs halbieren.

Die Zwiebeln schälen und fein hacken. Den Speck in kleine Würfel schneiden. Das Öl in einer Pfanne erhitzen und den Speck darin anbraten. Die Zwiebelwürfel hell mitschwitzen. Das Hackfleisch darin braten, bis es krümelig zerfällt.

Die Tomaten kurz blanchieren, häuten, Samen und Scheidewände entfernen und das Fruchtfleisch mit einer Gabel zerdrücken. Zum Hackfleisch geben, mit Salz, Pfeffer und Paprikapulver würzen, die Petersilie einrühren und 5 Minuten schmoren lassen.

Für die Sauce die Butter zerlassen und das Mehl unter Rühren darin 1 bis 2 Minuten farblos anschwitzen. Die Milch zugießen, glattrühren und mit Salz, Pfeffer und Muskatnuss würzen. Etwa 20 Minuten unter ständigem Rühren kochen. Das Eigelb mit 1 EL Sahne verquirlen, die Sauce damit legieren und unter Rühren einmal kräftig aufkochen. Die Sauce durch ein feines Sieb passieren.

Nochmals erhitzen und den Käse unter mehrmaligem Rühren darin schmelzen. Erst dann die geschlagene Sahne unterheben.

Erst die Hackfleischmasse in eine Auflaufform einfüllen und den Rosenkohl darüber verteilen. Mit der Sauce begießen und bei 20$^\circ$C im vorgeheizten Ofen etwa 25 Minuten backen.

Dazu passen Kartoffeln.


{\bfseries Menge:} 4 Portionen

{\bfseries Quelle:} Ch. Teubner: Die 100 besten Rezepte aus aller Welt: Aufläufe, Soufflés \& Gratins 

} 

%----------------------
\nopagebreak{ 
\subsection{Blumenkohlauflauf}

\index{Blumenkohl@{Blumenkohl\/}!Blumenkohlauflauf@{Blumenkohlauflauf\/}}
\index{vorbereiten@{vorbereiten\/}!Blumenkohlauflauf@{Blumenkohlauflauf\/}}
}{
\begin{multicols}{2}

\textit{Blumenkohlauflauf}



1 Blumenkohl (1 kg)

Salz

100 g Zwiebel

80 g durchwachsener Speck

250 g Fleischwurst

1 EL Öl

frisch gemahlener weißer Pfeffer

\textit{Für die Sauce:}



25 g Butter

30 g Mehl,

1/2 l Milch

Salz

frisch gemahlener Pfeffer

1 Eigelb

100 ml Sahne

60 g geriebener Havarti oder Tilsiter

1 Bund Schnittlauch

\textit{Außerdem:}



Butter für die Form


\end{multicols}

Von dem Blumenkohl die eng anliegenden Hüllblätter entfernen. Den Strunk herausschneiden und den Kohl in mittelgroße Röschen zerteilen; es sollen etwa 750 g verbleiben. Unter fließendem Wasser waschen und in sprudelnd kochendem Salzwasser 10 Minuten kochen. Abgießen und gut abtropfen lassen.

Die Zwiebeln schälen und ebenso wie den Speck fein würfeln. Die Fleischwurst häuten und in 5 mm große Würfel schneiden. Das Öl in einer Pfanne erhitzen und den Speck darin ausbraten. Die Zwiebel- und Wurstwürfel kurz mitbraten. Je nach Bedarf mit etwas Salz und Pfeffer würzen. Die Blumenkohlröschen zufügen und alles gut miteinander vermischen.

Für die Sauce die Butter zerlassen und das Mehl darin unter Rühren 1 bis 2 Minuten farblos anschwitzen. Die Milch zugießen, glattrühren, salzen und pfeffern. Etwa 20 Minuten unter ständigem Rühren kochen. Das Eigelb mit der Sahne verquirlen, die Sauce damit legieren und unter Rühren einmal aufkochen. Durch ein feines Sieb passieren.
Den Käse unter Rühren darin schmelzen. Den Schnittlauch waschen, trockenschütteln, in Röllchen schneiden und einstreuen.

Eine Form ausfetten und den Blumenkohl einfüllen. Die Sauce darübergießen und bei 200$^\circ$C im vorgeheizten Ofen 25 bis 30 Minuten backen.


{\bfseries Menge:} 4 Portionen

{\bfseries Quelle:} C. Teubner: Die 100 besten Rezepte aus aller Welt, Aufläufe, Souffles \& Gratins 

} 

%----------------------
\nopagebreak{ 
\subsection{Kartoffel-Lauch-Auflauf mit Gemüsebrühe}

\index{Kartoffeln@{Kartoffeln\/}!Kartoffel-Lauch-Auflauf mit Gemüsebrühe@{Kartoffel-Lauch-Auflauf mit Gemüsebrühe\/}}
\index{Lauch@{Lauch\/}!Kartoffel-Lauch-Auflauf mit Gemüsebrühe@{Kartoffel-Lauch-Auflauf mit Gemüsebrühe\/}}
\index{vegetarisch@{vegetarisch\/}!Kartoffel-Lauch-Auflauf mit Gemüsebrühe@{Kartoffel-Lauch-Auflauf mit Gemüsebrühe\/}}
\index{vorbereiten@{vorbereiten\/}!Kartoffel-Lauch-Auflauf mit Gemüsebrühe@{Kartoffel-Lauch-Auflauf mit Gemüsebrühe\/}}
}{
\begin{multicols}{2}



ungefähr zu gleichen Teilen

Kartoffeln

Lauch

Gemüsebrühe

Käse zum Bestreuen


\end{multicols}

Einfach Lauch und Kartoffeln in Scheiben abwechselnd in einer Auflaufform schichten oder aufrecht stellen, mit Gemüsebrühe übergießen (auffüllen) und im Ofen ohne Abdeckung ca. 1 Stunde bei 200$^\circ$C backen (Brühe sollte aufgesogen sein, evtl. nachgießen). Zu Ende der Backzeit nach Belieben mit Käse bestreuen und Grill zuschalten.


{\bfseries Menge:} 4 Portionen

{\bfseries Quelle:} Internet 

} 

%----------------------
\nopagebreak{ 
\subsection{Kartoffelgratin}

\index{Auflauf@{Auflauf\/}!Kartoffelgratin@{Kartoffelgratin\/}}
\index{Gratin@{Gratin\/}!Kartoffelgratin@{Kartoffelgratin\/}}
\index{Kartoffeln@{Kartoffeln\/}!Kartoffelgratin@{Kartoffelgratin\/}}
\index{Zwiebeln@{Zwiebeln\/}!Kartoffelgratin@{Kartoffelgratin\/}}
}{
\begin{multicols}{2}



800 g Kartoffeln

3 Frühlingszwiebeln

Butter (zum Ausfetten)

125 ml Gemüsebrühe

125 ml Sahne

125 ml Milch

1 Schuss Muskatnuss

4 Scheib. Frühstücksspeck

50 g geriebener Parmesan

50 g geriebener Gouda

Salz

Pfeffer


\end{multicols}

Die Kartoffeln schälen und in dünne Scheiben schneiden; die Frühlingszwiebeln putzen und in feine Ringe schneiden. Erst die Kartoffeln in eine gefettete Auflaufform schichten, dann darüber die Frühlingszwiebeln verteilen.

 Gemüsebrühe, Sahne und Milch verquirlen und mit Salz, Pfeffer und Muskat abschmecken. Die Kartoffeln mit der Sauce begießen und im auf 180$^\circ$C vorgeheizten Ofen ca. 30 Minuten garen. Nun den Speck auf das Gratin legen, den geriebenen Käse mischen, darüber streuen und weitere 20 Minuten im Ofen überbacken.

 Als Beilage empfiehlt sich ein frischer Salat.


{\bfseries Menge:} 4 Portionen

{\bfseries Quelle:} www.kochfreunde.de 

} 

%----------------------
\nopagebreak{ 
\subsection{Lauch-Nuss-Auflauf}

\index{Auflauf@{Auflauf\/}!Lauch-Nuss-Auflauf@{Lauch-Nuss-Auflauf\/}}
\index{Käse@{Käse\/}!Lauch-Nuss-Auflauf@{Lauch-Nuss-Auflauf\/}}
\index{Schweinefleisch@{Schweinefleisch\/}!Lauch-Nuss-Auflauf@{Lauch-Nuss-Auflauf\/}}
\index{vorbereiten@{vorbereiten\/}!Lauch-Nuss-Auflauf@{Lauch-Nuss-Auflauf\/}}
}{
\begin{multicols}{2}



1 1/2 kg Lauch

400 g gekochtes Kasseler

Salz

65 g Butter

30 g Mehl

1/4 l Schlagsahne

150 g frisch geriebener Emmentaler Käse

3 Eier

frisch gemahlener schwarzer Pfeffer

geriebene Muskatnuss

50 g gehackte Haselnüsse

3 EL Semmelbrösel

50 g frisch geriebener Parmesan


\end{multicols}

Den Lauch putzen, waschen und nur die geschlossenen Teile schräg in 2-3 cm lange Stücke schneiden. In reichlich Salzwasser 5 Minuten blanchieren, abgießen und dabei 1/4 l vom Kochwasser auffangen.

Das Kasseler knapp 2 cm groß würfeln.

40 g Butter aufschäumen lassen, das Mehl hineinrühren und hell anschwitzen. Mit dem Lauchwasser und der Sahne ablöschen und einige Minuten durchkochen lassen. Zuerst den Emmentaler Käse, dann die Eier darunterrühren und die Sauce mit Salz, Pfeffer und Muskat würzen.

Lauch, Kasseler und Sauce lagenweise in eine mit Butter eingefettete Auflaufform füllen. Nüsse, Semmelbrösel und Parmesan vermischen und den Auflauf damit bestreuen. Die restliche Butter zerlassen und darüberträufeln.

Dann auf der Mittelschiene des auf 20$^\circ$C vorgeheizten Ofens etwa 45 Minuten backen.

Dazu passen Petersilienkartoffeln.


{\bfseries Menge:} 4 Portionen

{\bfseries Quelle:} Winterliches Gemüse: Frisch und pfiffig. 

} 

%----------------------
\nopagebreak{ 
\subsection{Pastinaken-Kürbis-Auflauf}

\index{Pastinaken@{Pastinaken\/}!Pastinaken-Kürbis-Auflauf@{Pastinaken-Kürbis-Auflauf\/}}
\index{Kürbis@{Kürbis\/}!Pastinaken-Kürbis-Auflauf@{Pastinaken-Kürbis-Auflauf\/}}
}{
\begin{multicols}{2}



300 g Pastinaken

400 g Hokkaido-Kürbis

1 Knoblauchzehe

200 g Kasseler-Kotelett

2 EL Olivenöl

2 EL Thymian

1 Limette

30 g Butter

1 EL Mehl

Salz

Pfeffer

50 g Walnusskerne, gehackt

3 EL Semmelbrösel


\end{multicols}

Pastinaken putzen, schälen und würfeln. Kürbis putzen und halbieren, Kerngehäuse herausschaben. Fruchtfleisch würfeln. Knoblauch hacken. Kasseler von Knochen lösen und würfeln.

Olivenöl erhitzen. Pastinaken und Kürbis dari 8 min braten. Thymian und Knoblauch kurz mitbraten. Gemüse und Kasseler in eine Auflaufform geben.

Limenttenschale fein abreiben, Saft auspressen. Butter erhitzen, Mehl zufügen und unter Rühren 3 min anschwitzen. Unter Rühren mit Milch ablöschen und 5 min köcheln lassen. Mit Salz, Pfeffer und Limettenschale würzen. Sauce von der Kochstelle ziehen und Limettensaft zufügen. Sauce über das Gemüse gießen. Mit Walnüssen und Semmelbröseln bestreuen.

Im heißen Ofen bei 200 Grad (Umluft 180 Grad) auf der mittleren Schiene ca. 25 min backen.


{\bfseries Menge:} 2 Personen

{\bfseries Quelle:} essen und trinken für jeden Tag 11/09 

} 

%----------------------
\nopagebreak{ 
\subsection{Rosenkohlauflauf mit Pfifferlingen und Pute}

\index{Pute@{Pute\/}!Rosenkohlauflauf mit Pfifferlingen und Pute@{Rosenkohlauflauf mit Pfifferlingen und Pute\/}}
\index{Rosenkohl@{Rosenkohl\/}!Rosenkohlauflauf mit Pfifferlingen und Pute@{Rosenkohlauflauf mit Pfifferlingen und Pute\/}}
\index{Schinken@{Schinken\/}!Rosenkohlauflauf mit Pfifferlingen und Pute@{Rosenkohlauflauf mit Pfifferlingen und Pute\/}}
\index{Schweinefleisch@{Schweinefleisch\/}!Rosenkohlauflauf mit Pfifferlingen und Pute@{Rosenkohlauflauf mit Pfifferlingen und Pute\/}}
}{
\begin{multicols}{2}



1 kg Rosenkohl

4 Putenbrustfilets a 100 g

3 EL Öl

2 groß. Zwiebeln

75 g Schinkenspeck

150 g Pfifferlinge (oder Trockenpilze)

Salz

Pfeffer

1 TL edelsüßer Paprika

1/2 Bund Petersilie

1/2 Bund Schnittlauch

125 Gruyeère

2 EL Paniermehl


\end{multicols}

Rosenkohl in kochendem Salzwasser etwa 10 Minuten garen. Abgießen und abtropfen lassen.

Putenbrust salzen und pfefern und  in Öl von beiden Seiten etwa 4 Minuten braten.

Zwiebel und Schinkenspeck würfeln und im Bratfett der Pute unter Rühren andünsten. Pfifferlinge dazugeben und mitdünsten.
Mit Salz, Pfeffer und Paprika würzen.

Petersilie und Schnittlauch hacken, Käse fein reiben.
Rosenkohl und Speck-Zwiebelmischung mit Petersilie und Schnittlauch vermischen und in eine gefettete Auflaufform geben.
Die Putenfillets darauflegen. Käse und Paniermehl mischen und darübersteuen.

Im Ofen bei 20$^\circ$C (Umluft 180$^\circ$C) etwa 20 Minuten backen.


{\bfseries Bemerkung:} Eiweiß: 47 g, Fett: 8 g, Kohlenhydrate: 14 g, kcal: 335, BE: 1 

{\bfseries Menge:} 4 Portionen

{\bfseries Quelle:} 365 Aufläufe und andere Ofengerichte 

} 

%----------------------
\nopagebreak{ 
\subsection{Rotbarsch--Auflauf mit Käsekruste (florentinisch)}

\index{Auflauf@{Auflauf\/}!Rotbarsch--Auflauf mit Käsekruste (florentinisch)@{Rotbarsch--Auflauf mit Käsekruste (florentinisch)\/}}
\index{einfach@{einfach\/}!Rotbarsch--Auflauf mit Käsekruste (florentinisch)@{Rotbarsch--Auflauf mit Käsekruste (florentinisch)\/}}
\index{Fisch@{Fisch\/}!Rotbarsch--Auflauf mit Käsekruste (florentinisch)@{Rotbarsch--Auflauf mit Käsekruste (florentinisch)\/}}
\index{italienisch@{italienisch\/}!Rotbarsch--Auflauf mit Käsekruste (florentinisch)@{Rotbarsch--Auflauf mit Käsekruste (florentinisch)\/}}
\index{P2@{P2\/}!Rotbarsch--Auflauf mit Käsekruste (florentinisch)@{Rotbarsch--Auflauf mit Käsekruste (florentinisch)\/}}
\index{Rotbarsch@{Rotbarsch\/}!Rotbarsch--Auflauf mit Käsekruste (florentinisch)@{Rotbarsch--Auflauf mit Käsekruste (florentinisch)\/}}
}{
\begin{multicols}{2}



200 g Rotbarschfilet (TK oder frisch)

300 g Blattspinat (TK)

200 g Kräuterfrischkäse

4 EL Weißwein

4 EL Sahne

1 TL Basilikum, gerebelt

Salz

Pfeffer

1 Tasse Reis (gekocht)

1 TL Butter für die Form

60 g mittelalter Gouda, gerieben


\end{multicols}

Fisch und Spinat auftauen lassen. Den Frischkäse mit Weißwein und Sahne glattrühren und mit den Gewürzen (evtl. auch Knoblauch) pikant abschmecken. Auflaufform ausfetten. Reis, Spinat und Fisch einschichten. Jede Schicht salzen und pfeffern. Käsesauce darübergießen und mit Gouda bestreuen. Im vorgeheizten Backofen bei 20$^\circ$C ca. 20 Minuten garen.


{\bfseries Menge:} 2 Portionen

{\bfseries Quelle:} Internet 

} 

%----------------------
\nopagebreak{ 
\subsection{Shepherd's Pie}

\index{Kartoffeln@{Kartoffeln\/}!Shepherd's Pie@{Shepherd's Pie\/}}
\index{Lamm@{Lamm\/}!Shepherd's Pie@{Shepherd's Pie\/}}
}{
\begin{multicols}{2}



700 g Lamm- oder Rinderhack

2 Zwiebeln, gehackt

225 g Karotten, gewürfelt

2 Knoblauchzehen, zerdrückt

1 EL Mehl

200 ml Rinderbrühe

200 g Tomaten aus der Dose

1 TL Worcestersauce

Salz und Pfeffer

1 TL frisch gehackter oder

1/4 TL getrockneter Salbei oder Oregano

1 kg Kartoffeln

25 g Butter

4 EL Milch

125 g Champignons, in Scheiben


\end{multicols}

Das Fleisch in eine Pfanne ohne Fett geben und unter häufigem Rühren bei geringer Hitze anbraten.

Zwiebeln, Karotten und Knoblauch zufügen und weitere 10 Minuten garen. Das Mehl einrühren und die Fleischmischung noch etwa eine Minute anbraten. Brühe und Tomaten zugeben und alles unter Rühren zum Kochen bringen.

Worcestersauce, Salz, Pfeffer und Salbei oder Oregano zufügen, abdecken und unter gelegentlichem Rühren ca. 25 Minuten köcheln lassen.

Den Backofen auf 20$^\circ$C vorheizen. Die Kartoffeln in Salzwasser gar kochen, gründlich abgießen und zerstampfen.
Die Butter zufügen und ausreichend Milch einarbeiten, bis der Teig zähflüssig wird. Mit Salz und Pfeffer abschmecken. Den Teig in einen Spritzbeutel mit großer Sterntüte füllen.

Die Pilze unter das Fleisch rühren und abschmecken. Die Fleischmasse in eine flache Auflaufform füllen.

Dann den Kartoffelteig gleichmäßig über das Fleisch spritzen. Im Backofen 30 Minuten backen, bis der Belag heiß und goldbraun ist.


{\bfseries Bemerkung:} Eiweiß: 33 g, Fett: 12 g, Kohlenhydrate: 37 g, kcal: 378 

{\bfseries Menge:} 4 Portionen

{\bfseries Quelle:} Low-Fat -- Die besten Rezepte aus aller Welt 

} 

%----------------------
\nopagebreak{ 
\subsection{Spanisches Spargelgratin}

\index{Auflauf@{Auflauf\/}!Spanisches Spargelgratin@{Spanisches Spargelgratin\/}}
\index{Gratin@{Gratin\/}!Spanisches Spargelgratin@{Spanisches Spargelgratin\/}}
\index{Käse@{Käse\/}!Spanisches Spargelgratin@{Spanisches Spargelgratin\/}}
\index{Spargel@{Spargel\/}!Spanisches Spargelgratin@{Spanisches Spargelgratin\/}}
\index{vorbereiten@{vorbereiten\/}!Spanisches Spargelgratin@{Spanisches Spargelgratin\/}}
}{
\begin{multicols}{2}

\textit{Spargel}



1 kg weißer Spargel

Salz

etwas Zitronensaft

200 g mit Parika gewürzte Chorizos (luftgetrocknete spanische Würste)

4 Eier (hartgekocht)

\textit{Sauce}



50 g Butter

25 g Mehl

350 ml Milch

Salz

frisch gemahlener weißer Pfeffer

\textit{Außerdem}



Butter für die Form

50 g frisch geriebener Manchego (Hartkäse aus Schafmilch)

100 g Cabrales (Blauschimmelkäse), in Scheiben geschnitten

1 TL gehackte Petersilie


\end{multicols}

Spargel schälen und in 5 bis 6 cm lange Stücke schneiden. In einem Topf genügend Wasser mit etwas Salz und Zitronensaft zum Kochen bringen, Spargelstücke einlegen und 10 bis 15 Minuten kochen. Herausnehmen und beiseite stellen.

Die Chorizos in etwa 1 cm große Würfel schneiden. Die Eier schälen und achteln.

Für die Sauce die Butter in einer Kasserolle zerlassen. Das Mehl darin farblos anschwitzen, die Milch zugießen und gut durchrühren. Mit Salz und Pfeffer würzen und bei geringer Hitze 15 Minuten köcheln lassen.

Eine Auflaufform mit Butter ausfetten und den Spargel mit den Chorizos und den Eiachteln einfüllen. Die Sauce darübergießen und mit dem geriebenen Käse bestreuen. Bei 18$^\circ$C im vorgeheizten Ofen 10 Minuten backen.

Die Form aus dem Ofen nehmen, die Käsescheiben auf dem Spargel verteilen, die Form zurück in den Ofen schieben und das Gratin weitere 5 Minuten überbacken. Mit gehackter Petersilie bestreuen.


{\bfseries Menge:} 4 Portionen

{\bfseries Quelle:} Ch. Teubner: Die 100 besten Rezepte aus aller Welt: Aufläufe, Soufflés \& Gratins 

} 

%----------------------
\nopagebreak{ 
\subsection{Zucchini-Auflauf}

\index{Auflauf@{Auflauf\/}!Zucchini-Auflauf@{Zucchini-Auflauf\/}}
\index{Zucchini@{Zucchini\/}!Zucchini-Auflauf@{Zucchini-Auflauf\/}}
\index{vegetarisch@{vegetarisch\/}!Zucchini-Auflauf@{Zucchini-Auflauf\/}}
}{
\begin{multicols}{2}



800 g Zucchini, geraffelt

Salz

200 g rohen Schinken, feine Streifen

2 Zwiebeln, gehackt

3 Knoblauchzehen, gepresst

3 Eier

3 EL Mehl

125 ml Sahne

1 Bund Petersilie, gehackt

1/2 Bund Dill, gehackt

150 g Sbrinz-Käse, gerieben, alternativ:, Parmesan


\end{multicols}

Die Zucchini mit etwas Salz vermischen und stehen lassen. In der
Zwischenzeit den Schinken in einer Pfanne auslassen. Die Zwiebeln und
den Knoblauch dazu geben und mitdünsten bis sie glasig geworden sind.

Die gesalzenen Zucchini abspülen und ausdrücken.

Eier mit Mehl und Sahne verquirlen und die abgekühlte
Schinkenmischung, Zucchini und die Kräuter gut darunter mischen.

Die Masse in eine eingefettete Form füllen und darüber den Käse
streuen. In dem auf 220 Grad vorgeheizten Backofen 30 bis 40 Minuten
überbacken.


{\bfseries Menge:} 4 Portionen

{\bfseries Inhalt pro Portion:} 1668~kJ, 31~g Eiweiß, 23~g Fett, 14~g Kohlenhydrate, 398~kcal, 1.17~BE

{\bfseries Zeit:} Gesamtzeit 50 min, Kochzeit 35 min, Vorbereitungszeit: 15 min

{\bfseries Quelle:} Was Großmutter noch wusste Kochen mit Zucchini Folge 338, vom 12. Juni 2005 

} 

%----------------------
\nopagebreak{ 
\subsection{Zwiebelauflauf}

\index{Hackfleisch@{Hackfleisch\/}!Zwiebelauflauf@{Zwiebelauflauf\/}}
\index{vorbereiten@{vorbereiten\/}!Zwiebelauflauf@{Zwiebelauflauf\/}}
\index{Zwiebeln@{Zwiebeln\/}!Zwiebelauflauf@{Zwiebelauflauf\/}}
}{
\begin{multicols}{2}

\textit{Auflauf:}



600 g Zwiebeln

300 g Tomaten (frisch oder aus der Dose)

1 Knoblauchzehe

4 EL Pflanzenöl

500 g Hackfleisch

2 EL Tomatenmark

2 TL edelsüßes Paprikapulver

Salz

frisch gemahlener Pfeffer

1 EL gehackter Liebstöckel

100 ml Rinderfond

100 ml Rotwein

\textit{Für die Käsesauce:}



15 g Butter

10 g Mehl

1/4 l Milch

Salz

frisch gemahlener schwarzer Pfeffer

80 g frisch geriebener Tilsiter

\textit{Außerdem:}



gehackter Liebstöckel zum Bestreuen


\end{multicols}

Die Zwiebeln schälen und in 2 mm dicke Scheiben schneiden. Die Tomaten blanchieren, häuten, Samen entfernen und das Fruchtfleisch würfeln. Den Knoblauch schälen und fein hacken.

Das Öl erhitzen. Die Zwiebeln darin 3 bis 4 Minuten dünsten und herausnehmen. Das zurückgebliebene Öl wieder erhitzen und das Fleisch bei starker Hitze darin anbraten. Tomaten, Knoblauch und Tomatenmark zum Fleisch geben und gut vermischen. Mit Paprikapulver, Salz, und Liebstöckel würzen, den Fond zugießen und 5 Minuten köcheln lassen.

Den Wein aufgießen und weitere 5  Minuten köcheln lassen. Die Zwiebelscheiben zurück in die Pfanne geben und unter das Hackfleisch rühren. Die Mischung, in eine ungefettete Auflaufform füllen.

Für die Sauce die Butter zerlassen und das Mehl unter Rühren darin hell anschwitzen. Die Milch zugießen salzen und pfeffern. Gut durchrühren und etwa 10 Minuten leicht köcheln lassen. Den Käse darin schmelzen.

Die Sauce in die Form gießen und den Auflauf bei 200$^\circ$C im vorgeheizten Ofen 25 Minuten backen.

Mit Liebstöckel bestreuen.


{\bfseries Menge:} 4 Portionen

{\bfseries Quelle:} C. Teubner: Die 100 besten Rezepte aus aller Welt, Aufläufe, Souffles \& Gratins 

} 

%----------------------
\nopagebreak{ 
\subsection{Überbackene Chicoréerollen}

\index{Chicoree@{Chicoree\/}!Überbackene Chicoréerollen@{Überbackene Chicoréerollen\/}}
\index{einfach@{einfach\/}!Überbackene Chicoréerollen@{Überbackene Chicoréerollen\/}}
\index{Schinken@{Schinken\/}!Überbackene Chicoréerollen@{Überbackene Chicoréerollen\/}}
\index{überbacken@{überbacken\/}!Überbackene Chicoréerollen@{Überbackene Chicoréerollen\/}}
}{
\begin{multicols}{2}



8 Scheiben gekochter Schinken

8 kleine Chicorée (oder Hälften)

4 Eier

200 ml saure Sahne

200 ml Joghurt (oder evtl. MIlch)

Salz

Pfeffer

Muskat

50 g geriebener Käse


\end{multicols}

Herd vorheizen auf 175--200 Grad. Chicorée putzen, jeweils mit einer Scheibe gekochten Schinken umwickeln. In eine Tarteform (z.B. 28 cm Durchmesser) legen. Saure Sahne, Joghurt und Eier verquirlen, mit Salz, Pfeffer und Muskat würzen, über die Chicorée gießen. Ca. 20 Minuten backen, dann den Käse darüber streuen und nochmals 10 Minuten backen.


{\bfseries Menge:} 4 Portionen

{\bfseries Quelle:} Gerald und Frauke 

} 

\pagebreak 
 
%----------------------------------------------------
\nopagebreak{ 
\section{GEMUESE} 

\subsection{Blumenkohl mit Schinken--Käse--Sauce}

\index{Blumenkohl@{Blumenkohl\/}!Blumenkohl mit Schinken--Käse--Sauce@{Blumenkohl mit Schinken--Käse--Sauce\/}}
\index{Schinken@{Schinken\/}!Blumenkohl mit Schinken--Käse--Sauce@{Blumenkohl mit Schinken--Käse--Sauce\/}}
\index{überbacken@{überbacken\/}!Blumenkohl mit Schinken--Käse--Sauce@{Blumenkohl mit Schinken--Käse--Sauce\/}}
}{
\begin{multicols}{2}



1 großer Blumenkohl (2 kg)

2 l Wasser

Salz

2 EL Butter

2 EL Mehl

weißer Pfeffer

1 Prise Muskatnuß

1 Becher Sahne--Dickmilch  (175 g)

200 g gekochter Schinken

50 g geriebener Emmentaler

1/2 Bund Petersilie


\end{multicols}

Den Blumenkohl kräftig abbrausen, die Blätter entfernen und den Strunk abschneiden, dann kreuzweise einschneiden. In reichlich kochendes Salzwasser geben und nicht zu weich kochen. Dann herausnehmen und abtropfen lassen. 1/4 l Blumenkohlbrühe für die Sauce beiseite stellen. Die Butter erhitzen, das Mehl darin kurz anschwitzen und mit der Brühe ablöschen. Unter Rühren ca. 10 Minuten leicht kochen lassen, bis die Sauce glatt und dicklich ist. Mit Salz, dem frisch gemahlenen Pfeffer und den Gewürzen abschmecken. Zum Schluss die Dickmilch dazurühren. Den Schinken fein würfeln und untermischen. Den Topf vom Herd nehmen, den Käse einstreuen und unter Rühren schmelzen lassen. Den Blumenkohl im Ganzen in einen feuerfesten Topf setzen, die Sauce darüber verteilen und das Ganze im vorgeheizten Backofen bei 200 Grad Celsius leicht überbacken. Die Petersilie waschen, trockentupfen und fein schneiden. Den Blumenkohl vor den Servieren mit der Petersilie bestreuen.


{\bfseries Menge:} 4 Portionen

{\bfseries Quelle:} Besser essen: Gemüse 

} 

%----------------------
\nopagebreak{ 
\subsection{Gefüllte Kohlrabi I}

\index{Kohlrabi@{Kohlrabi\/}!Gefüllte Kohlrabi I@{Gefüllte Kohlrabi I\/}}
\index{Schinken@{Schinken\/}!Gefüllte Kohlrabi I@{Gefüllte Kohlrabi I\/}}
}{
\begin{multicols}{2}



4 mittelgroße Kohlrabi

500 g gekochter Schinken

2 Eier

1 Bund Petersilie

2 Zwiebeln

1 Knoblauchzehe

50 g Semmelbrösel

Salz

schwarzer Pfeffer

2 EL Butter

125 ml Brühe

200 g Crème fraîche

2 EL Ketchup


\end{multicols}

Die Kohlrabi schälen, etwa 10 Minuten in wenig Salzwasser halbgar dünsten. Einen Deckel abschneiden und mit einem Esslöffel aushöhlen. Das herausgeholte Kohlrabifleisch in Würfel schneiden, dann mit dem in kleine Würfel geschnittenen Schinken und den Eiern vermischen. Die Petersilie waschen, trocken tupfen und fein schneiden. Das zarte Kohlrabigrün kleinschneiden. Die Zwiebeln und die Knoblauchzehe schälen und fein hacken. Alle vorbereiteten Zutaten mit den Semmelbröseln, Salz und frisch gemahlenem Pfeffer zum Fleischteig geben und alles durchkneten. Die Kohlrabi mit dem Fleischteig füllen und den Deckel wieder aufsetzen. Die Kohlrabi in einen Topf setzen, die Brühe dazugießen und im geschlossenen Topf ca. 30 Minuten garen. Die Sauce mit Crème fraîche und Ketchup abrunden. Dazu schmecken neue Kartoffeln oder körnig gekochter Reis.


{\bfseries Menge:} 4 Portionen

{\bfseries Quelle:} Besser essen: Gemüse 

} 

%----------------------
\nopagebreak{ 
\subsection{Gefüllte Kohlrabi II}

\index{Hackfleisch@{Hackfleisch\/}!Gefüllte Kohlrabi II@{Gefüllte Kohlrabi II\/}}
\index{Käse@{Käse\/}!Gefüllte Kohlrabi II@{Gefüllte Kohlrabi II\/}}
\index{Kohlrabi@{Kohlrabi\/}!Gefüllte Kohlrabi II@{Gefüllte Kohlrabi II\/}}
\index{überbacken@{überbacken\/}!Gefüllte Kohlrabi II@{Gefüllte Kohlrabi II\/}}
}{
\begin{multicols}{2}



4 Kohlrabi (je 250 g)

300 g Thüringer Zwiebelmett

200 g Schlagsahne

30 g Butter

1 Zwiebel

100 g mittelalter Gouda

weißer Pfeffer


\end{multicols}

Die Kohlrabi schälen, etwa 15 Minuten in wenig Salzwasser vorgaren. Einen Deckel abschneiden und mit dem Kugelausstecher aushöhlen. Das Kohlrabiinnere und die geschälte Zwiebeln würfeln. Butter in einer einer ofenfesten Pfanne erhitzen und das gewürfelte Gemüse samt gehackten Blättchen darin anschwitzen. Salzen, pfeffern und die Sahne dazugießen. Käse grob reiben und jeweils gut 1 TL davon in die ausgehöhlten Kohlrabiknollen geben. Diese mit Mett füllen, nebeneinander in das Gemüse setzen und etwa 30 Minuten zugedeckt bei sanfter Hitze garen. Dann dick mit Käse überstreuen und diesen unter den heißen Grill zerlaufen, jedoch nicht bräunen lassen.


{\bfseries Menge:} 4 Portionen

{\bfseries Quelle:} Gemüse -- Jung, frisch, unwiderstehlich 

} 

%----------------------
\nopagebreak{ 
\subsection{Gefüllter Blumenkohl mit Käsehaube}

\index{Blumenkohl@{Blumenkohl\/}!Gefüllter Blumenkohl mit Käsehaube@{Gefüllter Blumenkohl mit Käsehaube\/}}
\index{Hackfleisch@{Hackfleisch\/}!Gefüllter Blumenkohl mit Käsehaube@{Gefüllter Blumenkohl mit Käsehaube\/}}
\index{Käse@{Käse\/}!Gefüllter Blumenkohl mit Käsehaube@{Gefüllter Blumenkohl mit Käsehaube\/}}
\index{überbacken@{überbacken\/}!Gefüllter Blumenkohl mit Käsehaube@{Gefüllter Blumenkohl mit Käsehaube\/}}
}{
\begin{multicols}{2}



1 Blumenkohl (750 g)

Salz

375 g gemischtes Hackfleisch

1 Ei

3 EL Semmelbrösel

2 EL gehackte Kräuter

1 TL Paprikapulver

Zwiebelpulver

Salz

1 EL Pflanzenmargarine

\textit{Sauce}



30 g Pflanzenmargarine

30 g Mehl

250 ml Milch

2 Ecken Schmelzkäse

Salz

4 Tomaten


\end{multicols}

Den Blumenkohl im Salzwasser 10 Minuten vorgaren und abtropfen lassen. Inzwischen das Hackfleisch, das Ei, die Semmelbrösel, die Kräuter, das Paprikapulver, Zwiebelpulver und Salz vermischen. Den Blumenkohl von der Unterseite damit füllen. Eine flache Auflaufform ausfetten, den Blumenkohl mit den Röschen nach oben hineinlegen und im vorgeheizten Backofen bei 200 Grad Celsius etwas 20 Minuten backen. Aus Pflanzenmargarine, Mehl, Milch und 250 ml von dem Blumenkohlwasser eine helle Grundsauce zubereiten. Den Schmelzkäse darin auflösen und die Sauce mit Salz abschmecken. Den Blumenkohl damit übergießen. Die über Kreuz eingeschnittenen Tomaten dazusetzen und weitere 20 Minuten backen.


{\bfseries Menge:} 4 Portionen

{\bfseries Quelle:} Besser essen: Gemüse 

} 

%----------------------
\nopagebreak{ 
\subsection{Mischpilz-Ragout mit Semmelknödeln}

\index{Brot@{Brot\/}!Mischpilz-Ragout mit Semmelknödeln@{Mischpilz-Ragout mit Semmelknödeln\/}}
\index{Petersilie@{Petersilie\/}!Mischpilz-Ragout mit Semmelknödeln@{Mischpilz-Ragout mit Semmelknödeln\/}}
\index{Pilze@{Pilze\/}!Mischpilz-Ragout mit Semmelknödeln@{Mischpilz-Ragout mit Semmelknödeln\/}}
\index{vegetarisch@{vegetarisch\/}!Mischpilz-Ragout mit Semmelknödeln@{Mischpilz-Ragout mit Semmelknödeln\/}}
}{
\begin{multicols}{2}

\textit{Pilzragout}



1 kg Pilze

2 Zwiebeln (nach Größe)

3 Knoblauchzehen (nach Geschmack)

2 EL Butter

Salz

Pfeffer

200 g Schlagsahne

1 Bund Petersilie

Zitronensaft

\textit{Semmelknödel}



6 (250 g) altbackene Semmeln (vom Vortag)

125 ml Milch

75 g Butter

3 Eier

1 Bund Petersilie

Salz

Pfeffer

Muskat


\end{multicols}

Hierin findet alles Verwendung, was die Pilzsuche erbracht hat. Die Semmelknödeln kann man auch auf Vorrat zubereiten und tiefkühlen.

Die Pilze putzen und klein schneiden. Zwiebeln und Knoblauch fein würfeln.

Butter in einer großen Pfanne erhitzen. Zuerst die Zwiebel und Knoblauch darin andünsten, dann die Pilze zufügen und so lange dünsten, bis etwa die Hälfte der ausgetretenen Flüssigkeit verkocht ist.

Salzen und pfeffern. Die Sahne angießen und schließlich die feingehackte Petersilie unterrühren. Mit einem Spritzer Zitronensaft abschmecken.

Für die Knödel die Semmeln in gleichmäßige kleine Würfel schneiden. Ein Drittel davon in einer Schüssel mit der warmen Milch beträufeln und einweichen. Den Rest in der heißen Butter langsam golden rösten.

Eingeweichtes Brot, geröstete Würfel und Eier miteinander mischen, gehackte Petersilie zufügen, die Masse mit Salz, Pfeffer und Muskat würzen.

Mit angefeuchteten Händen tennisballgroße Knödel daraus formen. In leicht siedendem Salzwasser etwa 20 bis 25 Minuten sanft garziehen lassen. Garprobe: Ein Holzstäbchen, das Sie hineinstechen, sollte sich überall gleichmäßig warm
anfühlen.

Das Pilzragout in tiefen Tellern anrichten, jeweils einen Knödel in die Mitte setzen.


{\bfseries Menge:} 4 Portionen

{\bfseries Quelle:} www.rezepte.li 

} 

%----------------------
\nopagebreak{ 
\subsection{Weimarer Zwiebelkuchen}

\index{Hefeteig@{Hefeteig\/}!Weimarer Zwiebelkuchen@{Weimarer Zwiebelkuchen\/}}
\index{vegetarisch@{vegetarisch\/}!Weimarer Zwiebelkuchen@{Weimarer Zwiebelkuchen\/}}
\index{Zwiebeln@{Zwiebeln\/}!Weimarer Zwiebelkuchen@{Weimarer Zwiebelkuchen\/}}
}{
\begin{multicols}{2}



1 Hefeteig von 500 g Mehl

750 g Zwiebeln

100 g Bauchspeck

1/4 l saure Sahne

4 Eier

Salz

Pfeffer

Paprika

Kümmel


\end{multicols}

Den Hefeteig ausrollen, ein gefettetes Kuchenblech damit belegen und nochmals gehen lassen. Den Speck in dünne Scheiben schneiden, auslassen, die Speckscheiben aus der Pfanne nehmen und auf dem Hefeteig verteilen. Die Zwiebeln in feine Scheiben schneiden, im ausgelassenen Speckfett glasig dünsten (nicht bräunen), über die Speckscheiben verteilen, leicht salzen, mit Paprika und etwas gemahlenem Kümmel würzen. Die Sahne mit den Eiern gut verquirlen, mit Salz, Pfeffer und Paprika würzen und über die Zwiebeln gießen. Bei guter Mittelhitze backen und noch warm servieren. Als vegetarische Variante auch ohne Speck möglich.


{\bfseries Menge:} 8 Portionen

{\bfseries Quelle:} Von Apfelkartoffeln bis Zwiebelkuchen 

} 

%----------------------
\nopagebreak{ 
\subsection{Zwiebel-Hokkaido-Kuchen - groß und mit Speck}

\index{Käse@{Käse\/}!Zwiebel-Hokkaido-Kuchen - groß und mit Speck@{Zwiebel-Hokkaido-Kuchen - groß und mit Speck\/}}
\index{Kürbis@{Kürbis\/}!Zwiebel-Hokkaido-Kuchen - groß und mit Speck@{Zwiebel-Hokkaido-Kuchen - groß und mit Speck\/}}
\index{Zwiebeln@{Zwiebeln\/}!Zwiebel-Hokkaido-Kuchen - groß und mit Speck@{Zwiebel-Hokkaido-Kuchen - groß und mit Speck\/}}
}{
\begin{multicols}{2}

\textit{Teig}



500 g Mehl

5 EL Olivenöl

300 ml Wasser

1 Päckchen Trockenhefe

1 EL Salz

\textit{Belag}



Öl

200 g Schinkenspeckwürfel

800 g Zwiebeln

800 g Kürbis

400 ml saure Sahne

4 Eier

Salz

Pfeffer

200 g geriebenener Emmentaler


\end{multicols}

Olivenöl in einem Messbecher mit Wasser auf 300 ml auffüllen. Mehl, Salz und Hefe vermischen. Flüssigkeit langsam beim Kneten (Mixer) zugeben. Dann gehen lassen.
Teig auf einem Blech ausrollen, an den Kanten etwas hoch ziehen.

Die Zwiebeln schälen und kleinschneiden; den Kürbis kleinschneiden und evtl. raspeln.

Öl in einer Pfanne erhitzen und die SChinkenwürfel und die Zwiebeln ca. 20 Minuten andünsten, dann den Kürbis hinzufügen und weitere 15 Minuten dünsten.

Die saure Sahne mit den Eiern und Gewürzen verrühren und über das Gemisch gießen, aufstocken lassen. Belag auf dem Teig verteilen.

Backzeit: Ca. 50 Minuten bei 175 Grad, 10 Minuten vor Ende der Backzeit den geriebenen Käse darüber verteilen


{\bfseries Menge:} 8 Portionen

{\bfseries Quelle:} www.kuerbis-company.de, abgewandelt von Gerald und Frauke 

} 

\pagebreak 
 
%----------------------------------------------------
\nopagebreak{ 
\section{GEFLUEGEL} 

\subsection{Chicken Tikka Masala}

\index{Geflügel@{Geflügel\/}!Chicken Tikka Masala@{Chicken Tikka Masala\/}}
\index{Indisch@{Indisch\/}!Chicken Tikka Masala@{Chicken Tikka Masala\/}}
}{
\begin{multicols}{2}

\textit{Chicken Tikka}



1/2 Zwiebel, grob gehackt

60 g Tomatenmark

1 TL Kreuzkümmelsamen

2 1/2 cm Ingwerwurzel, gehackt

3 EL Zitronensaft

2 Knoblauchzehen, gepresst

2 TL Chilipulver

750 g Hähnchenfleisch ohne Knochen

Salz und Pfeffer

1 EL gehackter frischer Koriander

\textit{Masala-Sauce:}



2 EL Ghee (reines Butterfett)

1 Zwiebel, in Scheiben geschnitten

1 EL schwarze Zwiebelsamen

3 Knoblauchzehen, gepresst

2 frische grüne Chilischoten, gehackt

200 g Tomaten aus der Dose

120 ml Naturjoghurt

120 ml Kokosmilch

1 EL gehackter frischer Koriander

1 EL gehackte frische Minze

2 EL Zitronen- oder Limonensaft

1/2 TL Garam Masala


\end{multicols}

Zwiebel, Tomatenmark, Kreuzkümmel, Ingwer, Zitronensaft, Knoblauch, Chilipulver, Salz und Pfeffer im Mixer pürieren und in eine Schüssel geben. Alternativ können Sie den Kreuzkümmel auch mit Stößel und Mörser mahlen und in eine Schüssel geben.
Zwiebel, Ingwer und Koriander fein gehackt unterrühren und anschließend Tomatenmark, Zitronensaft, Salz, Pfeffer, Knoblauch und Chilipulver dazugeben.
Das Hähnchenfleisch in 4 cm große Würfel schneiden. Das Fleisch unter die Gewürzmischung heben und 2 Stunden marinieren lassen.

Für die Masala-Sauce das Ghee in einem großen Topf erhitzen. Zwiebel hinein geben und bei mittlerer Hitze 5 Minuten dünsten. Zwiebelsamen, Knoblauch und Chilischoten dazugeben und kochen, bis sie ihr volles Aroma entfalten.
Tomaten, Joghurt und Kokosmilch unterrühren, aufkochen und 20 Minuten köcheln lassen.

In der Zwischenzeit das Hähnchen gleichmäßig auf 8 eingeölte Spieße verteilen und 15 Minuten unter dem vorgeheizten, sehr heißen Grill braten (dabei mehrfach wenden).
Das Hähnchen von den Spießen abziehen und in die Sauce geben. Frischen Koriander, Minze, Limonen- oder Zitronensaft und Garam Masala unterrühren. Mit Korianderblättern garnieren.

Tipp: Reichen Sie zu diesem sehr gehaltvollen Gericht verschiedene Beilagen (Reis,
indisches Fladenbrot), um die feurigen Aromen auszugleichen und zu neutralisieren.


{\bfseries Menge:} 4 Portionen

{\bfseries Quelle:} www.vox.de: voxtours 2005-05-10 

} 

%----------------------
\nopagebreak{ 
\subsection{Coq au vin (Burgund)}

\index{Alkohol@{Alkohol\/}!Coq au vin (Burgund)@{Coq au vin (Burgund)\/}}
\index{Geflügel@{Geflügel\/}!Coq au vin (Burgund)@{Coq au vin (Burgund)\/}}
\index{P6@{P6\/}!Coq au vin (Burgund)@{Coq au vin (Burgund)\/}}
\index{vorbereiten@{vorbereiten\/}!Coq au vin (Burgund)@{Coq au vin (Burgund)\/}}
}{
\begin{multicols}{2}

\textit{Zutaten}



2 Hähnchen

2 EL Öl

100 g geräucherter Speck

1 gestrichener EL Mehl

0.05 l Cognac

1/5 l Wasser

1 EL Tomatenpüree

Salz

frischer weißer Pfeffer

50 g Butter

80 g Perlzwiebeln

150 g Champignons

2 Knoblauchzehen

1 Bund Petersilie

\textit{Marinade}



2 EL Öl

1 Zwiebel

2 Schalotten

2 Knoblauchzehen

1 Flasche Burgunderrotwein

2 Nelken

einige Pfefferkörner

1 Bund Suppengrün

1 Möhre


\end{multicols}

Die küchenfertigen Hähnchen in Portionsstücke aufteilen, dabei alle fetten Teile entfernen und das Fleisch in eine große Schüssel geben.  Für die Marinade Zwiebel, Schalotten und Knoblauchzehen schälen und sehr fein hacken. Die geschälte Möhre in dünne Scheiben schneiden. Alle diese vorbereiteten Zutaten im heißen Öl gut durchdünsten, das geputzte und klein geschnittene Suppengrün zugeben, unter stetem Rühren kurz dämpfen, dann den Rotwein angießen. Nelken und Pfefferkörner beigeben und aufkochen. Etwas auskühlen lassen, über die Hähnchenstücke geben und zugedeckt mindestens 5 Stunden marinieren lassen. Dann das Fleisch aus der Marinade nehmen, mit Küchenpapier trockentupfen und die Flüssigkeit absieben. Das ganze Speckstück in einen Topf mit kaltem Wasser geben, zum Kochen bringen und 5 Minuten blanchieren. Den gut abgetropften Speck leicht ausdrücken und in Würfel schneiden. Das Öl in einem Schmortopf erhitzen und die Speckwürfel darin bei mittlerer Hitze goldbraun braten. Nun die Hähnchenteile zugeben und
auf allen Seiten kräftig anbraten. Das Mehl über das Fleisch stäuben und bei starker Hitze leicht bräunen. Den Cognac darüber gießen, flambieren und gleichzeitig die Fleischstücke wenden. Sobald die Flammen erloschen sind, die abgesiebte Marinade und das Wasser angießen sowie das Tomatenpüree einrühren. Aufkochen, mit Salz und frischem Pfeffer würzen und auf kleinem Feuer zugedeckt etwa 45 Minuten leise schmoren lassen. Unterdessen die Perlzwiebeln in 10 g Butter bei milder Hitze glasieren und nach 15 Minuten Garzeit den Hähnchen beigeben. Die geputzten, gewaschenen und halbierten Champignons in der Hälfte der verbleibenden Butter gut durchdünsten und eine Viertelstunde vor Beendigung der Garzeit zu dem Geflügel geben. Zum Servieren die Hähnchenteile mit den Speckwürfelchen, den Champignons und den Perlzwiebeln in einer Kasserolle oder auf einer Platte anrichten und warmstellen. Die Sauce nochmals absieben, entfetten, abschmecken und noch einmal aufkochen lassen. Das Geflügel mit der heißen Sauce übergießen,
mit gehackter Petersilie bestreuen und sofort servieren. Dazu reichen wir in Butter geschwenkte Nudeln oder Reis.


{\bfseries Menge:} 6 Portionen

} 

%----------------------
\nopagebreak{ 
\subsection{Enten--Curry}

\index{Curry@{Curry\/}!Enten--Curry@{Enten--Curry\/}}
\index{Ente@{Ente\/}!Enten--Curry@{Enten--Curry\/}}
\index{vorbereiten@{vorbereiten\/}!Enten--Curry@{Enten--Curry\/}}
}{
\begin{multicols}{2}



1 kg Entenbrust oder 1 ganze Ente (2 kg)

3 Zwiebeln

4 Knoblauchzehen, zerdrückt

3 frische rote Chilischoten, entkernt

125 ml Essig

1 EL geriebener Ingwer

1 EL Korianderpulver

3 TL Kreuzkümmelpulver

2 TL Kurkuma

1 TL Salz

1 1/2 EL Öl

1/2 TL Zucker

375 ml Hühnerbrühe


\end{multicols}

Ente in 1,5 cm breife Streifen oder, bei Verwendung einer ganzen Ente, in Stücke schneiden. Von überschüssigem Fett befreien. Eine halbe Zwiebel, Knoblauch, Chilischoten, Essig und Ingwer im Mixer zu einer glatten Masse verarbeiten. In eine Schüssel geben und mit Koriander, Kreuzkümmel, Kurkuma und Salz vermischen. Entenstückchen gründlich in der Gewürzmischung wenden oder mit dieser einreiben. Abdecken und mehrere Stunden oder über Nacht im Kühlschrank marinieren. Restliche Zwiebeln in Ringe schneiden. Öl in einem Topf erhitzen. Zwiebeln unter Rühren weich dünsten. Ente mit der Marinade dazugeben. Zucker und Hühnerbrühe in den Topf geben. Mit halb geschlossenem Deckel ca. 40 Minuten köcheln lassen, bis die Ente gar ist. Öl von der Oberfläche abschöpfen. Curry über Nacht in den Kühlschrank stellen, damit weiteres überschüssiges Fett an der Oberfläche erstarrt und abgenommen werden kann. Unmittelbar vor dem Servieren aufwärmen.

Resteverwertung: Bei Verwendung einer ganzen Ente kann man das überschüssige Fett zusammen mit Schweineschmalz, Äpfeln und Zwiebeln zu leckerem Entenschmalz verarbeiten. Dass Gerippe zusammen mit etwas Suppengrün ergibt noch eine Suppe.


{\bfseries Menge:} 4 Portionen

{\bfseries Quelle:} Anne Wilson: Klassische Curry--Gerichte 

} 

%----------------------
\nopagebreak{ 
\subsection{Entenkeule mit Lebkuchenkruste}

\index{Ente@{Ente\/}!Entenkeule mit Lebkuchenkruste@{Entenkeule mit Lebkuchenkruste\/}}
\index{Geflügel@{Geflügel\/}!Entenkeule mit Lebkuchenkruste@{Entenkeule mit Lebkuchenkruste\/}}
\index{Lebkuchen@{Lebkuchen\/}!Entenkeule mit Lebkuchenkruste@{Entenkeule mit Lebkuchenkruste\/}}
}{
\begin{multicols}{2}

\textit{Entenkeule:}



6 Entenkeulen

Salz, weißer Pfeffer

2 EL Olivenöl

Koriander

\textit{Kruste:}



100 g weiche Butter

120 g Paniermehl

1 TL Lebkuchengewürz

alternativ 120 g Lebkuchenbrösel

\textit{Dazu}



Couscous mit Dörrobst

Orangensauce


\end{multicols}

Den Backofen auf 180$^\circ$C vorheizen.

Die Entenkeulen abtrocknen, dann salzen und pfeffern. Öl in einem Bräter oder in einer heißen Pfanne erhitzen, die Keulen mit der Hautseite nach unten hineinlegen und anbraten, bis eine goldene Farbe erreicht ist. Dann wenden und auch von der anderen Seite anbraten. Den Bräter in den Ofen stellen und ca. 20 Minuten braten.

Inzwischen für die Kruste die Butter mit dem Paniermehl, einer Prise Salz und dem Lebkuchengewürz vermengen.

Die Keulen aus dem Ofen nehmen und diesen auf Oberhitze umstellen. Die Krustenmasse auf die Entenkeulen verteilen und diese auf die mittlere Schiene zurück in den Ofen geben. Dort die Kruste in 2 bis 3 Minuten bräunen lassen. Die fertigen Entenkeulen auf dem Couscous mit Dörrobst anrichten und mit der Orangensauce umgießen. Zum Schluss alles mit frischem Koriander garnieren.


{\bfseries Menge:} 6 Portionen

{\bfseries Quelle:} www.vox.de: Schmeckt nicht gibt's nicht 

} 

%----------------------
\nopagebreak{ 
\subsection{Geflügelragout mit Safran und Blumenkohl}

\index{Blumenkohl@{Blumenkohl\/}!Geflügelragout mit Safran und Blumenkohl@{Geflügelragout mit Safran und Blumenkohl\/}}
\index{Geflügel@{Geflügel\/}!Geflügelragout mit Safran und Blumenkohl@{Geflügelragout mit Safran und Blumenkohl\/}}
\index{Safran@{Safran\/}!Geflügelragout mit Safran und Blumenkohl@{Geflügelragout mit Safran und Blumenkohl\/}}
}{
\begin{multicols}{2}



1 kleiner Blumenkohl

400 g Hähnchenbrust / Putenschnitzel

100 g Erbsen (TK)

2 Zwiebeln

300 g saure Sahne

1 Glas Weißwein

Salz

Pfeffer

Safran

Muskat

Fett zum Andünsten


\end{multicols}

Blumenkohlröschen (klein) vom Strunk lösen, etwa 10 Minuten bissfest dünsten, in kaltem Wasser abschrecken. Den Strunk weichkochen und dann passieren. Die kleingehackten Zwiebeln in einem Topf in etwas Fett glasig dünsten. Den passierten Strunk, die saure Sahne und den Weißwein hinzugeben, verrühren und aufkochen lassen. 2 Messerspitzen Safranpulver in etwas Weißwein lösen und in die Sauce geben. Das Geflügelfleisch in ca. 2 cm große Würfel schneiden, in die Sauce geben und etwa 10 Minuten mitgaren. Erbsen und Blumenkohlröschen dazugeben. Gericht noch etwas ziehen lassen. Mit Salz, Pfeffer und Muskat abschmecken. Dazu passen Reis oder Salzkartoffeln.


{\bfseries Menge:} 4 Portionen

{\bfseries Quelle:} Internet 

} 

%----------------------
\nopagebreak{ 
\subsection{Geschmorte Ente in Rotwein}

\index{Alkohol@{Alkohol\/}!Geschmorte Ente in Rotwein@{Geschmorte Ente in Rotwein\/}}
\index{Ente@{Ente\/}!Geschmorte Ente in Rotwein@{Geschmorte Ente in Rotwein\/}}
\index{Wein@{Wein\/}!Geschmorte Ente in Rotwein@{Geschmorte Ente in Rotwein\/}}
}{
\begin{multicols}{2}



1 Flug- oder Bauernente

850 g geschälte Tomaten aus der Dose ohne Saft

30 g scharze Oliven mit Kern

1/2 Flasche Rotwein, Chianti

1/4 l Wasser

4 EL Balsamicoessig

1 Kräuterstrauß

(bestehend aus 1 Thymianstrauß, 1 Rosmarinzweig, 1/2 Möhre, 1 Lorbeerblatt)

Olivenöl

Salz

Pfeffer


\end{multicols}

Die Ente in 8 Teile zerlegen. 2 Keulen, 2 Flügel und die Brust mit Knochen in 4 Teile. Alles salzen und pfeffern und in Olivenöl in einem schweren Bräter rundherum goldbraun anbraten - sehr wichtig! Dann mit Rotwein, Wasser und Balsamicoessig ablöschen. Oliven, Kräuterstrauß und Tomaten dazugeben.

Alles bei geschlossenem Bräter im 200 Grad heißen Ofen 50 Minuten schmoren lassen. Falls zuviel Flüssigkeit während der Bratzeit verdunstet, einfach etwas Rotwein/Wasser beigeben.

Die Ente mit den Oliven und Tomaten servieren. Dazu passen Salzkartoffeln, Brot oder in Butter geschwenkte Bandnudeln.


{\bfseries Menge:} 4 Portionen

{\bfseries Quelle:} Das! Rezepte (NDR Fernsehen 1999) 

} 

%----------------------
\nopagebreak{ 
\subsection{Hühnchen mit Oliven, Aprikosen und Feigen}

\index{Früchte@{Früchte\/}!Hühnchen mit Oliven, Aprikosen und Feigen@{Hühnchen mit Oliven, Aprikosen und Feigen\/}}
\index{Geflügel@{Geflügel\/}!Hühnchen mit Oliven, Aprikosen und Feigen@{Hühnchen mit Oliven, Aprikosen und Feigen\/}}
\index{vorbereiten@{vorbereiten\/}!Hühnchen mit Oliven, Aprikosen und Feigen@{Hühnchen mit Oliven, Aprikosen und Feigen\/}}
}{
\begin{multicols}{2}

\textit{Für 6 Personen}



90 g getrocknete Feigen

90 g getrocknete Aprikosen

1/2 Becher schwarze entkernte Oliven

4 EL Rotweinessig

2 EL Olivenöl

3 zerdrückte Knoblauchzehen

1 1/2 TL getrocknete Thymianblätter

1 TL gemahlener Kreuzkümmel

1/2 TL gemahlener Ingwer

frischgemahlener schwarzer Pfeffer

1 1/2 kg kleine Hühnerteile

4 EL Rotwein oder Apfelsinensaft

1 EL brauner Zucker

feine Apfelsinenschalenstreifen oder frischer Thymian zur Garnieren


\end{multicols}

Obst, Oliven, Essig, Öl, Thymian und Gewürze in einer großen Glasschüssel vermengen.
Fett (und falls gewünscht, auch Haut) entfernen; Huhn in die Mischung geben.
Zudecken und über Nacht in den Kühlschrank stellen. Huhn gelegentlich wenden.
Mischung in einen großen flachen Bratentopf geben. Wein und Zucker mischen und über die Hühnerteile geben. Zudecken und bei mäßig niedriger Temperatur (160 Grad) 30 Minuten backen.
Abdecken und weiterbacken; dabei häufig mit dem Bratensaft begießen, bis das Hühnchen gar ist, etwa 45 bis 50 Minuten.
Hühnerteile mit dem Sieblöffel auf eine vorgewärmte Servierplatte heben. Feigen, Aprikosen und Oliven darauf verteilen, mit Apfelsinenschale bestreuen und mit frischem Thymian garnieren.


{\bfseries Menge:} 4 Portionen

{\bfseries Quelle:} Anne Wilson: Hähnchen auf vielerlei Art 

} 

%----------------------
\nopagebreak{ 
\subsection{Persisches Safranhuhn}

\index{Geflügel@{Geflügel\/}!Persisches Safranhuhn@{Persisches Safranhuhn\/}}
\index{Safran@{Safran\/}!Persisches Safranhuhn@{Persisches Safranhuhn\/}}
\index{vorbereiten@{vorbereiten\/}!Persisches Safranhuhn@{Persisches Safranhuhn\/}}
}{
\begin{multicols}{2}



1 großes Brathähnchen oder Hühnerstücke

6 Zitronen

1 Msp. gemahlener Safran

Olivenöl, etwas weniger als der Saft der 6 Zitronen

Salz

Pfeffer


\end{multicols}

Die Hähnchenteile kurz abspülen, mit Küchenkrepp gut trocknen und in eine flache Schüssel legen. Die Zitronen auspressen. Den Safran mt ganz wenig Zitronensaft anrühren, damit es keine Klümpchen gibt. Dann erst mit dem übrigen Saft vermischen.

Die Hähnchenteile salzen, pfeffern und mit dem Safran-Zitronensaft übergießen. Das Olivenöl darüberverteilen, bis die Stücke gut bedeckt sind. Im Kühlschrank abgedeckt 24 Stunden ziehen lassen, ab und zu umwenden.

Den Backofen auf 240 Grad vorheizen. Die Hühnerstücke aus der Marinade heben, auf eine Backblech legen und 45 Minuten im heißen Ofen backen, bis die Haut schön bräunlich ist.

Dazu passen Reis, bunter Salat und ein trockener Rotwein.


{\bfseries Menge:} 4 Portionen

{\bfseries Quelle:} Alfred Biolek: Die Rezepte meiner Gäste (Veronica Ferres) 

} 

%----------------------
\nopagebreak{ 
\subsection{Pute auf portugiesische Art}

\index{Pute@{Pute\/}!Pute auf portugiesische Art@{Pute auf portugiesische Art\/}}
\index{Reis@{Reis\/}!Pute auf portugiesische Art@{Pute auf portugiesische Art\/}}
\index{Tomaten@{Tomaten\/}!Pute auf portugiesische Art@{Pute auf portugiesische Art\/}}
\index{vorbereiten@{vorbereiten\/}!Pute auf portugiesische Art@{Pute auf portugiesische Art\/}}
}{
\begin{multicols}{2}



1 Pute

Salz

Pfeffer

250 g Reis

1/2 l Brühe

10 Tomaten

Olivenöl

2 Zwiebeln

etwas Butter

10 g Speisestärke


\end{multicols}

Reis in Brühe körnig ausquellen lassen. Kleingeschnittene Putenleber und 2 gewürfelte Tomaten in etwas Butter anbraten und unter den Reis heben. Die innen und außen mit Pfeffer und Salz eingeriebene Pute mit 2/3 der Reismischung füllen, zunähen. In der Bratenpfanne mit Öl bestreichen, gehackte Zwiebeln zufügen. Die Pute unter fleißigem Begießen in 2,5--4 Stunden hellbraun braten (evtl. mit Speckstreifen belegen, verhindern das Austrocknen). Acht ausgehöhlte, mit dem restlichen Reis gefüllte Tomaten gesondert einige Minuten in den heißen Ofen stellen. Deckel und Inneres der Tomaten in die Sauce geben. Pute aus dem Topf nehmen, Fäden entfernen, die Füllung in eine vorgewärmte Schüssel geben. Pute und Tomaten auf vorgewärmter Platte anrichten. Bratensatz passieren, mit angerührter Speisestärke binden und mit Pfeffer und Salz abschmecken. Dazu passen grüner Salat oder Endiviensalat, Salzkartoffeln oder Semmelknödel.


{\bfseries Menge:} 4 Portionen

} 

%----------------------
\nopagebreak{ 
\subsection{Würzige Hähnchenspieße}

\index{Appetizer@{Appetizer\/}!Würzige Hähnchenspieße@{Würzige Hähnchenspieße\/}}
\index{Geflügel@{Geflügel\/}!Würzige Hähnchenspieße@{Würzige Hähnchenspieße\/}}
\index{vorbereiten@{vorbereiten\/}!Würzige Hähnchenspieße@{Würzige Hähnchenspieße\/}}
}{
\begin{multicols}{2}



1 kg Hähnchenfilet

250 g Joghurt

1 TL Chilipulver

1 TL Kurkuma

1 TL Kreuzkümmel, gemahlen

1 TL Koriander, gemahlener

1 TL frisch geriebener Ingwer

1 Knoblauchzehe, zerdrückt

25 kleine Holzspieße

frische Kräuter nach Belieben


\end{multicols}

25 kleine Holzspieße 30 Minuten in kaltes Wasser legen, um ein Anbrennen während des Grillens zu verhindern. Hähnchenfilets von überschüssigem Fett und Sehnen befreien. Filets in kleine Stücke schneiden und vorsichtig auf die Spieße stecken, so dass diese zu ca. 3/4 der Länge bedeckt sind. Den Joghurt und die Gewürze in eine kleine Schüssel geben und gut mischen. Die Spieße in eine nicht-metallene flache Schüssel legen, mit der Joghurtsauce übergießen und mit Frischhaltefolie verschlossen mehrere Stunden oder über Nacht in den Kühlschrank stellen. Die Spieße gelegentlich wenden, damit das Hähnchen von allen Seiten gleichmäßig mariniert wird.

Hähnchenspieße auf einen heißen, leicht gefetteten Grillrost oder ein Grillblech geben. 8-10 Minuten backen, bis das Hähnchen zart und braun ist. Nach Wunsch mit frischen Kräutern garnieren.

Hinweis: Die Spieße erst kurz vor dem Servieren zubereiten.


{\bfseries Menge:} 4 Portionen

{\bfseries Quelle:} Anne Wilson: Ausgefallene Appetizer 

} 

\pagebreak 
 
%----------------------------------------------------
\nopagebreak{ 
\section{FLEISCH} 

\subsection{Chili con carne}

\index{einfach@{einfach\/}!Chili con carne@{Chili con carne\/}}
\index{Rindfleisch@{Rindfleisch\/}!Chili con carne@{Chili con carne\/}}
\index{vorbereiten@{vorbereiten\/}!Chili con carne@{Chili con carne\/}}
}{
\begin{multicols}{2}



500 g getrocknete Kidney-Bohnen

2 TL Salz

2 dicke Scheiben Schinken oder durchwachsener Speck

6 EL Pflanzenöl

3 kg Rinderhack

9 Zwiebeln, gehackt

4 Dosen Tomaten (a 400 g)

2 EL Salz

6 Knoblauchzehen, geschält und gehackt

1/2 Tasse Chili Powder (Chili-con-carne-Gewürzmischnung)

4 TL Korianderblätter oder Petersilie

2 TL Zucker

2 TL Tabasco

6 Chilischoten, gehackt

2 TL Zimt


\end{multicols}

Bohnen mit dem Schinken langsam in Salzwasser erhitzen, zum Kochen bringen.

Öl in einem großen Topf erhitzen, Fleisch und Zwiebeln langsam darin bräunen, alle anderen Zutaten zugeben und eine Stunde köcheln lassen. Bohnen hineinrühen und 30 Minuten köcheln, gelegentlich umrühren.

Wird mit Sour Cream, geriebenem Käse und Brot serviert.

Anmerkung: Chili Powder ist eine Gewürzmischung aus Chilischoten, Kreuzkümmel, Knoblauch, Oregano, meist auch Salz und manchmal Nelken oder Piment, vergleichbar mit der hier erhältlichen Chili-con-carne-Gewürzmischnung.


{\bfseries Menge:} 10 Portionen

{\bfseries Quelle:} The Collection. A Cookbook 

} 

%----------------------
\nopagebreak{ 
\subsection{Curry mit Lamm und Spinat}

\index{Curry@{Curry\/}!Curry mit Lamm und Spinat@{Curry mit Lamm und Spinat\/}}
\index{Lamm@{Lamm\/}!Curry mit Lamm und Spinat@{Curry mit Lamm und Spinat\/}}
\index{Spinat@{Spinat\/}!Curry mit Lamm und Spinat@{Curry mit Lamm und Spinat\/}}
\index{vorbereiten@{vorbereiten\/}!Curry mit Lamm und Spinat@{Curry mit Lamm und Spinat\/}}
}{
\begin{multicols}{2}



800 g mageres Lammfleisch aus der Keule (oder auch Rindfleisch)

1/2 TL geriebener Ingwer

1/2 TL zerdrückter Knoblauch

500 g tiefgefrorener gehackter Spinat (Rahmspinat)

4 EL Butter

1/2 TL Cayennepfeffer

2 TL Garam masala oder Curry

2 Kardamomkapseln, geschält und leicht zerdrückt

1 Zimtstange

Salz

Pfeffer

125 g Naturjoghurt

1/2 TL Muskat

1 EL gemahlene Mandeln

3 EL Crème fraîche


\end{multicols}

Lamm in 3 cm große Würfel schneiden und mit Ingwer und Knoblauch würzen. 20 Minuten ziehen lassen. In einen Topf geben, soviel Wasser dazugießen, dass das Fleisch bedeckt ist, und etwa 30 Minuten köcheln lassen. Grobgehackter Spinat sollte in einem Mixer püriert werden. Fleisch abtropfen lassen und die aufgefangene Flüssigkeit aufbewahren. Fleisch in Ghee oder Butter rasch anbraten. Gewürze und Joghurt hinzufügen und etwa 8 Minuten kochen, bis das Fleisch den Joghurt vollständig aufgenommen hat und sich im Topf kaum noch Flüssigkeit befindet. Spinat, gemahlene Mandeln, Muskat und etwa 200 ml des Bratensaftes zugeben. Gut verschlossen 7-8 Minuten köcheln lassen. Crème fraîche einrühren und nochmals vorsichtig erhitzen. Mit Chapatis oder Fladenbrot servieren.

Wird statt Lamm Rindfleisch verwendet, bitte längere Garzeiten beachten (2-3mal solange)! Gericht kann bis zur Zugabe des Spinates vorbereitet werden, der Spinat sollte erst kurz vor dem Servieren dazukommen, da sonst die grüne Farbe in braun übergeht.


{\bfseries Menge:} 4 Portionen

{\bfseries Quelle:} Anne Wilson: Indische Küche 

} 

%----------------------
\nopagebreak{ 
\subsection{Gefüllte Schweinelende mit Feigen und Portwein}

\index{Alkohol@{Alkohol\/}!Gefüllte Schweinelende mit Feigen und Portwein@{Gefüllte Schweinelende mit Feigen und Portwein\/}}
\index{Feigen@{Feigen\/}!Gefüllte Schweinelende mit Feigen und Portwein@{Gefüllte Schweinelende mit Feigen und Portwein\/}}
\index{P6@{P6\/}!Gefüllte Schweinelende mit Feigen und Portwein@{Gefüllte Schweinelende mit Feigen und Portwein\/}}
\index{Portwein@{Portwein\/}!Gefüllte Schweinelende mit Feigen und Portwein@{Gefüllte Schweinelende mit Feigen und Portwein\/}}
\index{Schweinefleisch@{Schweinefleisch\/}!Gefüllte Schweinelende mit Feigen und Portwein@{Gefüllte Schweinelende mit Feigen und Portwein\/}}
}{
\begin{multicols}{2}

\textit{für 6 Personen}



1 Schweinelende (oder aus der Oberschale), ca. 1 kg

Salz

Pfeffer

2 TL gemahlener Zimt

8 große frische Feigen (ca. 375 g)

1 EL Butter

1 EL Pflanzenöl

1/4 l dunkle Rinder- oder Kalbsbrühe (ggf. etwas mehr)

1/4 l Portwein

1 EL brauner Zucker

2 EL rotes Johannisbeergelee

Saft von 1/2 Zitrone

1 1/2 TL Stärkemehl in 2 EL Wasser angerührt

Faden zum Dressieren


\end{multicols}

Den Backofen auf 19$^\circ$C (Gas Stufe 2-3) vorheizen. Die Lende längs bis zur Hälfte einschneiden, so dass eine Tasche entsteht. Dabei die Enden nicht durchschneiden, damit die Füllung während des Bratens nicht herausfällt. Die Tasche und die Schnittfläche des Bratens mit Salz, Pfeffer und 1 TL Zimt würzen. 2 Feigen putzen und vierteln. Die Viertel in die Tasche drücken und den Braten in regelmäßigen Abständen dressieren. Mit Salz, Pfeffer und restlichem Zimt würzen.

Die Butter mit dem Öl in einem Bräter erhitzen, bis sie zu schäumen beginnt, und das Fleisch rund herum darin anbraten. Mit der Brühe begießen, in den vorgeheizten Ofen schieben und 1 bis 1 1/4 Stunden braten. Dabei gelegentlich umdrehen und begießen. Ist nicht mehr genug Flüssigkeit im Bräter, etwas Brühe nachgießen, damit das Fleisch saftig bleibt.

In der Zwischenzeit die restlichen Feigen glacieren. Die Früchte dazu oben kreuzförmig einschneiden und vorsichtig wie eine Blume auseinanderfalten. In eine kleine Auflaufform setzen, mit der Hälfte des Portweins übergießen und mit dem Zucker bestreuen. Zu dem Braten in den Ofen geben und 15 bis 20 Minuten backen, bis die Feigen weich und glaciert sind. Anschließend herausnehmen und warm stellen.

Den fertigen Braten auf ein Küchenbrett legen und zum Warmhalten mit Alufolie abdecken. Den restlichen Portwein, das Gelee und den Zitronensaft in den Bräter geben und aufkochen lassen. Dabei gut umrühren, damit sich der Bratensaft vom Boden löst. Mit dem Feigensaft auf etwa 375 ml auffüllen, gegebenfalls auch noch etwas Brühe hinzufügen. Das Ganze aufkochen lassen, die Sauce mit dem angerührten Stärkemehl andicken und abschmecken.

Die Fäden entfernen und den Braten in gut 1 cm dicke Scheiben schneiden. Diese schuppenförmig auf einer vorgewärmten Platte anrichten und mit etwas Sauce beträufeln. Die Feigen rund herum garnieren und die restliche Sauce getrennt dazu reichen.

Möglichst noch etwas feste Feigen verwenden, dann garen sie schön und geben reichlich Saft an die Sauce ab. Dazu passt ein Pilaw mit Wildreis oder Bulgur oder auch Gnocchi alla romana.


{\bfseries Menge:} 6 Portionen

{\bfseries Quelle:} Anne Willan: Auf den Punkt 

} 

%----------------------
\nopagebreak{ 
\subsection{Hasenpfeffer}

\index{Kaninchen@{Kaninchen\/}!Hasenpfeffer@{Hasenpfeffer\/}}
\index{vorbereiten@{vorbereiten\/}!Hasenpfeffer@{Hasenpfeffer\/}}
\index{Wild@{Wild\/}!Hasenpfeffer@{Hasenpfeffer\/}}
}{
\begin{multicols}{2}



1 Hase oder Kaninchen in Teile zerlegt

1 große Zwiebel (oder 2 mittlere)

6 Nelken

1 EL Pfefferkörner

1 TL Senfkörner

1 getrocknete Chilischote

1 Lorbeerblätter

etwa 2 l Wasser

Essig

1 Würfel Rinderbrühe (1l)

Salz

Pfeffer

Saucenbinder

Sahne


\end{multicols}

Nelken in die geschälte Zwiebel stecken. Pfefferkörner, Senfkörner, Chili und Lorbeerblätter in eine Schüssel geben. Wasser und soviel Essig dazu geben, dass die Flüssigkeit leicht säuerlich schmeckt. Fleisch einlegen (es muss bedeckt sein), Schüssel zugedeckt drei Tage im Kühlschrank aufbewahren.

Hasenteile mit Küchenkrepp gut abtrocknen, leicht salzen und pfeffern. Essigwasser absieben und auffangen. Das Fleisch in einem Bräter in heißem Öl anbraten, mit Essigwasser ablöschen, Brühwürfel und alle Gewürze zugeben. Den Hasen bei mittlerer Hitze zugedeckt etwa 50 Minuten braten, verdampfte Flüssigkeit ergänzen (Essigwasser, falls es zu sauer wird, nur Wasser nehmen).

Das Fleisch herausnehmen und warmstellen. Sud durchpassieren. Sauce mit braunem Saucenbinden binden, mit Sahne verfeinern und evtl. nachwürzen.

Dazu schmecken Spaghetti und Salat.


{\bfseries Menge:} 4 Portionen

{\bfseries Quelle:} Georg und Ina Zimmermann 

} 

%----------------------
\nopagebreak{ 
\subsection{Hirschgulasch}

\index{Pilze@{Pilze\/}!Hirschgulasch@{Hirschgulasch\/}}
\index{vorbereiten@{vorbereiten\/}!Hirschgulasch@{Hirschgulasch\/}}
\index{Wild@{Wild\/}!Hirschgulasch@{Hirschgulasch\/}}
}{
\begin{multicols}{2}



500 g Hirschfleisch

1 EL Öl

Salz

Pfeffer

1/2 TL Ingwerpulver

1/2 TL Kümmel

1/4 TL Majoran

1/2 TL Koriander

1 Spur Zimt

2 Zwiebeln

1 zerdrückte Knoblauchzehe

1 EL Mehl

Paprikapulver

2 TL Tomatenmark

etwas Fleischbrühe

150 g Pilze

1/4 l saure Sahne

80 g Margarine

1 Handvoll Trockenpilze


\end{multicols}

Trockenpilze einweichen. Das gewürfelte Fleisch mit den in Öl verrührten Gewürzen gut vermischen. 1 Stunde ziehen lassen. In der erhitzten Margarine die Zwiebelwürfel goldgelb braten, den Knoblauch zugeben und kurz durchdünsten. Das Fleisch und die eingeweichten Trockenpilze zugeben und zugedeckt im eigenen Saft garen (insgesamt ca. 40 Minuten). Nach der Hälfte der Garzeit die geputzten und zerkleinerten frischen Pilze dazugeben. Wenn alles gar ist, mit Mehl und Paprika bestäuben und Tomatenmark sowie etwas Brühe angießen. Zuletzt die saure Sahne zugeben und alles durchdünsten. Nochmals abschmecken. Dazu passen Salzkartoffeln, Apfelrotkohl und Preißelbeeren.


{\bfseries Menge:} 4 Portionen

{\bfseries Quelle:} Kochen, mit Veränderungen von Gerald und Frauke 

} 

%----------------------
\nopagebreak{ 
\subsection{Indonesisches Rindfleisch (Rendang)}

\index{asiatisch@{asiatisch\/}!Indonesisches Rindfleisch (Rendang)@{Indonesisches Rindfleisch (Rendang)\/}}
\index{Curry@{Curry\/}!Indonesisches Rindfleisch (Rendang)@{Indonesisches Rindfleisch (Rendang)\/}}
\index{Rindfleisch@{Rindfleisch\/}!Indonesisches Rindfleisch (Rendang)@{Indonesisches Rindfleisch (Rendang)\/}}
\index{vorbereiten@{vorbereiten\/}!Indonesisches Rindfleisch (Rendang)@{Indonesisches Rindfleisch (Rendang)\/}}
}{
\begin{multicols}{2}



1 kg Rindfleisch

2 Zwiebeln

4 Knoblauchzehen

5 rote Chilischoten

1 EL geriebener Ingwer

500 ml Kokosmilch

1 1/2 EL Öl

1 EL Korianderpulver

1 EL Kreuzkümmelpulver

1 TL Kurkuma

1 TL Zimtpulver

1/4 TL Gewürznelkenpulver

1/4 TL Chilipulver

1 großer Streifen Zitronenschale

1 1/2 EL Zitronensaft

1 EL brauner Zucker

1 TL Tamarindenkonzentrat


\end{multicols}

Fleisch von Fett und Sehnen befreien. In ca. 2,5 cm große Würfel schneiden, Zwiebel hacken, Knoblauch zerdrücken. Zwiebeln, Knoblauch, Chilis, Ingwer und 2 1/2 EL Kokosmilch im Mixer zu einer Paste verarbeiten. Öl in einem Topf erhitzen. Zwiebelpaste, Koriander, Kurkuma, Zimt, Nelkenpulver, Chilipulver, Zitronenschale, Fleischwürfel und restliche Kokosmilch hineingeben. Gründlich vermischen. Aufkochen lassen. Bei schwacher Hitze 1,5 Stunden köcheln, bis das Fleisch gar und die Flüssigkeit eingekocht ist. Wenn das Öl beginnt, sich an der Oberfläche abzusetzen, Zitronensaft, Zucker und Tamarindenkonzentrat dazugeben und unter Rühren mit erwärmen. Mit Reis servieren.

Hinweis: Dies ist ein''trockenes'' Curry, bei dem kaum Soßenflüssigkeit übrig bleibt. Dafür absorbiert das Fleisch das Gewürzaroma während des Schmorens besonders gut. Noch besser schmeckt das Curry, wenn man es über Nacht stehen läßt. Abkühlen lassen und in einem abgedeckten Topf in den Kühlschrank stellen.


{\bfseries Menge:} 4 Portionen

{\bfseries Quelle:} Anne Wilson: Klassische Currygerichte 

} 

%----------------------
\nopagebreak{ 
\subsection{Italienischer Kräuterbraten}

\index{italienisch@{italienisch\/}!Italienischer Kräuterbraten@{Italienischer Kräuterbraten\/}}
\index{Schweinefleisch@{Schweinefleisch\/}!Italienischer Kräuterbraten@{Italienischer Kräuterbraten\/}}
}{
\begin{multicols}{2}



2 Knoblauchzehen

2 EL Erdnüsse

5 EL italienische Kräuter

6 TL Olivenöl

Salz

Pfeffer

750 g Schweinebraten, mager

150 ml Gemüsebrühe


\end{multicols}

Knoblauchzehen zerdrücken, Erdnüsse hacken, mit Kräutern und 2 TL Olivenöl mischen. Mit Salz und Pfeffer würzen.

Fleisch aufschneiden, so dass eine flache Scheibe entsteht. Kräutermischung darauf verteilen, zusammenrollen und mit Küchengarn umwickeln. Wenn möglich etwas durchziehen lassen.

Restliches Öl erhitzen und Braten darin rundherum anbraten.

Gemüsebrühe angießen und zugedeckt bei schwacher Hitze ca. 90 min schmoren lassen.

Dazu passen provenzalisches geröstetes Gemüse und Polenta.


{\bfseries Menge:} 4 Portionen

{\bfseries Quelle:} Weihnachtsmeü Siemens München 2005 

} 

%----------------------
\nopagebreak{ 
\subsection{Kalbsbraten auf Quitten}

\index{Alkohol@{Alkohol\/}!Kalbsbraten auf Quitten@{Kalbsbraten auf Quitten\/}}
\index{Kalbfleisch@{Kalbfleisch\/}!Kalbsbraten auf Quitten@{Kalbsbraten auf Quitten\/}}
\index{Lauch@{Lauch\/}!Kalbsbraten auf Quitten@{Kalbsbraten auf Quitten\/}}
\index{Quiiten@{Quiiten\/}!Kalbsbraten auf Quitten@{Kalbsbraten auf Quitten\/}}
}{
\begin{multicols}{2}



3 Stangen Lauch

2 Quitten (a 300 g)

2 EL Zitronensaft

900 g Kalbfleisch (am Stück, evtl aus der Nuss)

Salz

Pfeffer
'' EL Öl

350 ml Weißwein

5 Stiele Majoran

3 EL Quittengelee


\end{multicols}

Lauchstangen putzen, waschen und in 5cm lange Stücke schneiden. Quitten schälen, vierteln, entkernen und sofort in Zitronensaft wenden.

Das Fleisch salzen und pfeffern. Öl in einem Bräter erhitzen, das Fleisch darin von allen Seiten kräftig anbraten und herausnehmen. Lauch und Quitten in den Bräter geben und kurz anbraten. Weißwein und 3 Majoranstiele zugeben und aufkochen.

Fleisch wieder in den Bräter geben und zugedeckt im vorgeheizten Ofen bei 190$^\circ$C auf der zweiten Schiene von unten 70 min schmoren (Umluft nicht emfehlenswert). Nach 35 min einmal wenden.

Am Ende der Garzeit das Gelee unter den Bratenfond rühren und alles mit Pfeffer und Salz würzen. Mit restlichem abgezupften Majoran garnieren.

Dazu passt z.B. Graupen-Risotto.

Tipp: Man kann das Kalbfleisch auch durch mageres Schweinefleisch ersetzen.


{\bfseries Bemerkung:} Eiweiß: 49 g, Fett: 9 g, Kohlenhydrate: 22 g, kJ: 1672, kcal: 398 

{\bfseries Menge:} 4 Portionen

{\bfseries Quelle:} essen und trinken für jeden Tag 10/2005 

} 

%----------------------
\nopagebreak{ 
\subsection{Lammkeule mit Rotwein}

\index{Lamm@{Lamm\/}!Lammkeule mit Rotwein@{Lammkeule mit Rotwein\/}}
\index{vorbereiten@{vorbereiten\/}!Lammkeule mit Rotwein@{Lammkeule mit Rotwein\/}}
\index{Wein@{Wein\/}!Lammkeule mit Rotwein@{Lammkeule mit Rotwein\/}}
}{
\begin{multicols}{2}



1 Lammkeule, 1.2kg mit Knochen oder 800g ohne Knochen

1 Stange Porree

2 Möhren

2 rote Zwiebeln

1 kleines Stück Sellerie

3 Knoblauchzehen

1 Petersilienwurzel

je  1/2 Bund Petersilie, Rosmarin, Majoran, Thymian, Salbei

(erstzweise getrocknete Kräuter)

1/2 Flasche trockener Rotwein

Olivenöl

Salz

Pfeffer

etwas Lamm- und Kalbsfond

Vermouth

Cognac


\end{multicols}

Die Lammkeule trockenwischen. Porree, Möhren, Zwiebeln, Sellerie und Knoblauch putzen und grob zerteilen. Die frischen Kräuter waschen und trocknen. Die Kräuter mit dem zerteilten Wurzelgemüse und 1/2-1 Flasche Rotwein in einem großen Gefrierbeutel mischen, die Lammkeule dazu legen und 24 oder länger im Kühlschrank marinieren, dabei öfter wenden.

Die Keule aus dem Beutel nehmen, mit Küchenkrepp trocknen und die Marinade durch ein Sieb gießen. In einem Bräter Olivenöl erhitzen und die Lammkeule scharf rundum anbraten, dann salzen, pfeffern und auf den Rost des Backofens legen. Darunter eine Saftpfanne stellen, um den Saft aufzufangen. Bei 80 Grad gut 3, besser 4 Stunden braten.

Die durchgesiebte Marinade mit etwas Fond oder Brühe in einem Topf stark einkochen lassen. Die Lammkeule aus dem Ofen nehmen und fest in Alufolie einwickeln. Mindestens 10 Minuten ruhen lassen, damit sich der Fleischsaft verteilen kann. Den Bratensaft aus der Saftpfanne zu der Sauce geben, mit Salz, Pfeffer, Vermouth und Cognac abschmecken. Das Fleisch in Scheiben aufschneiden und mit der Sauce servieren. Dazu passen weiße und grüne Bohnen und ein sahniges Kartoffelpüree. Und ein guter Rotwein.


{\bfseries Menge:} 4 Portionen

{\bfseries Quelle:} Alfred Biolek: Die Rezepte meiner Gäste (Hellmuth Karasek) 

} 

%----------------------
\nopagebreak{ 
\subsection{Pilz--Rollbraten (polnisch)}

\index{Pilze@{Pilze\/}!Pilz--Rollbraten (polnisch)@{Pilz--Rollbraten (polnisch)\/}}
\index{polnisch@{polnisch\/}!Pilz--Rollbraten (polnisch)@{Pilz--Rollbraten (polnisch)\/}}
\index{Rindfleisch@{Rindfleisch\/}!Pilz--Rollbraten (polnisch)@{Pilz--Rollbraten (polnisch)\/}}
}{
\begin{multicols}{2}



800 g Rinderrollbraten

(oder große Rouladen)

Salz

Pfeffer

Paprika edelsüß

2 EL Senf

2 große Zwiebeln

500 g gemischte Pilze

60 g Speck

2 EL Schmalz

5 EL Sahne

2 Bund Petersilie

2 EL Semmelbrösel

2 Eier

1/4 l heiße Fleischbrühe

30 g Butter


\end{multicols}

Fleisch flachdrücken, mit Salz, Pfeffer und Paprika bestreuen und mit Senf bestreichen. Zwiebeln und Pilze fein hacken. Speck würfeln und in heißer Butter ausbraten. Zwiebeln und Pilze zufügen, 10 Minuten braten, vom Herd nehmen und gehackte Petersilie mit Bröseln und Eiern einrühren. Würzen und auf das Fleisch streichen. Zusammenrollen (restliche Farce in die Sauce geben), mit Küchengarn umwickeln und in heißem Schmalz sehr stark anbraten. Brühe angießen und zugedeckt etwa 70 Minuten garen. Sahne in die Sauce rühren, getrennt reichen.


{\bfseries Menge:} 4 Portionen

} 

%----------------------
\nopagebreak{ 
\subsection{Rinderrouladen mit getrockneten Tomaten}

}{
\begin{multicols}{2}



50 g Pinienkerne

50 g getrocknete weiche Tomaten

2 Knoblauchzehen

7 EL Olivenöl

6 Rinderrouladen à 150 g

6 Scheibe(n) Pancetta

Salz

Pfeffer

1 Bund Basilikum

1 weiße Zwiebel

100 g Sellerie

4 Tomaten

3 Stiele Thymian

1 TL Fenchelsaat

100 ml Weißwein

350 ml Kalbsbrühe oder Rinderbrühe

1/2 Kartoffel (ca. 100 g)

5 Stiele glatte Petersilie

\bild{image/cookbook_10.jpg}


\end{multicols}

Pinienkerne in einer Pfanne ohne Fett rösten, dann abkühlen lassen. Die getrockneten Tomaten mit Knoblauch, der Hälfte der Pinienkerne und 3 Löffeln Olivenöl im Mixer zu einer noch leicht stückigen Paste verarbeiten.

Rouladen nebeneinander legen und mit ein wenig Salz, Pfeffer würzen. Mit der Tomatenpaste bestreichen, mit je 1 Scheibe Pancetta und 2 Basilikumblättern belegen. Rouladen aufrollen und mit Nadeln zusammen stecken.

Zwiebeln würfeln, Staudensellerie in 0,5 cm breite Scheiben schneiden. Rouladen von außen mit Salz und Pfeffer würzen. Rouladen in einem geeigneten Topf oder Bräter in 4 El Olivenöl hellbraun anbraten. Tomaten halbieren und den Strunk entfernen. Rouladen herausnehmen, Tomaten mit der Schnittfläche nach unten in den Topf setzen und mit Fenchelsaat zusammen anschwitzen, anschließend das Gemüse im Topf glasig, ohne Farbe dünsten. Thymian zugeben. Mit Weißwein ablöschen und mit 350 ml Kalbsbrühe oder Rinderbrühe auffüllen. Kartoffel schälen und in den Sud reiben (dient zur natürlichen, leichten Bindung). Rouladen zugeben und das ganze mit halbgeöffnetem Deckel im Backofen bei 160 - 170 Grad auf der untersten Schiene 2 bis 2 1/2 Stunden garen. Zum Schluss fein gehackte Petersilie unterrühren.

Rouladen mit dem Sud auf Tellern anrichten und mit den restlichen Pinienkernen und Basilikumblättern servieren. Dazu gibt es Baguette oder anderes Brot.

Abwandlung: eine oder mehrere Auberginen mit zum Gemüse nehmen macht eine schöne Sauce.


{\bfseries Menge:} 6 Personen

{\bfseries Zeit:} Gesamtzeit 180 min

{\bfseries Quelle:} daserste.de, Tim Mälzer kocht!, 29.8.2009 

} 

%----------------------
\nopagebreak{ 
\subsection{Rindfleisch mit Orangenstreifen}

\index{asiatisch@{asiatisch\/}!Rindfleisch mit Orangenstreifen@{Rindfleisch mit Orangenstreifen\/}}
\index{Rindfleisch@{Rindfleisch\/}!Rindfleisch mit Orangenstreifen@{Rindfleisch mit Orangenstreifen\/}}
\index{Wok@{Wok\/}!Rindfleisch mit Orangenstreifen@{Rindfleisch mit Orangenstreifen\/}}
}{
\begin{multicols}{2}



400 g Rumpsteak

2 TL Orangenschale; gerieben

2 TL Sesamöl

200 ml Orangensaft

1 1/2 TL Maisstärke

2 EL Erdnußöl

1 klein. Zwiebel; in großen Stücken

2 klein. Möhren; in schmalen Streifen

150 g Stangenbohnen; in 5 cm großen Stücken

1 EL Sesamkörner; geröstet

1 EL Sojasauce

1 EL Sesamöl; zusätzlich


\end{multicols}

Überschüssiges Fett und Sehnen des Rumpsteaks entfernen und das Fleisch quer zur Faser in dünne Streifen schneiden; mit geriebener Orangenschale und Sesamöl in eine Glas- oder Kunststoffschale legen. Mit den Fingern die Gewürze ins Fleisch einreiben; Schale mit Klarsichtfolie abdecken und 20 Minuten in den Kühlschrank stellen. Orangensaft und Maisstärke mischen und beiseite stellen.

 Öl im Wok oder in einer Pfanne erhitzen, das Fleisch portionsweise dazugeben und bei starker Hitze unter Rühren bräunen; anschließend auf einen Teller geben und warm halten.

 Die Pfanne wieder erhitzen. Zwiebeln, Möhren und Bohnen hineingeben und bei starker Hitze 2 Minuten unter Rühren garen, so dass das Gemüse noch Biss hat.

Die Maisstärkemischung hinzufügen und durchrühren, bis die Sauce andickt.

Fleisch in die Pfanne geben. Sesamkörner, Sojasauce und zusätzliches Sesamöl zugeben und erhitzen. Sofort servieren.

 Tip: : Sesamöl hat ein intensives Aroma und soll nur vorsichtig -
zum Marinieren oder nachträglichem Würzen der fertigen Speisen - verwendet werden.


{\bfseries Menge:} 4 Portionen

{\bfseries Quelle:} Anne Wilson  Wok-\& Pfannengerichte 

} 

%----------------------
\nopagebreak{ 
\subsection{Sauerbraten}

\index{Alkohol@{Alkohol\/}!Sauerbraten@{Sauerbraten\/}}
\index{Braten@{Braten\/}!Sauerbraten@{Sauerbraten\/}}
\index{Möhren@{Möhren\/}!Sauerbraten@{Sauerbraten\/}}
\index{Rindfleisch@{Rindfleisch\/}!Sauerbraten@{Sauerbraten\/}}
\index{Sellerie@{Sellerie\/}!Sauerbraten@{Sauerbraten\/}}
\index{vorbereiten@{vorbereiten\/}!Sauerbraten@{Sauerbraten\/}}
}{
\begin{multicols}{2}



1 kg Rindsbraten

\textit{Beize}



5 dl Rotwein

1 dl Rotweinessig

1 dl Wasser

5 1/2 TL Salz

2 Lorbeerblätter

2 Nelken

1 TL Wacholderbeeren

1 Zwiebel, geviertelt

2 Möhren, gewürfelt

1/3 Sellerie, gewürfelt

1 TL Thymian

4 Pfefferkörner

1/2 Handvoll Rosinen

\textit{Zum Anbraten}



2 EL Pflanzenfett

Knochen

\textit{Sauce}



4 dl Beize, Etwa

1 dl Wasser

2 EL geröstetes Mehl (evtl. mehr)

Salz oder Streuwürze

Pfeffer

1 dl Kaffeerahm oder Rahm


\end{multicols}

Alle Zutaten für die Beize aufkochen, 5 Minuten köcheln, erkalten lassen.

Das Fleisch in einen glasierten Topf, eine Schüssel oder ein Glasgefäß geben, die Beize darübergießen, 4-6
Tage ziehen lassen (eventuell im Kühlschrank). Wichtig: Das Fleisch sollte mit der Beize bedeckt sein, sonst muss es
jeden Tag gewendet werden!

Fleisch aus der Beize nehmen, abtropfen lassen, mit Haushaltpapier trocken tupfen, eventuell im Mehl wenden.
Beize absieben und verwahren. Pflanzenfett im Brattopf erhitzen, den Braten rundum stark anbraten, Knochen kurz mitbraten. Abgetropfte Gemüsewürfel (aus der Beize) beigeben, etwas Beize dazugeben, würzen und zugedeckt 2-21/2 Stunden auf kleinem Feuer garschmoren.

Braten aus dem Bräter nehmen und ruhen lassen.

Beize absieben, separat in einer Pfanne aufkochen, filtrieren. Mehl mit dem kalten Wasser anrühren, dazugeben und unter ständigem Rühren aufkochen. Vor dem Servieren die Sauce mit Kaffeerahm oder Rahm verfeinern.

Tipp:  Den Braten mit geviertelten oder halbierten, weich gekochten Äpfeln oder Birnen garnieren, die nach Belieben mit Preiselbeerkonfitüre gefüllt werden können. Rotkraut, Kastanien, Rosenkohl, Spätzli oder Kartoffelstock passen sehr gut dazu.


{\bfseries Menge:} 4 Portionen

{\bfseries Quelle:} www.kochfreunde.de, abgewandelt von Frauke 

} 

%----------------------
\nopagebreak{ 
\subsection{Sauerbraten vom Hasen mit gefülltem Bratapfel und Schneekartoffeln}

\index{Kaninchen@{Kaninchen\/}!Sauerbraten vom Hasen mit gefülltem Bratapfel und Schneekartoffeln@{Sauerbraten vom Hasen mit gefülltem Bratapfel und Schneekartoffeln\/}}
\index{vorbereiten@{vorbereiten\/}!Sauerbraten vom Hasen mit gefülltem Bratapfel und Schneekartoffeln@{Sauerbraten vom Hasen mit gefülltem Bratapfel und Schneekartoffeln\/}}
\index{Wild@{Wild\/}!Sauerbraten vom Hasen mit gefülltem Bratapfel und Schneekartoffeln@{Sauerbraten vom Hasen mit gefülltem Bratapfel und Schneekartoffeln\/}}
}{
\begin{multicols}{2}

\textit{Fleisch}



4 Hasenkeulen mit Knochen

nach Geschmack etwas magerer geräucherter Speck

\textit{für die Marinade}



1 l kräftiger tockener Rotwein

100 ml Weinessig

50 g Knollensellerie

50 g Möhren

100 g Zwiebeln

(alles gewürfelt)

20 Wacholderbeeren

1 Zweig Rosmarin

10 schwarze Pfefferkörner

3 Lorbeerblätter

20 g Salz

\textit{außerdem}



30 g Butterschmalz

2 EL Tomatenmark

60 g Butter

1 EL Johannisbeergelee

Salz und Pfeffer

\textit{für die Bratäpfel}



4 kleine Äpfel

2 EL Rosinen

2 EL Mandelblättchen

30 g Butter

100 g Marzipan-Rohmasse

1 Glas Weißwein


\end{multicols}

Die Zutaten für die Marinade kurz aufkochen, dann abkühlen lassen. In diesem Sud die Hasenkeulen 4 Tage im Kühlschrank ziehen lassen.

Dann die Keulen aus dem Sud nehmen. Die Marinade durch ein Haarsieb in eine Schüssel passieren. Gemüse und Gewürze abtropfen lassen. Die Hasenkeulen in Butterschmalz rundherum anbraten. In einem Bräter das abgetropfte Gemüse (aus der Marinade) in wenig Butterschmalz (und nach Geschmack mit dem gewürfelten Speck) anrösten. Das Tomatenmark dazu geben, kurz mit anrösten lassen, dann mit wenig Marinade ablöschen und rühren, bis alle Flüssigkeit verdampft ist. Diesen Vorgang drei mal wiederholen, dann die restliche Marinade angießen und die angebratenen Hasenkeulen hinein geben. Im Backofen bei 175 Grad rund 90 Minuten schmoren. In dieser Zeit mehrfach mit dem Bratenfond begießen.

Danach die Sauce durch ein Sieb passieren, Johannisbeergelee unterrühren und aufkochen lassen. Nun vom Herd nehmen und mit dem Schneebesen zur Bindung kalte Butterwürfel unterrühren. Mit Salz und Pfeffer abschmecken.

Aus den Äpfeln mit einem Ausstecher das Kerngehäuse entfernen. In halber Höhe die Apfelschale rundherum mit einem Messer einschneiden. Die Äpfel mit Rosinen und Mandelblättchen (vermischt) füllen, etwas Weißwein auf die Füllung gießen, mit etwas Butter verschließen und mit einer kleinen Scheibe Marzipan abdecken. Die Äpfel auf ein Blech legen und 10 Minuten vor Ende der Garzeit zu den Hasenkeulen in den Ofen geben.

Schneekartoffeln: 1000 g Salzkartoffeln durch die Püree- oder Spätzlepresse drücken.

Dazu passt eine kräftige badische Spätburgunder Spätlese.


{\bfseries Menge:} 4 Portionen

{\bfseries Quelle:} WDR Spitzenkochtip vom 29.111 

} 

%----------------------
\nopagebreak{ 
\subsection{Sauerbraten vom Lamm mit dicken Bohnen und Rosmarinkartoffeln}

\index{Alkohol@{Alkohol\/}!Sauerbraten vom Lamm mit dicken Bohnen und Rosmarinkartoffeln@{Sauerbraten vom Lamm mit dicken Bohnen und Rosmarinkartoffeln\/}}
\index{Lamm@{Lamm\/}!Sauerbraten vom Lamm mit dicken Bohnen und Rosmarinkartoffeln@{Sauerbraten vom Lamm mit dicken Bohnen und Rosmarinkartoffeln\/}}
\index{vorbereiten@{vorbereiten\/}!Sauerbraten vom Lamm mit dicken Bohnen und Rosmarinkartoffeln@{Sauerbraten vom Lamm mit dicken Bohnen und Rosmarinkartoffeln\/}}
}{
\begin{multicols}{2}

\textit{Lammkeule}



800 g Lammkeule ohne Knochen

\textit{für die Marinade}



1 große Zwiebel

2 Möhren

150 g Sellerie

1/2 Stange Porree

(alle Gemüsezutaten in Würfeln)

300 ml Rotwein

100 ml Rotweinessig

2 EL Balsamico

2 Lorbeerblätter

10 Pfefferkörner (zerdrückt)

5 Wacholderbeeren (zerdrückt)

5 Pimentkörner (zerdrückt)

\textit{für die Sauce}



1 EL Balsamico

2 EL Rübenkraut

Beurre manié (je 1 EL Butter und Mehl gut verkneten)

\textit{für die Bohnen}



400 g Kerne von dicken Bohnen (aus der Tiefkühltruhe)

1 große Tomate

1 kleine Zwiebel

2 EL Olivenöl

\textit{für die Rosmarinkartoffeln}



ca. 16 kleine neue Kartoffeln

2 EL Olivenöl

Salz und Pfeffer

2 Rosmarinzweige

\textit{außerdem}



Salz und Pfeffer aus der Mühle


\end{multicols}

Alle Zutaten für die Marinade in eine Schüssel geben. Das Fleisch von Fett und Sehnen befreien, mit Küchengarn zu einem Braten binden und diesen roh in die Marinade legen. Für mindestens zwei Tage in den Kühlschrank geben.

Nun das Fleisch aus der Marinade nehmen und mit Küchenkrepp abtrocknen. Pfeffern, salzen und in heißem Olivenöl rundherum kräftig anbraten. Dann die abgetropften Gemüse aus der Marinade dazu geben und 2 Minuten mit anbraten. Alles mit der Marinadeflüssigkeit ablöschen, kurz aufkochen und den Braten mindestens 90 Minuten bei geringer Hitze mit Deckel schmoren.

Nun das Fleisch herausnehmen und warm stellen. Den Bratenfond durch ein feines Sieb passieren und dann aufkochen. Die Sauce mit Rübenkraut, Balsamico, Salz und Pfeffer abschmecken. Auf Wunsch mit etwas Kalbsfond verfeinern.

Die Bohnenkerne rund drei Minuten in Salzwasser blanchieren, dann kalt abschrecken. Die Zwiebel fein würfeln und in Öl glasig anschwitzen. Die Bohnenkerne dazu geben und kurz durchschwitzen lassen. Unterdessen die Tomate schälen, entkernen und würfeln und 2 Minuten mit den Bohnen gar schwenken. Mit Salz und Pfeffer würzen.

Die Kartoffeln in Salzwasser mit etwas Kümmel in der Schale gar kochen. Dann die Kartoffeln halbieren und ungeschält in heißem Olivenöl knusprig braten. Gegen Ende 4 Rosmarinzweige mit braten lassen. Mit Salz und grobem Pfeffer würzen.

Auf jeden Teller einen kleinen Spiegel der Sauce gießen. Darauf je 2 Scheiben Sauerbraten geben. Daneben jeweils die dicken Bohnen und die Rosmarinkartoffeln platzieren. Mit gebratenen Rosmarinzweigen dekorieren. Dazu passt ein leichter Rotwein.


{\bfseries Menge:} 4 Portionen

{\bfseries Quelle:} WDR Spitzenkochtip vom 2832 

} 

%----------------------
\nopagebreak{ 
\subsection{Schweinebraten mit Backpflaumen}

\index{Alkohol@{Alkohol\/}!Schweinebraten mit Backpflaumen@{Schweinebraten mit Backpflaumen\/}}
\index{Apfel@{Apfel\/}!Schweinebraten mit Backpflaumen@{Schweinebraten mit Backpflaumen\/}}
\index{Backpflaumen@{Backpflaumen\/}!Schweinebraten mit Backpflaumen@{Schweinebraten mit Backpflaumen\/}}
\index{Kartoffeln@{Kartoffeln\/}!Schweinebraten mit Backpflaumen@{Schweinebraten mit Backpflaumen\/}}
\index{Schweinefleisch@{Schweinefleisch\/}!Schweinebraten mit Backpflaumen@{Schweinebraten mit Backpflaumen\/}}
}{
\begin{multicols}{2}



1/2 TL Zimt

1/2 TL Nelkenpulver

1/5 l Rotwein (ersatzweise Apfelsaft)

250 g Backpflaumen ohne Stein

1.7 kg Schweineschulter mit Schwarte

Salz

Pfeffer

Majoran

1/4 l Brühe

3 säuerliche Äpfel

1 kg kleine Kartoffeln

2 EL Zucker


\end{multicols}

Zimt und Nelkenpulver mit dem Rotwein verrühren. Backpflaumen über Nacht darin einweichen.

Die Schweineschulter rundherum mit Salz, Pfeffer und Majoran einreiben. In einen Bräter legen. Brühe zugießen. In den Backofen schieben, auf 175 Grad 90 Minuten braten.

Apfelviertel und geschälte Kartoffelhälften mit den Backpflaumen in den Bräter legen. Mit Rotwein begießen und mit Zucker bestreuen. Eine Stunde weiterbraten.

Dazu passen Kartoffelknödel.


{\bfseries Bemerkung:} Eiweiß: 52 g, Fett: 65 g, Kohlenhydrate: 65 g, kcal: 1130 

{\bfseries Menge:} 6 Portionen

{\bfseries Quelle:} Brigitte Rezepte: Die 300 beliebtesten Sammelrezepte 

} 

%----------------------
\nopagebreak{ 
\subsection{Schweinefilet mit roten Zwiebeln}

\index{Schweinefleisch@{Schweinefleisch\/}!Schweinefilet mit roten Zwiebeln@{Schweinefilet mit roten Zwiebeln\/}}
\index{Zwiebeln@{Zwiebeln\/}!Schweinefilet mit roten Zwiebeln@{Schweinefilet mit roten Zwiebeln\/}}
}{
\begin{multicols}{2}



400 g rote Zwiebeln

3 Knoblauchzehen

4 EL Olivenöl

2 Schweinefilets à 350 g

Salz, Pfeffer

150 ml Rotwein

100 ml Portwein

1 EL flüssiger Honig

2 Lorbeerblätter

1 Zweig Rosmarin

\textit{Dazu:}



Zerdrückte Pellkartoffeln

\bild{image/cookbook_11.jpg}


\end{multicols}

Die Zwiebeln pellen und in grobe Streifen oder Ringe schneiden. Knoblauch pellen und fein hacken. Die Schweinefilets in Medaillons schneiden und in 2 EL Öl von jeder Seite ca. 3 bis 4 Minuten anbraten. Aus der Pfanne nehmen und das restliche Öl erhitzen. Darin die Zwiebeln anbraten und den Honig leicht karamellisieren lassen.

Mit Portwein ablöschen und den Rotwein, sowie Lorbeer hinzugeben. Rosmarin hineinzupfen, das Fleisch darauf verteilen und mit Backpapier bedeckt ca. 10 Minuten leicht simmern lassen.


{\bfseries Menge:} 4 Portionen

{\bfseries Quelle:} www.vox.de: Schmeckt nicht, gibt's nicht 

} 

%----------------------
\nopagebreak{ 
\subsection{Schweinekarree mit Senfkruste}

\index{Schweinefleisch@{Schweinefleisch\/}!Schweinekarree mit Senfkruste@{Schweinekarree mit Senfkruste\/}}
\index{vorbereiten@{vorbereiten\/}!Schweinekarree mit Senfkruste@{Schweinekarree mit Senfkruste\/}}
}{
\begin{multicols}{2}



2 EL körniger Senf

2 Zweige Rosmarin

2 Zweige Thymian

1 TL gestoßene Pfefferkörner

1.8 kg Schweinerücken mit Fett am Knochen

Salz

150 g Zwiebeln

50 g Möhren

50 g Sellerie

2 EL Öl

300 ml lieblicher Weißwein


\end{multicols}

Senf, zerstoßenen Pfeffer, Öl, Salz, Rosmarin und Thymian mischen.

Das Fett des Schweinerückens mit einem kleinen Messer kreuzweise einritzen und mit der Würzpaste einreiben. Zwiebeln, Möhren und Sellerie grob zerkleinern und im Bräter etwas rösten. Dann den Schweinerücken darauf setzen. Weißwein angießen. In den auf 200$^\circ$C (Umluft: 180 $^\circ$C, Gas: Stufe 3)vorgeheizten Ofen auf der unteren Schiene garen.

Nach 20 Minuten die Temperatur auf 175 $^\circ$C (150 $^\circ$C Umluft, Gas: Stufe 2) reduzieren und weitere
40 Minuten garen. Anschließend den Braten aus dem Ofen nehmen.

Den Braten vom Knochen lösen und in Scheiben geschnitten mit Waldpilzen und Senf-Kartoffeln servieren.


{\bfseries Menge:} 4 Portionen

{\bfseries Quelle:} www.vox.de: Schmeckt nicht gibt's nicht, 2006-11-27 

} 

%----------------------
\nopagebreak{ 
\subsection{Schweinemedaillons auf Rosmarinspieß}

\index{P6@{P6\/}!Schweinemedaillons auf Rosmarinspieß@{Schweinemedaillons auf Rosmarinspieß\/}}
\index{Rosmarin@{Rosmarin\/}!Schweinemedaillons auf Rosmarinspieß@{Schweinemedaillons auf Rosmarinspieß\/}}
\index{Schweinefleisch@{Schweinefleisch\/}!Schweinemedaillons auf Rosmarinspieß@{Schweinemedaillons auf Rosmarinspieß\/}}
}{
\begin{multicols}{2}



12 gleichmäßige Schweinemedaillons à 70 g

6 dicke Rosmarinzweige

Salz, Pfeffer

\textit{Dazu:}



Grüne Bohnen italienisch

Oliven-Serviettenknödel

\bild{image/cookbook_12.jpg}


\end{multicols}

Von den Rosmarinzweigen ca. 3 cm unterhalb der Spitze die Nadeln entfernen. Danach jeweils 2 Medaillons auf den Rosmarinzweig stecken. Medaillons mit Salz und Pfeffer würzen, von allen Seiten anbraten und 5 Minuten im Ofen bei 180$^\circ$C garen. Bis zum Anrichten ruhen lassen.


{\bfseries Menge:} 6 Portionen

{\bfseries Quelle:} www.vox.de: Schmeckt nicht, gibt's nicht 

} 

%----------------------
\nopagebreak{ 
\subsection{Süßsaures Schweinefleisch}

\index{asiatisch@{asiatisch\/}!Süßsaures Schweinefleisch@{Süßsaures Schweinefleisch\/}}
\index{einfach@{einfach\/}!Süßsaures Schweinefleisch@{Süßsaures Schweinefleisch\/}}
\index{Schweinefleisch@{Schweinefleisch\/}!Süßsaures Schweinefleisch@{Süßsaures Schweinefleisch\/}}
}{
\begin{multicols}{2}



4 Schweineschnitzel, dünn (ca. 400 g)

2 rote Paprikaschoten

Salz

Reis

1 Bund Lauchzwiebel

414 ml Mandarin-Orangen (1 Dose)

2 EL Öl

Pfeffer

Paprika, edelsüß

4 EL Sojasauce

2 EL Tomatenketchup

1/4 l Gemüsebrühe

200 g Erbsen (TK)

1 EL Speisestärke

Petersilie


\end{multicols}

Fleisch waschen, trockentupfen und in Streifen schneiden. Paprika putzen, waschen und in Stücke schneiden. Reis in kochendem Salzwasser garen. In der Zwischenzeit Lauchzwiebeln putzen, waschen und in Stücke schneiden. Mandarinenspalten abtropfen lassen, Fruchtsaft auffangen. Öl in einer hohen Pfanne/Topf erhitzen, Fleisch kräftig anbraten. Mit Salz, Pfeffer und Paprika würzen. Paprikaschoten und Lauchzwiebeln kurz mit anbraten. Sojasauce und Ketchup unterrühren. Mit Brühe und 4 EL Fruchtsaft ablöschen. Erbsen darin aufkochen lassen. Speisestärke in 4 EL Wasser glattrühren, Sauce damit leicht andicken. Nochmals aufkochen und mit Salz und Pfeffer würzen. Mandarinen unterheben. Mit Reis anrichten. Mit Petersilie garnieren und mit Paprika bestäuben.


{\bfseries Menge:} 4 Portionen

{\bfseries Quelle:} Prisma (Zeitschrift), kochmeister.com 

} 

%----------------------
\nopagebreak{ 
\subsection{Tims Beef Wellington}

\index{Alkohol@{Alkohol\/}!Tims Beef Wellington@{Tims Beef Wellington\/}}
\index{Rindfleisch@{Rindfleisch\/}!Tims Beef Wellington@{Tims Beef Wellington\/}}
\index{Rotwein@{Rotwein\/}!Tims Beef Wellington@{Tims Beef Wellington\/}}
\index{Schalotten@{Schalotten\/}!Tims Beef Wellington@{Tims Beef Wellington\/}}
}{
\begin{multicols}{2}



10 Schalotten

6 Knoblauchzehen

400 g Champignons

1 kg Rinderfilet aus der Mitte

Salz

Pfeffer

4 EL Öl

2 EL Honig

2 Zweige Rosmarin

300 ml Rotwein

3 EL grober Senf

4 Blätter TK-Blätterteig (10 x 22 cm), aufgetaut

2 Eigelbe

\bild{image/cookbook_13.jpg}


\end{multicols}

Fleisch von Sehnen und Fett befreien. Schalotten und Knoblauch pellen, aber nicht hacken. Champignons putzen und vierteln. Rinderfilet salzen und pfeffern.

2 EL Öl in einer Pfanne erhitzen und das Filet von allen Seiten kräftig anbraten. Herausnehmen. und in eine ofenfeste Form legen. Die Schalotten, Knoblauch und Champignons zusammen unter Rühren im restlichen heißen Öl 3 bis 4 Minuten braten. Rosmarinnadeln abzupfen, hacken und mit dem Honig dazugeben.

Mit dem Rotwein ablöschen, aufkochen lassen. Filet mit dem Senf einstreichen.

Die Pilzmasse zum Fleisch in die Form geben. Eigelbe verquirlen und den Rand der Form bestreichen.

Blätterteig-Blätter aufeinander legen und mit dem Nudelholz knapp größer als die Maße der Form ausrollen. Den Teig über die Form legen und rundherum fest andrücken. Teig mit dem restlichen Eigelb einstreichen. Im vorgeheizten Ofen bei 200$^\circ$C (Umluft 180 $^\circ$C, Gas Stufe 3-4) auf der zweiten Schiene von unten ca. 20 bis 25 Minuten backen.


{\bfseries Menge:} 4 Portionen

{\bfseries Quelle:} www.vox.de: Schmeckt nicht, gibt's nicht 

} 

%----------------------
\nopagebreak{ 
\subsection{Zucchini--Tomaten--Gemüse mit Hirse und Lamm}

\index{Hauptgericht@{Hauptgericht\/}!Zucchini--Tomaten--Gemüse mit Hirse und Lamm@{Zucchini--Tomaten--Gemüse mit Hirse und Lamm\/}}
\index{Hirse@{Hirse\/}!Zucchini--Tomaten--Gemüse mit Hirse und Lamm@{Zucchini--Tomaten--Gemüse mit Hirse und Lamm\/}}
\index{Lamm@{Lamm\/}!Zucchini--Tomaten--Gemüse mit Hirse und Lamm@{Zucchini--Tomaten--Gemüse mit Hirse und Lamm\/}}
\index{P1@{P1\/}!Zucchini--Tomaten--Gemüse mit Hirse und Lamm@{Zucchini--Tomaten--Gemüse mit Hirse und Lamm\/}}
\index{Tomaten@{Tomaten\/}!Zucchini--Tomaten--Gemüse mit Hirse und Lamm@{Zucchini--Tomaten--Gemüse mit Hirse und Lamm\/}}
\index{Zucchini@{Zucchini\/}!Zucchini--Tomaten--Gemüse mit Hirse und Lamm@{Zucchini--Tomaten--Gemüse mit Hirse und Lamm\/}}
}{
\begin{multicols}{2}



5 EL Hirse (50 g)

Salz

1 Zucchini

1 Möhre

1 kleine Zwiebel

1 Knoblauchzehe

125 g geschälte Tomaten aus der Dose

1 Messerspitze Chiligewürz

1 Messerspitze Kumin (Kreuzkümmel)

1 Messerspitze Zimt

1 Messerspitze Koriander

2 dünne Lammkoteletts ohne Fett (oder 80 g Lammfilet)

1 TL Olivenöl


\end{multicols}

1. Hirse mit gut der doppelten Menge Salzwasser zugedeckt zum Kochen bringen  und bei geringer Hitze 20 Minuten quellen lassen.
Zuviel Flüssigkeit zum Schluss offen verdampfen lassen.

2. Zucchini, Möhre und Zwiebel kleinschneiden, Knoblauch hacken, mit den Tomaten, Gewürzen und etwas Salz in einem Topf bei geringer Hitze 7 bis 10 Minuten kochen.

3. Lamm mit Salz und Pfeffer würzen. Eine Pfanne erhitzen, Öl hineingeben und die Koteletts auf jeder Seite etwa 2 Minuten braten. Fleisch herausnehmen und warm halten.

4. Das Gemüse kurz in der Pfanne schwenken, eventuell nachwürzen und mit der Hirse zum Fleisch anrichten.


{\bfseries Bemerkung:} Fett: 11 g, BE: 4 

{\bfseries Menge:} 1 Portion

{\bfseries Quelle:} Brigitte: Ideal-Diät 1999 

} 

%----------------------
\nopagebreak{ 
\subsection{Überbackene Medaillons mit Thymianbirnen}

\index{Birnen@{Birnen\/}!Überbackene Medaillons mit Thymianbirnen@{Überbackene Medaillons mit Thymianbirnen\/}}
\index{Gorgonzola@{Gorgonzola\/}!Überbackene Medaillons mit Thymianbirnen@{Überbackene Medaillons mit Thymianbirnen\/}}
\index{Käse@{Käse\/}!Überbackene Medaillons mit Thymianbirnen@{Überbackene Medaillons mit Thymianbirnen\/}}
\index{Schweinefleisch@{Schweinefleisch\/}!Überbackene Medaillons mit Thymianbirnen@{Überbackene Medaillons mit Thymianbirnen\/}}
}{
\begin{multicols}{2}



2 Birnen

3 Stiele Thymian

4 längliche Schalotten

12 Scheiben Bacon

12 Schweinemedaillons (a 60 g)

12 Holzstäbchen

120 g Gorgonzola

2 EL Öl

125 ml Cidre

125 ml Brühe

40 g kalte Butter


\end{multicols}

Birnen achteln und entkernen. Thymian abregeln und fein hacken.
Schalotten längs in feine Streifen schneiden.

Schweinemedaillons mit Speckstreifen umwickeln, mit Holzstäbchen fixieren und mit Pfeffer
würzen. Öl in einer Pfanne erhitzen und Medaillons darin von jeder Seite 1 Minute bei starker Hitze anbraten.

Auf ein Blech legen und den Gorgonzola darauf verteilen. Im vorgeheizten Ofen bei 200$^\circ$C auf der zweiten Schien von unten 8 Minuten braten (Umluft nicht empfehlenswert).

Öl in der Pfanne erhitzen, Birnen und Schalotten darin bei starker Hitze 2 Minuten braten, Thymian zugeben, salzen, pfeffern
und mit Cidre und Brühe ablöschen. Um 1/3 einkochen lassen, dann kalte butte in kleine Stücken dazugeben.

Holzstäbchen entfernen und die Medaillons mit den Thymianbirnen servieren.


{\bfseries Bemerkung:} Eiweiß: 49 g, Fett: 33 g, Kohlenhydrate: 9 g, kJ: 2233, kcal: 533 

{\bfseries Menge:} 4 pro Portion

{\bfseries Quelle:} Essen und Trinken für jeden Tag 04/2007 

} 

\pagebreak 
 
%----------------------------------------------------
\nopagebreak{ 
\section{FISCH} 

\subsection{Fenchelnudeln mit Krabbenfleisch}

\index{Fenchel@{Fenchel\/}!Fenchelnudeln mit Krabbenfleisch@{Fenchelnudeln mit Krabbenfleisch\/}}
\index{Fisch@{Fisch\/}!Fenchelnudeln mit Krabbenfleisch@{Fenchelnudeln mit Krabbenfleisch\/}}
\index{Hauptgericht@{Hauptgericht\/}!Fenchelnudeln mit Krabbenfleisch@{Fenchelnudeln mit Krabbenfleisch\/}}
\index{Krabben@{Krabben\/}!Fenchelnudeln mit Krabbenfleisch@{Fenchelnudeln mit Krabbenfleisch\/}}
\index{Lauchzwiebel@{Lauchzwiebel\/}!Fenchelnudeln mit Krabbenfleisch@{Fenchelnudeln mit Krabbenfleisch\/}}
\index{P1@{P1\/}!Fenchelnudeln mit Krabbenfleisch@{Fenchelnudeln mit Krabbenfleisch\/}}
\index{Pasta@{Pasta\/}!Fenchelnudeln mit Krabbenfleisch@{Fenchelnudeln mit Krabbenfleisch\/}}
}{
\begin{multicols}{2}



1 Fenchel

1 Tasse Gemüsebrühe

1 Lachzwiebel

1/2 Bund glatte Petersilie oder etwas Fenchelkraut

60 g Nudeln

3 TL Tomatenmark

2 EL Kapern

50 g Krabbenfleisch

1 1/2 TL Olivenöl

frisch gemahlener Pfeffer


\end{multicols}

1. Nudeln bissfest garen.

2. Den Fenchel in Streifen schneiden und 3 Minuten in der Brühe kochen. Lauchzwiebel kleinschneiden, zufügen und 2 Minuten offen weitergaren, Flüssigkeit dabei etwas einkochen.

3. Petersilie oder Fenchelkraut grob hacken, mit den gekochten Nudeln, dem Tomatenmark und den Kapern in den Topf geben und einmal aufkochen.

4. Krabben und Öl unterheben, mit Pfeffer würzen, nicht mehr kochen.


{\bfseries Bemerkung:} Fett: 11 g, BE: 3 

{\bfseries Menge:} 1 pro Portion

{\bfseries Quelle:} Brigitte: Idealdiät 1999 

} 

%----------------------
\nopagebreak{ 
\subsection{Fischfilet auf Tomaten gedünstet mit Kräuterreis}

\index{Fisch@{Fisch\/}!Fischfilet auf Tomaten gedünstet mit Kräuterreis@{Fischfilet auf Tomaten gedünstet mit Kräuterreis\/}}
\index{Hauptgericht@{Hauptgericht\/}!Fischfilet auf Tomaten gedünstet mit Kräuterreis@{Fischfilet auf Tomaten gedünstet mit Kräuterreis\/}}
\index{P1@{P1\/}!Fischfilet auf Tomaten gedünstet mit Kräuterreis@{Fischfilet auf Tomaten gedünstet mit Kräuterreis\/}}
\index{Tomaten@{Tomaten\/}!Fischfilet auf Tomaten gedünstet mit Kräuterreis@{Fischfilet auf Tomaten gedünstet mit Kräuterreis\/}}
}{
\begin{multicols}{2}



3 EL Reis

Salz

1 Knoblauchzehe

1 Lauchzwiebel

200 g Dosentomaten mit Saft (1/2 kleine Dose)

100 g TK-Seefisch (z.B. Seelachs)

frisch gemahlener Pfeffer

2 TL Butter oder Margarine

etwas Schnittlauch oder Petersilie


\end{multicols}

Den Reis in Salzwasser kochen und abgießen.

Inzwischen den Knoblauch abziehen und hacken. Die Lauchzwiebel putzen und in Ringe schneiden.

Beides mit den Tomaten in einen kleinen Topf geben. Das gefrorene Fischfilet darauf legen.

Alles mit Salz und Pfeffer würzen, mit Butter- oder Margarineflöckchen belegen und zugedeckt zum Kochen bringen. Die Hitze reduzieren und den Fisch in ca.
8-10 Minuten garen.

Kräuter fein schneiden, unter den abgegossenen Reis mischen und mit dem Fischfilet mit Tomatengemüse anrichten.

Würz-Tipp: Statt Pfeffer Harissa oder Curry in das Gemüse rühren und
mit frischem Koriander anrichten.

Tipps: Von den Lauchzwiebeln auch den dunkelgrünen Teil mitverwenden.
Wer frisches Fischfilet kochen möchte, legt es nur die letzten 2-4 Minuten in die Tomatensauce. Die Garzeit hängt von der Dicke der Fischstücke ab.


{\bfseries Bemerkung:} Fett: 9 g 

{\bfseries Menge:} 1 pro Portion

{\bfseries Quelle:} Brigitte: Einsteigerdiät 2003 

} 

%----------------------
\nopagebreak{ 
\subsection{Fischfilet auf Tomaten--Lauch}

\index{Fisch@{Fisch\/}!Fischfilet auf Tomaten--Lauch@{Fischfilet auf Tomaten--Lauch\/}}
\index{Hauptgericht@{Hauptgericht\/}!Fischfilet auf Tomaten--Lauch@{Fischfilet auf Tomaten--Lauch\/}}
\index{Lauchzwiebel@{Lauchzwiebel\/}!Fischfilet auf Tomaten--Lauch@{Fischfilet auf Tomaten--Lauch\/}}
\index{P1@{P1\/}!Fischfilet auf Tomaten--Lauch@{Fischfilet auf Tomaten--Lauch\/}}
\index{Tomaten@{Tomaten\/}!Fischfilet auf Tomaten--Lauch@{Fischfilet auf Tomaten--Lauch\/}}
}{
\begin{multicols}{2}



Salz

Estragon

1 TL-Spitze Rosenpaprika

7 EL Kartoffelpüreeflocken

etwas glatte Petersilie

2 Lauchzwiebeln

4 Tomaten

1 EL Kürbiskerne

125 g Fischfilet (Seelachs oder Kabeljau)

frisch gemahlener Pfeffer

1 TL Senf

4 TL Crème fraîche


\end{multicols}

1. 180 ml Salzwasser mit dem Estragon sehr langsam erhitzen (dann entfaltet sich das Kräuteraroma besser). Kurz vor dem Essen die Kartoffelpüreeflocken hineinrühren und die gehackte Petersilie untermengen.

2. Inzwischen Lauchzwiebeln in Ringe und Tomaten in Scheiben schneiden. Kürbiskerne hacken. Einen Topf erhitzen, die Lauchzwiebeln hineingeben, salzen und 1 Minute rühren. Dann die Tomaten zufügen und zugedeckt 1 bis 2 Minuten dünsten.

3. Den Fisch mit Salz und Pfeffer einreiben. Eine Seite mit Senf und Crème fraîche bestreichen und den Fisch mit dieser Seite nach oben auf das Gemüse setzen. Mit Kürbiskernen bestreuen.
Zugedeckt bei mittlerer Hitze etwa 3 bis 4 Minuten garen.

4. Fischfilet auf einem Teller warm stellen. Zuviel Flüssigkeit offen einkochen. Gemüse und Püree zum Fisch anrichten.

Tipp: Bei Tiefkühl-Fisch die Packungsangabe beachten, den
gefrorenen Fisch würzen und gleich mit den Tomaten in den Topf geben. Er ist bei mittlerer Hitze nach etwa 10 Minuten gar.


{\bfseries Bemerkung:} Fett: 13 g, BE: 3 

{\bfseries Menge:} 1 pro Portion

{\bfseries Quelle:} Brigitte: Idealdiät 1999 

} 

%----------------------
\nopagebreak{ 
\subsection{Rotbarschfilet in Sesamkruste}

\index{Fisch@{Fisch\/}!Rotbarschfilet in Sesamkruste@{Rotbarschfilet in Sesamkruste\/}}
\index{Rotbarsch@{Rotbarsch\/}!Rotbarschfilet in Sesamkruste@{Rotbarschfilet in Sesamkruste\/}}
\index{Sesam@{Sesam\/}!Rotbarschfilet in Sesamkruste@{Rotbarschfilet in Sesamkruste\/}}
}{
\begin{multicols}{2}



800 g Rotbarschfilet

1 Zitrone

Salz

Pfeffer

1 Ei

1 TL Sojasauce

1 TL Honig

1 TL geriebener frischer Ingwer

2 EL Paniermehl

3 EL Sesamsaat

7 EL Sesamöl


\end{multicols}

Den Fisch waschen, trocken tupfen, die Zitrone auspressen, den Fisch mit Zitronensaft einreiben, mit Salz und Pfeffer bestreuen. Das Ei in einen Teller geben, mit der Sojasauce, dem Honig und dem Ingwer verrühren. In einem
anderen Teller das Paniermehl und Sesam mischen. Das Fischfilet zuerst in etwas Mehl, dann in Ei, dann in der Sesammischung wälzen, letzteres wiederholen. Die Paniermischung gut andrücken. Das Öl in einer großen Pfanne
erhitzen und den Fisch darin vorsichtig auf beiden Seiten knusprig braten. Nach 10 Minuten den Fisch warm stellen.

Dazu passt scharfes Möhrengemüse und Reis.


{\bfseries Menge:} 4 Portionen

{\bfseries Quelle:} Internet 

} 

